% Tipe dokumen adalah report dengan satu kolom. 
% Menggatur setting halaman 
\documentclass[12pt, a4paper, onecolumn, oneside, final]{report}
\makeatother
\usepackage{amsmath}
\usepackage{float}
\usepackage{array}
\usepackage{longtable}
\usepackage{adjustbox}
\usepackage{gensymb}
% Load konfigurasi LaTeX untuk tipe laporan thesis
\usepackage{if_ithb}
\usepackage{enumitem}
\usepackage{multirow}
\usepackage{tikz}
\usetikzlibrary{matrix}
% Daftar pemenggalan suku kata dan istilah dalam LaTeX
\include{hype.indonesia}

% Variabel baru untuk menyimpan nomor halaman
\newcounter{originalpagenumber}%

% Awal bagian penulisan laporan
\begin{document}

	% Sampul Laporan
	\begin{titlepage}
\begin{center}
	\onehalfspacing
	{\large \bfseries PENERAPAN METODE HOUGH TRANSFORM UNTUK DETEKSI PLAT NOMOR KENDARAAN DAN HISTOGRAM OF ORIENTED GRADIENT UNTUK PENGENALAN KARAKTER PLAT NOMOR KENDARAAN\\
	\vspace{1.5cm}
	 \large TUGAS AKHIR}\\
           Diajukan sebagai syarat untuk menyelesaikan\\ Program Studi Strata-1 Departemen Informatika

	\vspace{1.5cm}
          {\bfseries Disusun Oleh: \\
           Jeffry Saputra \\
	1115040}
	
	\vspace{1.5cm}
	\includegraphics[width=8cm]{images/ithb.jpg}
	
	
	\vspace{3.5cm}
	
{\large \bfseries PROGRAM STUDI INFORMATIKA \\
INSTITUT TEKNOLOGI HARAPAN BANGSA \\
BANDUNG\\
2019}

	
\end{center}

\end{titlepage}

\newpage
	
	% Daftar isi, gambar, dan tabel
	% Gunakan penomeran Romawi (i, ii, iii, ...) setelah bagian ini.

	\newcounter{savepage}
	\pagenumbering{roman}
	
	% Lembar Pengesahan
	\phantomsection \addcontentsline{toc}{chapter}{LEMBAR PENGESAHAN}
	%\renewcommand{\headrulewidth}{3pt} 
\lhead{\includegraphics[width=0.3\textwidth]{images/ithb.jpg}\\[0.01cm]}
\rhead{{\bfseries DEPARTEMEN TEKNIK INFORMATIKA \\
 INSTITUT TEKNOLOGI HARAPAN BANGSA\\[0.01cm]}}
\thispagestyle{fancy}

\hspace{-2cm}\\[1cm]
\begin{center}
{\bfseries LEMBAR PENGESAHAN}\\[1.0 cm]
{\bfseries PENERAPAN METODE CONVOLUTIONAL NEURAL NETWORK MENGGUNAKAN KALMAN FILTER UNTUK MENDETEKSI DAN MELACAK MANUSIA PADA VIDEO RGB-D} \\[0.5 cm]
\end{center}

\vspace{0.5cm}
%\begin{wrapfigure}{r}{0.90\textwidth}
%\includegraphics[width= 3.5 cm, height= 5 cm]{images/icon.jpg}
%\vspace{-5cm}
%\vspace{1cm}
%\end{wrapfigure} 

%\hspace{1.5cm}
%\begin{table}[ht]
%\centering
%\hspace{-1.3cm} Disusun Oleh:\\
%	\begin{tabular}{lll}
%		\hspace{2 cm} Nama & : & XXX XXX XXX\\
%		\hspace{2 cm} NIM & : & XXXXXXX \\
%	\end{tabular}
%\end{table} 
%\\[1.5cm]

\begin{center}   
\begin{tabular}{ p{4.5cm}  p{3.5cm}}
 \includegraphics[width=4cm, height =6cm]{images/icon.jpg} &
\vspace{-4cm}{Disusun oleh:\newline Nama: Jovin Angelico \newline NIM	: 1115029}

\end{tabular}
\end{center}
\doublespacing
{\center
\vspace{1cm}
Telah Disetujui dan Disahkan\\ Sebagai laporan Tugas Akhir Departemen Teknik Informatika\\
Institut Teknologi Harapan Bangsa\\[0.5cm]
Bandung,   Agustus 2016\\
Disetujui,\\[0.5cm]
Pembimbing\\[2cm]
\bfseries 
{\underline {Ken Ratri Retno Wardani, S.Kom., M.T.}\\
NIK. 105033\\}}

	% Lembar Pernyataan Pribadi
	\phantomsection \addcontentsline{toc}{chapter}{LEMBAR PERNYATAAN HASIL KARYA PRIBADI}
	%\renewcommand{\headrulewidth}{3pt} 
\lhead{\includegraphics[width=0.3\textwidth]{images/ithb.jpg}\\[0.01cm]}
\rhead{{\bfseries DEPARTEMEN TEKNIK INFORMATIKA \\
 INSTITUT TEKNOLOGI HARAPAN BANGSA\\[0.01cm]}}
\thispagestyle{fancy}

\hspace{-2cm}\\[1cm]
\begin{center}
{\bfseries PERNYATAAN HASIL KARYA PRIBADI}\\[1.0 cm]
\end{center}
Saya yang bertanda tangan di bawah ini:\\[0.5 cm]
\renewcommand{\arraystretch}{1.5}
\begin{table}[ht]
	\begin{tabular}{lll}
		Nama & : & Jovin Angelico \\
		NIM & : &  1115029\\
	\end{tabular}
\end{table} 
\\Dengan ini menyatakan bahwa laporan Tugas Akhir dengan Judul : ” {\bfseries PENERAPAN METODE CONVOLUTIONAL NEURAL NETWORK MENGGUNAKAN KALMAN FILTER UNTUK MENDETEKSI DAN MELACAK MANUSIA PADA VIDEO RGB-D}” adalah hasil pekerjaan saya dan seluruh ide, pendapat atau materi dari sumber lain telah dikutip dengan cara penulisan referensi yang sesuai.\\[0.5 cm]
Pernyataan ini saya buat dengan sebenar-benarnya dan jika pernyataan ini tidak sesuai dengan kenyataan maka saya bersedia menanggung sanksi yang akan dikenakan pada saya.

\noindent
\vspace{0.3cm}
\begin{tabularx}{\linewidth}{XX}

\begin{minipage}{\linewidth}\raggedleft
\vspace{2cm}
Bandung, Agustus 2016\\
Yang membuat pernyataan,\\
\vspace{2cm}
Jovin Angelico\\
\end{minipage}
\end{tabularx}

	% Lembar Abstrak
	\phantomsection \addcontentsline{toc}{chapter}{ABSTRAK}
	%\include{abstrak}

	% Lembar Abstract
	\phantomsection \addcontentsline{toc}{chapter}{ABSTRACT}
	%\include{abstract}

	% Lembar Pedoman
	\phantomsection \addcontentsline{toc}{chapter}{PEDOMAN PENGGUNAAN TUGAS AKHIR}
	%
%
% Halaman Pedoman Pengunaan Tugas Akhir

\chapter*{PEDOMAN PENGGUNAAN TUGAS AKHIR}
{\raggedleft Laporan tugas akhir yang tidak dipublikasikan terdaftar dan tersedia di Perpustakaan Institut Teknologi Harapan Bangsa, dan terbuka untuk umum dengan ketentuan bahwa hak cipta ada pada pengarang dan pembimbing Tugas Akhir. Referensi kepustakaan diperkenankan dicatat, tetapi pengutipan atau peringkasan hanya dapat dilakukan dengan seizin pengarang atau pembimbing Tugas Akhir dan harus disertai dengan ketentuan penulisan ilmiah untuk menyebutkan sumbernya.}\\[1.0 cm]
Tidak diperkenankan untuk memperbanyak atau menerbitkan sebagian atau seluruh laporan tugas akhir tanpa seizin dari pengarang atau pembimbing Tugas Akhir yang bersangkutan.


\newpage


	% Kata Pengantar
	\phantomsection \addcontentsline{toc}{chapter}{KATA PENGANTAR}
	%% Kata Pengantar
\chapter*{KATA PENGANTAR}
{\raggedleft Terima kasih kepada Tuhan yang Maha Esa karena dengan bimbingan-Nya dan karunia-Nya penulis dapat melaksanakan Tugas Akhir yang berjudul \textquotedblright PENERAPAN METODE CONVOLUTIONAL NEURAL NETWORK MENGGUNAKAN KALMAN FILTER UNTUK MENDETEKSI DAN MELACAK MANUSIA PADA VIDEO RGB-D \textquotedblleft. Laporan ini disusun sebagai salah satu syarat kelulusan di Institut Teknologi Harapan Bangsa. Pada kesempatan ini penulis menyampaikan terima kasih yang sebesar-besarnya kepada:} \\
\begin{enumerate}
\item Tuhan Yang Maha Esa, karena oleh bimbingan-Nya penulis selalu mendapat pengharapan untuk menyelesaikan tugas akhir ini.
\item Ibu Elisafina Siswanto, S.T., M.T., selaku pembimbing I Tugas Akhir yang  senantiasa memberi dukungan, semangat, ilmu-ilmu, saran dan dukungan kepada penulis selama tugas akhir berlangsung dan selama pembuatan laporan tugas akhir ini.
\item Bapak Victor Libtuselah Brilliam Manu, S.T.,  selaku pembimbing II Tugas Akhir yang senantiasa memberi dukungan, semangat, ilmu-ilmu, saran dan dukungan kepada penulis selama tugas akhir berlangsung dan selama pembuatan laporan tugas akhir ini.
\item Ibu Ken Ratri Retno W, S.Kom., M.T, selaku penguji I Tugas Akhir. Terima kasih atas dukungan, semangat, ilmu-ilmu, dan masukan yang telah diberikan kepada penulis dalam menyelesaikan Laporan Tugas Akhir ini
\item Ibu Ir. Inge Martina, M.T., selaku penguji II dalam Tugas Akhir Terima kasih atas dukungan, semangat, ilmu-ilmu, dan masukan yang telah diberikan kepada penulis dalam menyelesaikan Laporan Tugas Akhir ini.
\item Seluruh dosen dan staff Departemen Teknik Informatika ITHB yang telah membantu dalam menyelesaikan Laporan Tugas Akhir ini.
\item Segenap jajaran staf dan karyawan ITHB yang turut membantu kelancaran dalam menyelesaikan Laporan Tugas Akhir ini.
\item Kedua orang tua tercinta yang selalu menyediakan waktu untuk memberikan doa, semangat dan dukungan yang tak habis-habisnya kepada penulis untuk menyelesaikan Laporan Tugas Akhir ini. Terima kasih untuk nasihat, masukan, perhatian, teguran dan kasih sayang yang diberikan hingga saat ini.
\\
\end{enumerate}
Penulis menyadari bahwa laporan ini masih jauh dari sempurna karena keterbatasan waktu dan pengetahuan yang dimiliki oleh penulis. Oleh karena itu, kritik dan saran untuk membangun kesempurnaan tugas akhir ini sangat diharapkan. Semoga tugas akhir ini dapat membantu pihak-pihak yang membutuhkannya.\\[1.5cm]  
\hfill
{\begin{flushright} Bandung, Agustus 2016\\[1.5cm] Hormat  Saya,\\ Penulis.\end{flushright}}
\newpage
	
	%\vspace*{-2.5cm}
	%\tableofcontents
	%\phantomsection
	%\addcontentsline{toc}{chapter}{DAFTAR ISI}
	%\clearpage
	%\vspace*{-2.5cm}
	%\listoftables
	%\phantomsection
	%\addcontentsline{toc}{chapter}{DAFTAR TABEL}
	%\clearpage
	%\vspace*{-2.5cm}
	%\listoffigures
	%\phantomsection
	%\addcontentsline{toc}{chapter}{DAFTAR GAMBAR}
	%\clearpage
	
	\setcounter{savepage}{\arabic{page}}
	\makeatletter
	\def\MyPagenumbering#1{%
		\global\c@page \@ne \gdef\thepage{\arabic{chapter}-\csname @#1\endcsname
			\c@page}}
	\makeatother
	\pagestyle{fancy}
	\renewcommand{\chaptermark}[1]{%
		\markboth{\thechapter.\ #1}{}}
	
	\fancyhf{}
	% Gunakan penomeran Arab (1, 2, 3, ...) setelah bagian ini.
	\MyPagenumbering{arabic}
	
	% Untuk mengatur posisi pagenumber
	%\pagestyle{plain}
	\setlength\LTleft{0pt}            % default: \fill
	\setlength\LTright{0pt}           % default: \fill
	\lhead{\leftmark}
	\renewcommand{\headrulewidth}{1pt}
	
	\fancypagestyle{plain}{%
		\renewcommand{\headrulewidth}{0pt}%
		\fancyhf{}%
		\fancyfoot[R]{\arabic{chapter}-1}%
	}
	
	\onehalfspacing
	\setcounter{page}{1}
	\rfoot{\arabic{chapter}-\arabic{page}}
	%-----------------------------------------------------------------------------%
\chapter{PENDAHULUAN}
%-----------------------------------------------------------------------------%

\vspace{4.5pt}

\section{Latar Belakang} \label{sec:latar_belakang}
\noindent \textit{Computer Vision} adalah sebuah cabang ilmu komputer yang mempelajari bagaimana komputer dapat memiliki kemampuan untuk dapat menginterpretasikan suatu kondisi melalui sebuah citra dan dapat bekerja selayaknya seperti penglihatan manusia. Terdapat beberapa tahapan dalam \textit{computer vision} yang digunakan untuk persepsi visual, seperti akuisisi citra, pengolahan citra, formasi citra, ekstraksi dan pencocokan fitur, segmentasi, deteksi dan pengenalan objek, dan lain sebagainya. Deteksi objek adalah metode untuk mendeteksi suatu objek dan digunakan untuk mencari objek-objek dari suatu citra. Dalam aplikasi sistem kecerdasan untuk transportasi, objek dapat berupa mobil, bagian dari mobil (logo, plat nomor kendaraan), ataupun rambu-rambu lalu lintas.

\noindent Sistem kecerdasan untuk transportasi merupakan bidang yang saat ini sedang berkembang dengan pesat dalam ranah \textit{computer vision} \cite{tabrizi}. Penerapannya pun semakin nyata dalam kehidupan manusia sehari-hari. Sistem navigasi satelit, sistem pengenalan rambu lalu lintas, sistem parkir otomatis, pengenalan plat kendaraan, dan keamanan kendaraan merupakan contoh dari aplikasi sistem kecerdasan untuk transportasi. Pengenalan plat nomor kendaraan memegang beberapa peranan penting dalam bidang transportasi, diantaranya untuk sistem pembayaran elektronik, dan penegakan hukum \cite{gou2014}.

\noindent Walaupun sistem pengenalan plat nomor kendaraan sudah memiliki sejarah penelitian yang panjang, hal ini tetap saja memiliki tantangan. Hal ini disebabkan banyak faktor yang mempengaruhi hasil akhir dari pengenalan plat nomor, contohnya adalah kondisi pencahayaan yang tidak merata, kondisi tulisan karakter pada plat nomor yang kurang jelas, dan lain sebagainya \cite{gou2014}.

\noindent Secara umum, sistem pengenalan plat nomor kendaraan dibagi kedalam tiga bagian utama: deteksi area plat nomor kendaraan, segmentasi karakter, dan pengenalan karakter. Gou et al. menerapkan ketiga hal tersebut dengan menggunakan metode \textit{Extremal Region} untuk mendeteksi lokasi plat nomor kendaraan sekaligus mendapatkan area dari karakter plat nomor kendaraan tersebut kemudian melakukan pengenalan karakter menggunakan \textit{Restricted Boltzmann Machines} \cite{gou2016}.

\noindent Penelitian lain menggunakan \textit{Maximally Stable Extremal Region} untuk mendeteksi area karakter dari plat nomor kendaraan kemudian dilanjutkan dengan metode \textit{Histogram of Oriented Gradient} untuk mendapatkan fitur dari masing-masing karakter dan menggunakan metode \textit{Extreme Learning Machine} untuk melakukan proses pengenalan karakter \cite{gou2014}.

\noindent Penelitian lain menggunakan metode morfologi citra untuk deteksi plat kendaraan dan menggunakan \textit{K-Nearest Neighbors} untuk melakukan klasifikasi terhadap karakter dan \textit{Support Vector Machine} untuk melakukan klasifikasi terhadap karakter yang memiliki kemiripan (pasangan B dengan 8, 5 dengan S, 4 dengan A, dsb) \cite{tabrizi}.

\noindent Penelitian lain menggunakan metode \textit{Hough Transform} untuk mendeteksi lokasi plat kendaraan kemudian dilanjutkan dengan metode \textit{Template Matching} untuk mengenali karakter dari plat nomor tersebut \cite{rasheed}. 

\noindent Penelitian ini menggunakan metode  \textit{Hough Transform} untuk mendeteksi plat nomor kendaraan, kemudian karakter-karakter pada plat kendaraan akan disegmentasi dengan menghitung grafik horizontal pita, kemudian dilanjutkan dengan menggunakan metode \textit{Histogram of Oriented Gradient} untuk mengambil fitur dari karakter-karakter dari citra hasil segmentasi dan terakhir akan diklasifikasikan dengan menggunakan metode \textit{Support Vector Machine}.

%\noindent Penelitian ini metode \textit{Convolutional Neural Network} dengan Kalman \textit{filter} untuk melakukan deteksi serta melacak manusia. Metode CNN telah mencakup proses utama dari ekstraksi fitur hingga klasifikasi dan telah terbukti kelebihannya untuk mengklasifikasi banyak fitur kompleks secara bersamaan. Metode \textit{neural network} dapat menyesuaikan dengan berbagai macam kasus namun membutuhkan waktu untuk pelatihan. Bila dibandingkan dengan \textit{Artificial Neural Network} (ANN) atau \textit{Multilayer Perceptron} (MLP) konvensional, pada bidang \textit{computer vision}, CNN memiliki performa yang lebih cepat hingga 2 kali lipat, karena CNN memiliki perbedaan metode yang tidak terdapat pada ANN/MLP untuk memproses banyak data besar, seperti teknik \textit{weight sharing} dan \textit{subsampling}/\textit{pooling}. Sementara metode Kalman \textit{filter} dipilih karena dapat meningkatkan efisiensi komputasi untuk melacak manusia.

\noindent Pada tahapan pengujian akan dilakukan dua macam pengujian akurasi, yaitu akurasi deteksi plat kendaraan dan akurasi pengenalan karakter plat nomor kendaraan. Perhitungan akurasi ini masing-masing akan menggunakan \textit{confusion matrix}.\\

\section{Rumusan Masalah}
\noindent Berdasarkan latar belakang di atas rumusan masalah yang didapatkan adalah sebagai berikut:
\begin{enumerate}[nolistsep,leftmargin=0.5cm]
\item Berapa akurasi pendeteksian plat nomor kendaraan dengan menggunakan metode \textit{Hough Transform} ?
\item Berapa akurasi pengenalan plat nomor kendaraan jika menggunakan metode \textit{Histogram of Oriented Gradient} dan \textit{Support Vector Machine} ? \\
\end{enumerate}

\section{Tujuan Penelitian}
\noindent Berdasarkan rumusan masalah di atas, tujuan penelitian Tugas Akhir ini adalah sebagai berikut:
\begin{enumerate}[nolistsep,leftmargin=0.5cm]
\item Menguji akurasi pendeteksian plat nomor kendaraan dengan metode \textit{Hough Transform} dengan beragam rentang nilai \textit{theta}.
\item Menerapkan metode \textit{Histogram of Oriented Gradient} untuk ekstraksi fitur pada karakter.
\item Menguji akurasi pengenalan karakter pada plat nomor kendaraan dengan metode \textit{Support Vector Machine} dengan beragam nilai sigma.\\
%\item Menguji akurasi penggunaan \textit{Convolutional Neural Network} untuk mendeteksi manusia.
%\item Menguji akurasi proses dari penggunaan metode Kalman \textit{filter} untuk melacak manusia.
%\item Membuat sebuah aplikasi yang dapat mendeteksi dan melacak manusia melalui kamera RGB-D.\\
\end{enumerate}

\section{Batasan Masalah}
\noindent Dalam penelitian ini, peneliti akan membatasi masalah yang akan diteliti antara lain:
\begin{enumerate}[nolistsep,leftmargin=0.5cm]
\item Plat kendaraan yang akan dideteksi adalah plat nomor kendaraan Indonesia.
\item Plat kendaraan Indonesia yang akan dideteksi adalah plat nomor kendaraan pribadi (plat hitam dengan tulisan putih), plat nomor kendaraan umum (plat kuning dengan tulisan hitam), dan plat nomor kendaraan dinas negara (plat merah dengan tulisan putih).\\ 
%\item Mendeteksi dan melacak manusia melalui tubuh manusia bagian atas, yaitu sekitar daerah bahu hingga ke ujung atas kepala.
%\item Masukan berupa sekumpulan citra RGB-D statis yang ditangkap secara berurutan (video).
%\item Dalam arsitektur metode \textit{Convolutional Neural Network} hanya digunakan 1 buah \textit{kernel} dalam setiap lapisan.\\
\end{enumerate}

\section{Kontribusi Penelitian}
\noindent Kontribusi yang diberikan dari penelitian ini adalah:

\begin{enumerate}[nolistsep,leftmargin=0.5cm]
\item Membuat penerapan metode \textit{Hough Transform} untuk pendeteksian plat nomor kendaraan.
\item Membuat penerapan metode \textit{Histogram of Oriented Gradient} untuk proses ekstraksi fitur pada proses pengenalan karakter pada plat nomor kendaraan.\\
%\item Melakukan analisis pengaruh \textit{Convolutional Neural Network} terhadap akurasi dari aplikasi deteksi dan melacak manusia.
%\item Melakukan analisis pengaruh Kalman \textit{filter} terhadap akurasi dan efisiensi komputasi dari aplikasi deteksi dan melacak manusia. \\
\end{enumerate}

\section{Metodologi Penelitian}
\noindent Metode penelitian yang dilakukan dalam penelitian ini adalah sebagai berikut:
\begin{enumerate}[nolistsep,leftmargin=0.5cm]
\item Studi Literatur \\
Penulisan ini dimulai dengan studi kepustakaan yaitu mengumpulkan bahan-bahan referensi baik dari buku, artikel, \textit{paper}, jurnal, makalah mengenai sistem pengenalan plat nomor kendaraan.
\item Data sampling \\
Data sampling yang akan digunakan berupa citra kendaraan yang berasal dari Universitas Telkom yang bernama \textit{Tel-U Vehicle License Plate Data-set V1.0}. Dataset ini merupakan dataset yang disusun oleh akademisi Universitas Telkom untuk keperluan penelitian mengenai plat nomor kendaraan.
\item Analisis Masalah \\
Pada tahap ini dilakukan analisis permasalahan yang ada, batasan yang dimiliki dan kebutuhan yang diperlukan.
\item Perancangan dan Implementasi Algoritme \\
Pada tahap ini dilakukan pendefinisian beberapa aturan dalam teknik \textit{preprocessing} citra, serta perancangan pada algoritme yang akan dipakai untuk menyelesaikan masalah berdasarkan metode yang telah dipilih.
\item Pengujian \\
Pada tahap ini dilakukan pengujian terhadap aplikasi yang telah dibangun.
\item Dokumentasi \\
Pada tahap ini dilakukan pendokumentasian hasil analisis dan implementasi secara tertulis dalam bentuk laporan skripsi. \\
\end{enumerate}

\section{Sistematika Pembahasan}
\noindent Pada penelitian ini peneliti menyusun berdasarkan sistematika penulisan sebagai berikut: \\[0.5cm]
\noindent \textbf{BAB I \hspace{1cm} Pendahuluan}
\begin{addmargin}[2.35cm]{0em}
Pendahuluan yang berisi latar belakang, rumusan masalah, tujuan penelitian, batasan masalah, kontribusi penelitian, serta metode penelitian.
\end{addmargin}
\noindent \textbf{BAB II \hspace{0.8cm} Landasan Teori}
\begin{addmargin}[2.35cm]{0em}
Landasan Teori yang berisi penjelasan dasar teori yang mendukung penelitian ini.
\end{addmargin}
\noindent \textbf{BAB III \hspace{0.7cm} Analisis dan Perancangan}
\begin{addmargin}[2.35cm]{0em}
Analisis dan Perancangan yang berisi analisis berupa algoritme yang digunakan.
\end{addmargin}
\noindent \textbf{BAB IV \hspace{0.7cm} Implementasi dan Pengujian}
\begin{addmargin}[2.35cm]{0em}
Implementasi dan Pengujian yang berisi implementasi pengujian dengan berbagai data testing beserta hasilnya.
\end{addmargin}
\noindent \textbf{BAB V \hspace{0.8cm} Kesimpulan dan Saran}
\begin{addmargin}[2.35cm]{0em}
Penutup yang berisi kesimpulan dari penelitian dan saran untuk penelitian lebih lanjut di masa mendatang.
\end{addmargin}

\newpage
	\setcounter{page}{1}
	%-----------------------------------------------------------------------------%
\chapter{LANDASAN TEORI}
%-----------------------------------------------------------------------------%
Bab ini menjelaskan teori-teori yang berkaitan mengenai teori penunjang dan jurnal terkait yang digunakan dalam proses penelitian tugas akhir ini. \\

%
\vspace{4.5pt}

\section{Tinjauan Pustaka}
\noindent Penelitian ini menggunakan beberapa teori terkait yang diperlukan dalam pengerjaan yang dilakukan. Penjelasan mengenai teori-teori tersebut akan dijelaskan sebagai berikut. \\

\subsection{Citra Digital}
\label{subsec:CitraDigital}
\noindent Citra digital merupakan sebuah fungsi dua dimensi \textit{f(x,y)}, di mana \textit{x} dan \textit{y} adalah koordinat, dan nilai \textit{f} menyatakan intensitas atau tingkat keabuan yang dimiliki citra pada titik atau \textit{pixel} (\textit{picture element}) tersebut. Nilai \textit{f} merupakan nilai berhingga dan bersifat diskrit \cite{gonzalez}.\\
\noindent Jenis citra digital bergantung pada jenis perangkat keras yang digunakan dan dapat dikelompokkan ke dalam beberapa jenis model warna, yang paling umum digunakan adalah model RGB. Citra model RGB merupakan citra yang menggunakan 3 kombinasi warna, yaitu merah, hijau, dan biru. Pada citra RGB 24-bit, setiap warna mempunyai nilai \textit{f} antara 0 hingga 255 sehingga perpaduan dari ketiga warna tersebut akan menghasilkan $256^{3}$ jenis warna \cite{gonzalez}. \\

\subsection{Pengolahan Citra}
\noindent Pengolahan citra dapat didefinisikan sebagai suatu bidang yang menggunakan citra sebagai masukan, lalu masukan tersebut diolah sehingga menghasilkan citra kembali. Berdasarkan definisi tersebut, perhitungan rata-rata dari suatu citra yang menghasilkan sebuah angka tidak termasuk pengolahan citra \cite{gonzalez}.\\
\noindent Terdapat satu paradigma yang mengategorikan 3 jenis proses komputasi dalam pengolahan citra, yaitu tingkat rendah, tingkat sedang, dan tingkat tinggi. Proses tingkat rendah mencakup operasi yang sangat sederhana seperti \textit{image preprocessing} untuk mengurangi \textit{noise}, meningkatkan kontras, dan mempertajam citra. Proses tingkat sedang meliputi segmentasi untuk membagi daerah citra menjadi \textit{region} atau objek. Sedangkan proses tingkat tinggi memampukan komputer untuk mengerti seperti pengenalan objek dan analisis citra \cite{gonzalez}. \\

\subsection{Pengabuan Citra}
\noindent Citra RGB yaitu citra berwarna memiliki ukuran yang lebih besar dibandingkan dengan citra \textit{grayscale}. Untuk mempercepat proses komputasi pada citra, maka citra RGB perlu diubah menjadi citra \textit{grayscale} dengan skala keabuan 256. Persamaan pengabuan citra dapat dilihat pada persamaan \ref{eq:grayscale}.

\begin{table}[H]
	\begin{adjustbox}{width=1\textwidth}
		\begin{tabular}{|p{13.55cm}|}
			\hline
			\begin{equation}
			Gray value = 0.299 R + 0.587 G + 0.114 B
			\label{eq:grayscale}
			\end{equation}\\
			\hline
		\end{tabular}
	\end{adjustbox}
\end{table}

\noindent Persamaan \ref{eq:grayscale} menyimpulkan bahwa persentase warna hijau yang paling besar karena manusia cenderung lebih sensitif terhadap perubahan warna hijau yang memiliki panjang gelombang sekitar 500-570 nm, merah, lalu biru \cite{ragb}. Persamaan \ref{eq:grayscale} merupakan rekomendasi dari \textit{International Telecommunication Union Radiocommunication Sector}.\\

\subsection{Deteksi Tepi}
\noindent Tepian memiliki arti yaitu terjadinya perubahan intensitas secara signifikan pada sebuah citra. Deteksi tepi yang digunakan pada penelitian ini adalah dengan menggunakan operator \textit{Canny} untuk mendapatkan tepian citra sebesar 1 piksel. Proses deteksi tepi Canny memiliki tahapan sebagai berikut:
\begin{enumerate}
	\item Penghalusan \\
	Pada tahapan pertama citra dihaluskan dengan \textit{Gaussian filter} untuk mengurangi derau yang dapat menghasilkan detail yang mengganggu.
	\item Menghitung Gradien \\
	Untuk menghitung gradien yang dapat menghasilkan tepian yang masih tebal dengan beberapa operator, diantaranya adalah Sobel, Prewitt, atau Robert.
	\item \textit{Non-maxima Supression}
	Tahapan ini digunakan untuk menipiskan tepian tebal yang diperoleh dari operasi sebelumnya dengan cara mencari nilai maksimum di tepian.
	\item \textit{Double Thresholding} \\
	Dari hasil \textit{non-maxima supression} bisa ditemukan tepian yang belum sempurna, sehingga perlu dilakukan \textit{thresholding} untuk menghilangkan derau yang tidak diinginkan. Caranya adalah dengan menetapkan 2 nilai \textit{threshold} yaitu \textit{high threshold} dan \textit{low threshold} untuk menentukan apakah piksel tersebut akan masuk dalam \textit{threshold} untuk dijadikan tepian. Piksel yang nilainya berada di atas \textit{high threshold} akan menjadi tepian kuat, sebaliknya jika di bawah \textit{low threshold} akan dijadikan sebagai \textit{background}.
	\item \textit{Edge Tracking}
	Tahapan terakhir yaitu \textit{Edge Tracking} atau \textit{Edge Linking} digunakan untuk menghubungkan tepian kuat dan tepian lemah yang nilai pikselnya berada diantara \textit{high threshold} dan \textit{low threshold}. Ketika tepian lemah yang tidak terhubung dengan tepian kuat maka piksel tersebut akan dianggap sebagai \textit{background}. Hasil akhir dari deteksi tepi Canny adalah tepian halus yang memiliki lebar sebesar 1 piksel.\\
\end{enumerate} 

\subsection{\textit{Hough Transform}}
\noindent \textit{Hough Transform} adalah sebuah teknik untuk mengidentifikasi bentuk spesifik dalam sebuah citra. \textit{Hough Transform} mengkonversikan semua titik dalam sebuah kurva ke dalam sebuah lokasi tunggal dalam ruang parametrik (ruang akumulator) lain dengan transformasi koordinat. Metode ini bertujuan untuk memetakan fitur global ke fitur lokal. Konsep ini juga dapat diterapkan untuk mendeteksi garis lurus, lingkaran, elips atau bentuk geometrik lainnya. \textit{Hough Transform} yang digunakan adalah untuk ekstraksi garis lurus. Berikut adalah rumus yang digunakan untuk mencari jarak antara titik \textit{origin} dengan garis yang terbentuk \cite{shih}:

\begin{table}[H]
	\begin{adjustbox}{width=1\textwidth}
		\begin{tabular}{|p{13.55cm}|}
			\hline
			\begin{equation}
			\rho = x\cos(\theta)+y\sin(\theta)
			\label{eq:PersamaanRho}
			\end{equation}\\
			\hline
		\end{tabular}
	\end{adjustbox}
\end{table}

\noindent Dimana:\\
$\rho$ = jarak antara titik \textit{origin} dengan garis\\
x = koordinat titik x\\
y = koordinat titik y\\
$\theta$ = sudut derajat; $0^\circ$ $\leq$ $\theta$ $\leq$ $180^\circ$

\noindent Metode \textit{Hough Transform} menerapkan skema \textit{voting}. Sebuah \textit{array} akumulator diperlukan untuk menyimpan hasil \textit{voting}. Rentang nilai $\theta$ (\textit{theta}) yang digunakan adalah antara nilai 0 hingga 180 derajat. Sedangkan rentang nilai $\rho$ (\textit{rho}) yang digunakan dalam akumulator adalah \cite{comvis}:

\begin{table}[H]
	\begin{adjustbox}{width=1\textwidth}
		\begin{tabular}{|p{13.55cm}|}
			\hline
			\begin{equation}
			-D \leq \rho \leq D
			\end{equation}\\
			\hline
		\end{tabular}
	\end{adjustbox}
\end{table}

\noindent Dimana:\\
D = jarak diagonal dari gambar

\noindent Karena citra yang digunakan berbentuk persegi panjang maka jarak diagonal memenuhi persamaan berikut:

\begin{table}[H]
	\begin{adjustbox}{width=1\textwidth}
		\begin{tabular}{|p{13.55cm}|}
			\hline
			\begin{equation}
			D = \sqrt{N^2 + M^2}
			\end{equation}\\
			\hline
		\end{tabular}
	\end{adjustbox}
\end{table}

\noindent Dimana:\\
D = jarak diagonal dari gambar\\
N = ukuran \textit{width} dari gambar\\
M = ukuran \textit{height} dari gambar

\noindent Sehingga rentang nilai \textit{rho} yang digunakan dalam penelitian ini adalah:

\begin{table}[H]
	\begin{adjustbox}{width=1\textwidth}
		\begin{tabular}{|p{13.55cm}|}
			\hline
			\begin{equation}
			-\sqrt{N^2 + M^2} \leq \rho \leq \sqrt{N^2 + M^2}
			\end{equation}\\
			\hline
		\end{tabular}
	\end{adjustbox}
\end{table}

\noindent Dimana:\\
N = ukuran \textit{width} dari gambar\\
M = ukuran \textit{height} dari gambar

\noindent Setelah proses perhitungan \textit{voting} dalam \textit{accumulator space} selesai, yang dilakukan selanjutnya adalah memilih \textit{peak} terbaik yang terdapat dalam \textit{accumulator space}. Nilai \textit{peak} yang tinggi memberikan indikasi yang baik dari garis \cite{oechsle}. Berdasarkan R. Varun, \textit{et al}. (2015), diketahui \textit{local maxima} dari \textit{accumulator space} dipertimbangkan sebagai \textit{peak} yang menonjol. Jumlah \textit{peak} merupakan hal yang krusial. Jumlah \textit{peak} yang terlalu sedikit atau terlalu banyak dapat mempengaruhi kinerja sistem.

\noindent Algoritma pencarian \textit{peak} menerapkan nilai \textit{threshold} dan pencarian lokal yang berdasarkan dari ukuran \textit{neighbourhood} (NS). Nilai \textit{threshold} untuk membatasi nilai \textit{voting} untuk mempertimbangkan \textit{peak} yang menonjol. Nilai NS bernilai 1 atau lebih. Algoritma pencarian \textit{peak} tersebut dapat menghindari hasil ganda untuk sebuah garis \cite{oechsle}.

\noindent Serangkaian \textit{peaks} yang diekstrak oleh metode \textit{Hough Transform} akan menghasilkan beragam nilai \textit{theta}. Intensitas kemunculan dari setiap nilai \textit{theta} akan dihitung dan dijadikan fitur yang mewakili sebuah citra plat nomor.

\noindent Dari penjelasan di atas maka \textit{output} dari proses ekstraksi fitur dengan metode \textit{Hough Transform} adalah intensitas kemunculan dari setiap nilai \textit{theta} yang didapat dari keseluruhan \textit{peak} yang terpilih. Karena rentang nilai \textit{theta} adalah 0 - 180 derajat maka ukuran fitur yang diekstrak adalah 181.\\ 

\subsection{Fitur pada Citra}
\noindent Dalam sistem pengenalan objek, fitur merupakan hal yang penting. Fitur merupakan atribut yang menonjol atau karakteristik yang dapat membedakan antara satu objek dengan objek lainnya. Fitur pada sebuah citra dapat digunakan untuk proses segmentasi dan klasifikasi. Sebuah objek dapat dibedakan berdasarkan fitur internal dan fitur eksternal. Fitur internal didapatkan berdasarkan komposisi piksel yang membentuk suatu wilayah (\textit{region}), sedangkan fitur eksternal membahas mengenai batas wilayah (\textit{region boundary}) dari sebuah objek. Contoh fitur internal adalah fitur tekstur, fitur dasar geometri, momen, histogram, dan \textit{Euler Number}. Fitur eksternal adalah \textit{Chain codes}, \textit{signatures}, dan \textit{Fourier descriptors} untuk menggambarkan bentuk dari objek (\textit{shape descriptor}) \cite{gonzalez}.\\

\subsection{\textit{Histogram of Oriented Gradient}}
\noindent\textit{Histogram of Oriented Gradients} merupakan salah satu teknik pengambilan fitur yang bertujuan untuk mengambil informasi penting dari sebuah citra. Cara kerja metode ini yaitu dengan mengevaluasi histogram lokal yang sudah ternormalisasi secara baik dari distribusi gradien citra dalam \textit{grid} yang padat. Teknik mengekstrak fitur untuk metode ini yaitu dari distribusi lokal dari intensitas gradient tiap piksel yang terdapat pada sebuah objek citra. Dalam metode \textit{Histogram of Oriented Gradient}, ukuran sel berupa kumpulan atau gabungan piksel dan blok berupa kumpulan atau gabungan sel beserta jumlah \textit{orientation bin} yang merupakan tempat unutk menampung hasil arah dan besar gradien akan mempengaruhi hasil keluaran fitur vektor yang dihasilkan dan juga akurasi yang didapat. Pertama untuk setiap piksel dari citra akan dihitung gradiennya dari sumbu x dan y dengan menggunakan persamaan :

\begin{table}[H]
	\small
	\begin{adjustbox}{width=1\textwidth}
		\begin{tabular}{|p{13.55cm}|}
			\hline
			\begin{equation} 
			G_{x}(x, y) = I(x+1, y) - I(x-1, y)
			\label{eq:PersamaanGradienX}
			\end{equation}\\
			\begin{equation} 
			G_{y}(x, y) = I(x, y+1) - I(x, y-1)
			\label{eq:PersamaanGradienY}
			\end{equation}\\
			\hline
		\end{tabular}
	\end{adjustbox}
\end{table}

\noindent Dimana:\\
$G_{x}(x,y)$ = nilai gradient untuk sumbu x\\
$G_{y}(x,y)$ = nilai gradient untuk sumbu y\\
$I(x,y)$ = nilai piksel citra dari baris x dan kolom y

\noindent Setelah didapat nilai gradient dari sumbu x dan y untuk setiap pikselnya, proses selanjutnya adalah menghitung besar nilai dan arah gradiennya dengan menggunakan rumus:

\begin{table}[H]
	\small
	\begin{adjustbox}{width=1\textwidth}
		\begin{tabular}{|p{13.55cm}|}
			\hline
			\begin{equation} 
			M(x, y) = \sqrt{G_{x}(x,y)^2 + G_{y}(x,y)^2}
			\label{eq:PersamaanMagnitude}
			\end{equation}\\
			\begin{equation} 
			\theta(x, y) = arctan\frac{G_{y}(x,y)}{G_{x}(x,y)}
			\label{eq:PersamaanArah}
			\end{equation}\\
			\hline
		\end{tabular}
	\end{adjustbox}
\end{table}
\noindent Dimana:\\
$M(x,y)$ = besar nilai gradient dari sumbu x dan y\\
$\theta(x,y)$ = arah nilai gradient dari sumbu x dan y

\noindent Kemudian, setiap piksel citra akan dibagi ke dalam beberapa sel yang dari setiap sel, akan dihitung persebaran \textit{Histogram of Oriented Gradient}-nya melalui proses \textit{voting}. Proses \textit{voting} dalam \textit{Histogram of Oriented Gradient} pertama akan menentukan nilai-nilai dari \textit{bin} dengan membagi total jumlah sudut gradien ke dalam jumlah \textit{orientation bin}. Kemudian untuk setiap arah sudut gradien dari setiap piksel dalam sel akan dimasukkan ke dalam rentang \textit{orientation bin} yang sudah ditentukan pada pertama kali, kemudian membagi besar nilai gradiennya dengan \textit{orientation bin} yang terkait.

\noindent Setelah \textit{Histogram of Oriented Gradient} sudah dibuat untuk setiap sel, proses selanjutnya adalah melakukan normalisasi terhadap hasil \textit{vote} pada setiap \textit{bin} dalam sel. Normalisasi akan dilakukan dalam 1 blok, dengan ukuran blok merupakan m $\times$ n sel. Terdapat 4 macam metode untuk normalisasi, yaitu: \textit{L2-Norm}, \textit{L2-Hys}, \textit{L1-sqrt}, dan \textit{L1-Norm}. Persamaannya adalah sebagai berikut:

\begin{table}[H]
	\small
	\begin{adjustbox}{width=1\textwidth}
		\begin{tabular}{|p{13.55cm}|}
			\hline
			\begin{equation} 
			V_{i} = \frac{V_{i}}{\sum_{i=1}^{N}V_{i}}
			\end{equation}\\
			\hline
		\end{tabular}
	\end{adjustbox}
\end{table}
\noindent Dimana:\\
$V_{i}$ = bobot vektor hasil \textit{L1-Norm} yang merepresentasikan nilai setiap \textit{bin}\\
$i$ = nilai \textit{counter} dari 1 sampai N\\
$N$ = jumlah total nilai total \textit{bin} yang digunakan dalam proses normalisasi

\begin{table}[H]
	\small
	\begin{adjustbox}{width=1\textwidth}
		\begin{tabular}{|p{13.55cm}|}
			\hline
			\begin{equation} 
			V_{i} = \sqrt{\frac{V_{i}}{\sum_{i=1}^{N}V_{i}}}
			\end{equation}\\
			\hline
		\end{tabular}
	\end{adjustbox}
\end{table}
\noindent Dimana:\\
$V_{i}$ = bobot vektor hasil \textit{L1-sqrt} yang merepresentasikan nilai setiap \textit{bin}\\
$i$ = nilai \textit{counter} dari 1 sampai N\\
$N$ = jumlah total nilai total \textit{bin} yang digunakan dalam proses normalisasi

\begin{table}[H]
	\small
	\begin{adjustbox}{width=1\textwidth}
		\begin{tabular}{|p{13.55cm}|}
			\hline
			\begin{equation} 
			V_{i} = \frac{V_{i}}{\sqrt{\sum_{i=1}^{N}V_{i}^2}}
			\label{eq:L2-Norm}
			\end{equation}\\
			\hline
		\end{tabular}
	\end{adjustbox}
\end{table}
\noindent Dimana:\\
$V_{i}$ = bobot vektor hasil \textit{L2-Norm} yang merepresentasikan nilai setiap \textit{bin}\\
$i$ = nilai \textit{counter} dari 1 sampai N\\
$N$ = jumlah total nilai total \textit{bin} yang digunakan dalam proses normalisasi

\noindent Untuk algoritma rumus normalisasi \textit{L2-Hys} merupakan algoritma mengikuti dari \textit{L2-Norm}, namun dengan membatasi nilai maksimal hasil normalisasi sebesar 0,2.

\noindent Adapun proses normalisasi blok akan dilakukan dalam \textit{sliding window} yang akan bergerak melakukan proses dengan pergeseran sebesar 1$\times$ ukuran sel secara vertikal dan horizontal. Proses ini kemudian akan bersifat \textit{overlapping} untuk beberapa sel yang dinormalisasi sehingga menimbulkan informasi yang redundan, namun akurasi yang dihasilkan justru semakin meningkat karenanya. Terakhir, hasil dari normalisasi tiap blok akan digabungkan menjadi 1 fitur vektor besar.\\

\subsection{\textit{Support Vector Machine}}
\noindent\textit{Support Vector Machine} merupakan salah satu algoritme \textit{supervised learning} untuk melakukan klasifikasi serta regresi dengan menggunakan teori vektor. SVM dapat memetakan vektor input ke dalam sebuah ruang beukuran n-dimensional (n adalah jumlah fitur). Konsep dasar dari SVM adalah menemukan sebuah \textit{separating hyperplane} (bidang) yang dapat memisahkan dua kelas dengan margin maksimal. 

\noindent Garis putus-putus yang berada paling dekat dengan masing-masing kelas merupakan \textit{hyperplane} paralel untuk memisahkan kedua kelas (lihat gambar \ref{fig:hyperplanesvm}). Asumsinya adalah semakin besar jarak atau margin antara 2 \textit{hyperplane} pendukung ini maka semakin baik hasil klasifikasinya. \textit{Hyperplane} yang optimal harus memenuhi persamaan \ref{eq:PersamaanHyperplane} berikut:

\begin{table}[H]
	\small
	\begin{adjustbox}{width=1\textwidth}
		\begin{tabular}{|p{13.55cm}|}
			\hline
			\begin{equation} 
			w^T \cdot x + b = 0
			\label{eq:PersamaanHyperplane}
			\end{equation}\\
			\hline
		\end{tabular}
	\end{adjustbox}
\end{table}

\noindent Dimana $w^T$ adalah vektor berat dan $x$ adalah vektor input dan $b$ merupakan nilai bias. Tanda “.” menggambarkan perkalian dot vektor.
\begin{table}[H]
	\small
	\begin{adjustbox}{width=1\textwidth}
		\begin{tabular}{| p {14cm} |}
			\hline
			\begin{figure}[H]
				\centering
				\includegraphics[width=14cm]{images/svm.jpeg}
			\end{figure} \\
			\hline
		\end{tabular}
	\end{adjustbox}
	\captionof{figure}{Contoh \textit{Hyperplane} pada SVM}
	\label{fig:hyperplanesvm}
\end{table}

\noindent SVM pada mulanya digunakan untuk menangani klasifikasi yang terdiri dari 2 kelas saja. Namun seriring dengan perkembangan zaman masalah yang dihadapi semakin kompleks sehingga membutuhkan teknik untuk melakukan proses klasifikasi lebih dari 2 kelas. Untuk melakukan klasifikasi lebih dari 2 kelas, terdapat 2 pendekatan yang bisa digunakan yaitu \textit{One-Versus-One} dan \textit{One-Versus-Rest}. Pada pendekatan \textit{One-Versus-One}, akan dibuat sebanyak $k(k-1)/2$  pasangan kelas untuk pengujian untuk klasifikasi dengan kelas sebanyak $k$. Untuk menentukan kelas mana yang menjadi klasifikasi untuk suatu kumpulan data caranya adalah sistem \textit{voting}. Kelas dengan jumlah \textit{voting} terbanyak akan menjadi \textit{classifier} untuk data tersebut. Pada pendekatan \textit{One-Versus-Rest} akan dibuat sebanyak $k$ pasangan kelas untuk klasifikasi dengan kelas sebanyak $k$. Setiap kelas yang diuji akan dibandingkan dengan sisa kelas yang ada. Misal terdapat 3 kelas A,B, dan C, maka kelas A akan dibandingkan dengan kelas B dan C, kelas B dibandingkan dengan kelas A dan C, kelas C dibandingkan dengan kelas A dan B. Kekurangan dari pendekatan ini adalah jumlah \textit{training set} yang tidak seimbang \cite{svm}.

\noindent Untuk proses klasifikasi \textit{non-linear} dapat dicari dengan persamaan 2.13 berikut:
\begin{table}[H]
	\small
	\begin{adjustbox}{width=1\textwidth}
		\begin{tabular}{|p{13.55cm}|}
			\hline
			\begin{equation}
			f(x) = sign(\sum_{i=1}^{l}\alpha_iy_iK(x,x_i)+b)
			\end{equation}\\
			\hline
		\end{tabular}
	\end{adjustbox}
\end{table}
\noindent
\renewcommand{\arraystretch}{1} 
\begin{tabularx}{\textwidth}{lll}
	Dimana: \\
	$l$ & = & Banyaknya kelas citra\\
	$\alpha_i$ & = & Nilai alpha ke $i$\\
	$y_i$ & = & Nilai kelas citra ke $i$\\
	$K(x,x_i)$ & = & Fungsi kernel\\
	$b$ & = & Nilai bias\\
\end{tabularx}

\noindent Nilai $\alpha$ dan $b$ dapat dicari dengan persamaan linear yang membentuk \textit{hyperplane} SVM. Persamaan \ref{eq:PersamaanSVM1} sampai \ref{eq:PersamaanSVM2} berikut merupakan persamaan \textit{hyperplane} SVM:

\begin{table}[H]
	\small
	\begin{adjustbox}{width=1\textwidth}
		\begin{tabular}{|p{13.55cm}|}
			\hline
			\begin{equation}
			\sum_{i=1}^{l}\alpha_iy_iK(x,x_i)+b)=0
			\label{eq:PersamaanSVM1}
			\end{equation}
			\begin{equation}
			\sum_{i=1}^{l}\alpha_iy_iK(x,x_i)+b)=1
			\end{equation}
			\begin{equation}
			\sum_{i=1}^{l}\alpha_iy_iK(x,x_i)+b)=-1
			\label{eq:PersamaanSVM2}
			\end{equation}\\
			\hline
		\end{tabular}
	\end{adjustbox}
\end{table}
\noindent
\renewcommand{\arraystretch}{1} 
\begin{tabularx}{\textwidth}{lll}
	Dimana: \\
	$l$ & = & banyaknya kelas citra\\
	$\alpha_i$ & = & nilai alpha ke $i$\\
	$y_i$ & = & nilai kelas citra ke $i$\\
	$K(x,x_i)$ & = & fungsi kernel\\
	$b$ & = & nilai bias\\
\end{tabularx}

\noindent Seringkali kasus yang ada dalam dunia nyata tidak selalu bisa dipisahkan secara linier (\textit{linearly separable}) seperti pada contoh gambar di atas. Misalnya suatu kumpulan data memiliki fitur yang memiliki n-dimensi. Linear SVM tidak bisa diterapkan untuk kasus tersebut, sehingga diperlukan teknik agar membuat \textit{hyperplane} yang bisa memisahkan antara 2 kelas dalam ruang multidimensi. Cara yang umum digunakan untuk menyelesaikan masalah tersebut adalah dengan menggunakan kernel. Kernel yang umum digunakan pada SVM yaitu \textit{Radial Basis Function} seperti pada persamaan \ref{eq:PersamaanRBF} berikut:

\begin{table}[H]
	\small
	\begin{adjustbox}{width=1\textwidth}
		\begin{tabular}{|p{13.55cm}|}
			\hline
			\begin{equation}
			RBF = K(x_i,x_j) = \exp (-\frac{\|x_i-x_j\|}{2\sigma^2})
			\label{eq:PersamaanRBF}
			\end{equation}\\
			\hline
		\end{tabular}
	\end{adjustbox}
\end{table}
\noindent
\renewcommand{\arraystretch}{1} 
\begin{tabularx}{\textwidth}{lll}
	Dimana: \\
	$K$ & = & nilai fungsi kernel RBF\\
	$x_i$ & = & vektor input 1\\
	$x_j$ & = & vektor input 2\\
	$\sigma$ & = & konstanta sigma\\
\end{tabularx}
\vspace{4.5pt}

\subsection{\textit{Confusion Matrix}}
\noindent \textit{Confusion Matrix} merupakan metode pengukuran untuk mengevaluasi hasil klasifikasi. Dengan melakukan klasifikasi sebanyak $C$ kelas, dihasilkan \textit{confusion matrix} $M$ berukuran $C \times C$, di mana elemen $M_{ij}$ dalam matriks menunjukkan jumlah sampel yang salah diklasifikasikan, sementara $M_{ii}$ adalah jumlah sampel yang hasil klasifikasinya adalah benar. \textit{Confusion matrix} pada gambar \ref{fig:ConfusionMatrix} digunakan pada kasus klasifikasi dua buah kelas sehingga membentuk matriks berukuran $2 \times 2$ \cite{16}.

\begin{adjustbox}{width=1\textwidth}
\noindent\begin{minipage}{\linewidth}
	\framebox[\textwidth]{\includegraphics[width=6cm]{images/ConfusionMatrixForBinaryClassification.jpg}}
	\captionof{figure}{\textit{Confusion Matrix untuk Dua Kelas} \cite{16}}
	\label{fig:ConfusionMatrix}
\end{minipage}
\end{adjustbox}

\noindent Elemen $M_{11}$ pada matriks menunjukkan jumlah sampel yang pada kenyataannya adalah kelas 1 dan diklasifikasikan sebagai kelas 1, sehingga disebut sampel \textit{true-positive} (TP). Elemen $M_{12}$ menunjukkan jumlah sampel yang pada kenyataannya adalah kelas 1 tetapi diklasifikasikan sebagai kelas -1, sehingga disebut sampel \textit{false-negative} (FN). Elemen $M_{21}$ menunjukkan jumlah sample yang pada kenyataannya adalah kelas -1 tetapi diklasifikasikan sebagai kelas 1, sehingga disebut sampel \textit{false-positive} (FP). Dan elemen $M_{22}$ menunjukkan jumlah sampel yang kenyataannya adalah kelas -1 dan diklasifikasikan sebagai kelas -1, sehingga disebut \textit{true-negative} (TN). Maka untuk menghitung akurasi dapat digunakan persamaan \ref{eq:accuracy}. Hasil akurasi yang semakin baik akan mendekati nilai 1, sebaliknya akurasi yang buruk mendekati nilai 0.
\begin{table}[H]
	\begin{adjustbox}{width=1\textwidth}
	\begin{tabular}{|p{13.55cm}|}
		\hline
		\begin{equation}
			Accuracy = \frac{TP + TN}{TP + TN + FP + FN}
			\label{eq:accuracy}
		\end{equation}\\
	\hline
	\end{tabular}
	\end{adjustbox}
\end{table}

Lalu perhitungan \textit{precision} yang merupakan perbandingan dari hasil positif dapat dihitung dengan persamaan \ref{eq:precision}.
\begin{table}[H]
	\begin{adjustbox}{width=1\textwidth}
	\begin{tabular}{|p{13.55cm}|}
		\hline
		\begin{equation}
			Precision = \frac{TP}{TP + FP}
			\label{eq:precision}
		\end{equation}\\
	\hline
	\end{tabular}
	\end{adjustbox}
\end{table}

Dan perhitungan \textit{recall} atau disebut juga sebagai sensitivitas dapat dihitung dengan persamaan \ref{eq:recall}.
\begin{table}[H]
	\begin{adjustbox}{width=1\textwidth}
	\begin{tabular}{|p{13.55cm}|}
		\hline
		\begin{equation}
			Recall = \frac{TP}{TP + FN}
			\label{eq:recall}
		\end{equation}\\
	\hline
	\end{tabular}
	\end{adjustbox}
\end{table}

%\section{Pustaka}
%\noindent Berikut adalah penjelasan dari \textit{library} yang digunakan di dalam penelitian. \\

%\subsection{BufferedImage}
%\noindent BufferedImage merupakan \textit{subclass} yang menyatakan citra dengan data \textit{buffer} citra yang dapat diakses. Pada \textit{subclass} tersebut terdapat informasi seperti jenis \textit{color model} dan raster yang digunakan pada citra. Pada tabel \ref{tbl:BufferedImage} dijelaskan mengenai \textit{method} dari \textit{subclass} BufferedImage yang digunakan.
%\begingroup
%\setlength{\LTleft}{-20cm plus -1fill}
%\setlength{\LTright}{\LTleft}
%\begin{small}
%	\begin{longtable}{ |p{3cm}|p{5.5cm}|p{4cm}| }
%		\caption{Tabel BufferedImage.java} \label{tbl:BufferedImage}\\
%		\hline
%		\textbf{Nama \textit{Method}} & \textbf{Definisi} & \textbf{Atribut}\\
%		\endfirsthead
%		\multicolumn{3}{c}{\textbf{\tablename~\thetable} Tabel BufferedImage.java (Lanjutan)}\\
%		\hline
%		\textbf{Nama \textit{Method}} & \textbf{Definisi} & \textbf{Atribut}\\
%		\endhead
%		\hline
%		public BufferedImage(int width, int height, int imageType) & Konstruktor yang membentuk sebuah objek BufferedImage di mana objek tersebut memiliki ukuran lebar dan panjang serta \textit{color model} yang telah ditentukan. & width = lebar citra \newline height = tinggi citra \newline imageType = jenis \textit{color model} yang digunakan citra \\
%		\hline
%		public BufferedImage getSubimage(int x, int y, int w, int h) &
%		Fungsi akan mengembalikan citra yang telah dipotong berdasarkan posisi persegi panjang yang ditentukan. & x = koordinat x awal,\newline y = koordinat y awal,\newline w = lebar persegi panjang,\newline h = tinggi persegi panjang.\\
%		\hline
%		public Graphics2D createGraphics() &
%		Fungsi yang akan mengembalikan objek Graphics2D setelah objek tersebut dibuat, objek akan digunakan untuk menggambar ke objek BufferedImage. & -\\
%		\hline
%		public int getRGB(int x, int y) &
%		Fungsi untuk mendapatkan nilai RGB dari suatu piksel pada citra. & x = koordinat piksel x,\newline y = koordinat piksel y.\\
%		\hline
%		public int getWidth() & Fungsi untuk mendapatkan informasi lebar dari citra. & -\\
%		\hline
%		public int getHeight() & Fungsi untuk mendapatkan informasi tinggi dari citra. & -\\
%		\hline
%	\end{longtable}
%\end{small}
%\endgroup

%\subsection{Graphics2D}
%\noindent Graphics2D merupakan \textit{class} yang menyediakan pengaturan yang lebih banyak seperti geometri, transformasi, warna, dan \textit{layout}. Pada tabel \ref{tbl:Graphics2D} dijelaskan mengenai \textit{method} dari \textit{class} Graphics2D yang digunakan.
%\begingroup
%\setlength{\LTleft}{-20cm plus -1fill}
%\setlength{\LTright}{\LTleft}
%\begin{small}
%	\begin{longtable}{ |p{3cm}|p{5.5cm}|p{4cm}| }
%		\caption{Tabel Graphics2D.java} \label{tbl:Graphics2D}\\
%		\hline
%		\textbf{Nama \textit{Method}} & \textbf{Definisi} & \textbf{Atribut}\\
%		\endfirsthead
%		\multicolumn{3}{c}{\textbf{\tablename~\thetable} Tabel Graphics2D.java (Lanjutan)}\\
%		\hline
%		\textbf{Nama \textit{Method}} & \textbf{Definisi} & \textbf{Atribut}\\
%		\endhead
%		\hline
%		public abstract boolean drawImage(Image img, int x, int y, int width, ImageObserver observer); & Fungsi untuk membuat sebuah citra dengan perubahan yang telah ditentukan. Bila hasil pengembalian adalah bernilai \textit{true} berarti citra berhasil dimuat dan diproses, semetara bila nilai \textit{false} berarti citra sedang dimuat. & img = citra yang diproses \newline x = koordinat x,\newline y = koordinat y,\newline observer = \textit{interface} yang akan dinotifikasi mengenai informasi citra yang telah dikonstruksi. \\
%		\hline
%		public abstract void dispose(); & Fungsi untuk membuang informasi yang berkenaan dengan citra dan membebaskan alokasi sumber daya yang digunakannya. Setelah dilakukan fungsi ini, objek Graphics yang bersangkutan tidak akan dapat digunakan kembali. & -\\
%		\hline
%		public abstract void setColor(Color c) & Fungsi untuk menentukan jenis warna yang akan digunakan. & c = warna dalam bentuk RGB dengan tipe data Color.\\
%		\hline
%		public void drawRect(int x, int y, int width, int height) & Fungsi untuk menggambar 4 garis pada persegi panjang dalam citra. & x = koordinat x awal,\newline y = koordinat y awal,\newline width = lebar persegi panjang,\newline height = tinggi persegi panjang.\\
%		\hline
%	\end{longtable}
%\end{small}
%\endgroup

%\subsection{\textit{Library} OpenCV}
%\noindent OpenCV (\textit{Open Source Computer Vision}) adalah salah satu \textit{library open source} yang fokus dalam \textit{real-time computer vision} dan mendukung \textit{deep learning}. Pada awal mulanya, dikembangkan oleh Intel lalu didukung oleh Willow Garage dan sekarang dipelihara oleh Itseez. OpenCV dapat digunakan secara gratis untuk kepentingan akademik atau penggunaan secara komersial. Jenis \textit{interface}-nya beragam mulai dari C, C++, Python, hingga Java dan mendukung sistem operasi Windows, Linux, Mac OS, iOS, ataupun Android. Dengan berbagai macam algoritme yang telah dioptimasi, OpenCV dapat digunakan dalam berbagai bidang serta telah digunakan lebih dari 47 ribu orang dan telah diunduh lebih dari 14 juta kali.\\
%\noindent Dalam penelitian ini, OpenCV digunakan untuk mendeteksi dan melacak manusia. Fungsi-fungsi yang digunakan dalam penelitian ini dijelaskan pada tabel \ref{tbl:OpenCV}.
%\begingroup
%\setlength{\LTleft}{-20cm plus -1fill}
%\setlength{\LTright}{\LTleft}
%\begin{small}
%\begin{longtable}{|p{0.4cm}|p{2cm}|p{1.8cm}|p{1.8cm}|p{1.7cm}|p{3.55cm}|}
%	\caption{Daftar \textit{Method} yang Digunakan} \label{tbl:OpenCV}\\
%	\hline
%	\multirow{2}{*}{\textbf{No}} & \multirow{2}{*}{\textit{\textbf{Method}}} & \multicolumn{2}{c|}{\textit{\textbf{Input}}} & \multirow{2}{*}{\textit{\textbf{Output}}} & 
%	\multirow{2}{*}{\textbf{Keterangan}}\\
%	\cline{3-4}
%	& & \textbf{Tipe} & \textbf{Variabel} & & \\
%	\endfirsthead
%	\multicolumn{6}{c}{\textbf{\tablename~\thetable} Daftar \textit{Method} yang Digunakan (Lanjutan)}\\
%	\hline
%	\multirow{2}{*}{\textbf{No}} & \multirow{2}{*}{\textit{\textbf{Method}}} & \multicolumn{2}{c|}{\textit{\textbf{Input}}} & \multirow{2}{*}{\textit{\textbf{Output}}} & 
%	\multirow{2}{*}{\textbf{Keterangan}}\\
%	\cline{3-4}
%	& & \textbf{Tipe} & \textbf{Variabel} & & \\
%	\endhead
%	\hline
%	1 & Cascade-\newline Classifier & String & filename & - & Konstruktor dari \textit{class} CascadeClassifier untuk memuat berkas xml.\\
%	\hline
%	2 & detect-\newline MultiScale & Mat,\newline MatOfRect,\newline double,\newline int,\newline int,\newline Size,\newline Size & image,\newline objects,\newline scaleFactor,\newline min-\newline Neighbors,\newline flags,\newline minSize,\newline maxSize & - & Fungsi dari \textit{class} Cascade Classifier untuk mendeteksi lokasi objek berdasarkan ketentuan faktor skala, banyak tetangga, dan ukuran minimum, maksimum yang digunakan. Hasil disimpan pada parameter objects yang memiliki tipe data MatOfRect.\\
%	\hline
%	3 & MatOfRect & - & - & - & Konstruktor dari class MatOfRect untuk melakukan inisialisasi.\\
%	\hline
%	4 & Imgcodecs.\newline imread & String & filename & Mat & Memuat citra berdasarkan alamat berkas.\\
%	\hline
%	5 & Imgcodecs.\newline imwrite & String,\newline Mat & filename,\newline img & boolean & Menyimpan berkas citra pada alamat yang ditentukan.\\
%	\hline
%	6 & Imgproc.\newline rectangle & Mat,\newline Point,\newline Point,\newline Scalar & img,\newline pt1,\newline pt2,\newline color & - & Menggambar persegi panjang pada citra berdasarkan posisi dan warna yang ditentukan.\\
%	\hline
%	7 & Point & double,\newline double & x,\newline y & - & Konstruktor dari \textit{class} Point.\\
%	\hline
%\end{longtable}
%\end{small}
%\endgroup

\section{Tinjauan Studi}
\noindent Pada bagian ini akan dijelaskan mengenai perbandingan dari berbagai penelitian terkait metode deteksi dan pengenalan plat nomor mobil. \\

\subsection{\textit{State of the Art}}
\noindent Terdapat beberapa metode lain yang memiliki ruang lingkup yang mirip dengan penelitian ini khususnya mengenai deteksi dan pengenalan plat nomor mobil. Tabel \ref{tbl:StateoftheArt} \textit{State of the Art} akan menjelaskan perbedaan-perbedaan metode yang telah dipelajari oleh penulis dari jurnal.\\
\\
\\
\\
\\

\begingroup
\setlength{\LTleft}{-20cm plus -1fill}
\setlength{\LTright}{\LTleft}
\begin{small}
\begin{longtable}{ |p{5cm}|p{3.5cm}|p{3.6cm}| }
\caption{\textit{State of the Art}}\\
\hline
\textbf{Jurnal} & \textbf{Rumusan Masalah} & \textbf{Metode}\\
\endfirsthead

\multicolumn{3}{c}{\textbf{\tablename~\thetable} \textit{State of the Art} (Lanjutan)}\\
\hline
\textbf{Jurnal} & \textbf{Rumusan Masalah} & \textbf{Metode}\\
\endhead

\hline
 Gou, C., Wang, K., Yao, Y., Li, Z. (2016). Vehicle license plate recognition based on extremal regions and restricted Boltzmann machines. \emph{IEEE Transactions on Intelligent Transportation Systems, 17}(4), 1096-1107. & Apakah dengan menerapkan pendeteksian plat nomor dengan \textit{Extremal Region} dan pengenalan karakter plat nomor menggunakan \textit{Hybrid Discriminative Restricted Boltzmann Machine} dapat meningkatkan akurasi pendeteksian dan pengenalan dalam berbagai kondisi cuaca dan \textit{background} yang kompleks? & 
\begin{enumerate}[wide, labelwidth=!, labelindent=0pt, topsep=0pt]
\item \textit{Extremal Region}
\item \textit{AdaBoost}
\item \textit{Histogram of Oriented Gradient}
\item \textit{Hybrid Discriminative Restricted Boltzmann Machine} 
\end{enumerate}\\
\hline
 Gou, C., Wang, K., Yu, Z., Xie, H. (2014, October). License plate recognition using MSER and HOG based on ELM. In \emph{Proceedings of 2014 IEEE International Conference on Service Operations and Logistics, and Informatics} (pp. 217-221). IEEE. & Apakah dengan menerapkan \textit{Maximally Stable Extremal Region} untuk pendeteksian plat nomor dan \textit{Extreme Learning Machine} untuk pengenalan karakter plat nomor dapat meningkatkan performa dan akurasi sistem? & 
\begin{enumerate}[wide, labelwidth=!, labelindent=0pt, topsep=0pt]
\item \textit{Maximally Stable Extremal Region}
\item \textit{Histogram of Oriented Gradient}
\item \textit{Extreme Learning Machine}
\end{enumerate}\\
\hline
Tabrizi, S. S., Cavus, N. (2016). A hybrid KNN-SVM model for Iranian license plate recognition. \emph{Procedia Computer Science, 102}, pp. 588-594. & Apakah dengan menggabungkan metode klasifikasi \textit{K-Nearest Neighbours} dan \textit{Support Vector Machine} akan menghasilkan akurasi yang lebih baik dan mengurangi \textit{cost} pada proses pengenalan karakter plat nomor kendaraan? &
\begin{enumerate}[wide, labelwidth=!, labelindent=0pt, topsep=0pt]
\item \textit{Structural Feature}
\item \textit{Horizontal and Vertical Crossing Count Histogram}
\item \textit{Zoning Feature Extraction}
\item \textit{K-Nearest Neigbhours}
\item \textit{Support Vector Machine}
\end{enumerate}\\
\hline
Rasheed, S., Naeem, A., Ishaq, O. (2012, October). Automated number plate recognition using hough lines and template matching. In \emph{Proceedings of the World Congress on Engineering and Computer Science} (Vol. 1, pp. 24-26). & Apakah dengan menggunakan metode \textit{Hough Transform} untuk mendeteksi plat kendaraan dan \textit{Template Matching} untuk pengenalan karakter dapat menghasilkan akurasi yang baik untuk sistem pengenalan plat nomor kendaraan? &
\begin{enumerate}[wide, labelwidth=!, labelindent=0pt, topsep=0pt]
	\item \textit{Canny Detector}
	\item \textit{Hough Transform}
	\item \textit{Morphological Process}
	\item \textit{Template Matching}
\end{enumerate}
\label{tbl:StateoftheArt}\\
\hline
\end{longtable}
\end{small}
\endgroup
 
\subsection{Pembahasan Penelitian Terkait}
\noindent Terdapat beberapa metode yang dapat khususnya untuk mendeteksi plat nomor dan mengenali karakter pada plat nomor.
\noindent Pada referensi pertama \cite{gou2016} menggunakan metode \textit{Extremal Region} sebagai proses untuk mendapatkan daerah-daerah karakter dari suatu plat nomor, kemudian \textit{Extremal Region} yang didapat diseleksi dengan menggunakan \textit{AdaBoost} sehingga bisa didapatkan daerah karakter plat nomor yang sesuai dengan kriteria yang diinginkan, dari daerah-daerah karakter yang didapatkan barulah kandidat plat nomor yang benar bisa didapatkan. Proses selanjutnya adalah pengambilan fitur karakter dengan menggunakan metode \textit{Histogram of Oriented Gradient} sehingga didapatkan fitur vektor dari setiap karakter pada plat nomor, yang nantinya akan menjadi masukkan bagi metode klasifikasi \textit{Hybrid Discriminative Restricted Boltzmann Machine}.

\noindent Pada referensi kedua \cite{gou2014} menggunakan metode \textit{Maximally Stable Extremal Region} untuk memilih kandidat daerah karakter yang nantinya akan menentukan lokasi dari plat nomor berdasarkan letak geometris dari kandidat-kandidat karakter tersebut. Setelah lokasi plat nomor didapatkan, \textit{HOG Descriptor} dari setiap karakter diambil dengan menggunakan metode \textit{Histogram of Oriented Gradient} dan setiap karakternya akan dikenali menggunakan metode \textit{neural network} bernama \textit{Extreme Learning Machine}.

\noindent Pada referensi ketiga \cite{tabrizi} digabungkan metode klasifikasi \textit{K-Nearest Neighbours} dengan metode klasifikasi \textit{Support Vector Machine}. \textit{K-Nearest Neighbours} digunakan karena sifatnya yang mudah dipelajari, bersifat tangguh terhadap data yang memiliki derau dan efektif jika jumlah yang dimiliki berjumlah banyak. Sedangkan metode \textit{Support Vector Machine} digunakan untuk karakter-karakter yang memiliki kemiripan karakteristik, sehingga akurasi dari pengenalan karakter dapat meningkat.

\noindent Pada referensi keempat \cite{rasheed} menggunakan \textit{Hough Transform} untuk mendeteksi lokasi plat kendaraan dan menghasilkan akurasi yang baik, yaitu sekitar 94\% untuk plat nomor yang terdeteksi dan dengan menggunakan metode \textit{Template Matching} menghasilkan akurasi pengenalan karakter plat nomor kendaraan sebesar 90\%.

%\section{Tinjauan Objek}
%\noindent Pada bagian ini akan diulas mengenai objek-objek yang terkait dengan deteksi dan pelacakan manusia.\\

%\subsection{Manusia}
%\noindent Berdasarkan pengertian dari Kamus Besar Bahasa Indonesia, manusia adalah makhluk yang berakal budi (mampu menguasai makhluk lain). Dalam bidang taksonomi, manusia yang akan dideteksi dan dilacak merupakan manusia modern atau \textit{homo sapiens}. Dan dalam perkembangannya, manusia memiliki beberapa tahap kehidupan yaitu balita, anak-anak, remaja, dewasa, dan usia lanjut. \\

%\subsection{Struktur Tubuh Manusia}
%\noindent Secara umum, tubuh manusia terdiri atas kepala, leher, batang tubuh, sepasang lengan, dan sepasang kaki. Pada penelitian ini untuk meningkatkan akurasi pendeteksian manusia maka digunakan bagian tubuh atas saja yaitu mulai dari bahu hingga bagian kepala manusia. \\

%\subsubsection{Pendeteksian dan Pelacakan Manusia}
%\noindent Pada bagian ini akan dijelaskan beberapa teori mengenai deteksi manusia dan pelacakan manusia.\\

%\subsubsection{Deteksi dan Melacak Manusia}
%\noindent Pendeteksian manusia adalah proses atau tahap awal untuk menemukan dan menentukan posisi objek manusia dari sebuah citra. Pendeteksian manusia ini fokus untuk menemukan bagian atas badan manusia dari tampak depan maupun tampak belakang, seperti yang terlihat pada gambar \ref{fig:ContohDeteksiManusia}. Sementara pelacakan manusia adalah proses untuk mengikuti jejak pergerakan dari objek manusia yang telah terdeteksi pada tahap sebelumnya. Tahap ini menggunakan estimasi perubahan atau pergerakan sehingga tidak perlu terus-menerus melakukan proses pendeteksian manusia pada setiap citra.\\
%\begin{adjustbox}{width=1\textwidth}
%\noindent\begin{minipage}{\linewidth}
%	\framebox[\textwidth]{\includegraphics[width=8cm]{images/DeteksiManusia.jpg}}
%	\captionof{figure}{Contoh dari pendeteksian manusia \cite{15}}
%	\label{fig:ContohDeteksiManusia}
%\end{minipage}
%\end{adjustbox}\\

%\subsection{\textit{Database RGB-D}}
%\noindent \textit{Database RGB-D} didapatkan dari \textit{Fudan University}. \textit{Fudan University} memiliki dan menyediakan \textit{database} yang dapat digunakan secara gratis oleh umum. Dalam penelitian ini digunakan 2 buah \textit{dataset} positif yang terdapat manusia di dalamnya, ditangkap oleh \textit{Fudan University} menggunakan sensor Microsoft Kinect yang memiliki resolusi berukuran 640 $\times$ 480 dengan \textit{frame rate} 30 fps. Sensor yang digunakan Microsoft Kinect dapat menangkap sepasang citra dalam waktu bersamaan yaitu citra RGB dan citra kedalaman sehingga menjadi citra RGB-D.

%\noindent Dalam masing-masing \textit{dataset} terdapat sepasang kumpulan citra yaitu citra tipe RGB dan citra kedalaman. Selain itu, diberitahukan juga koordinat \textit{ground truth} pada setiap pasangan citra yang menandakan posisi manusia yang terdapat pada citra tersebut.
%\textit{Dataset} yang pertama bernama \textit{Clothing Store RGBD}, berada pada salah satu toko pakaian dengan jumlah citra 1.000 buah untuk masing-masing tipe citra. Dalam 1.000 citra tersebut terdapat 2.367 posisi manusia pada berkas \textit{ground truth}. Contoh sepasang citra yang terdapat pada \textit{dataset} pertama dapat dilihat pada gambar \ref{fig:DatasetClothingStore}, di mana seharusnya terdapat 2 orang yang terdeteksi. \\
%\begin{adjustbox}{width=1\textwidth}
%\noindent\begin{minipage}{\linewidth}
%	\framebox[\textwidth]{\includegraphics[width=12cm]{images/DBClothingStore.jpg}}
%	\captionof{figure}{\textit{Dataset} CLOTHING STORE \cite{15}}
%	\label{fig:DatasetClothingStore}
%\end{minipage}
%\end{adjustbox}\\

%\noindent Pada gambar \ref{fig:GroundTruthClothingStore} terlihat \textit{ground truth} dari setiap citra \textit{dataset Clothing Store}, di mana angka awal sebelum titik dua merupakan citra pada waktu \textit{fps} tertentu dan setiap kurung siku menyatakan keberadaan posisi manusia pada citra, yang masing-masing terdiri atas $x_{1},y_{1}$ untuk koordinat titik kiri atas dan $x_{2},y_{2}$ untuk koordinat titik kanan bawah. \\
%\begin{adjustbox}{width=1\textwidth}
%\noindent\begin{minipage}{\linewidth}
%	\framebox[\textwidth]{\includegraphics[width=8cm]{images/GroundTruthClothingStore.jpg}}
%	\captionof{figure}{Berkas \textit{ground truth} CLOTHING STORE \cite{15}}
%	\label{fig:GroundTruthClothingStore}
%\end{minipage}
%\end{adjustbox}\\

%\noindent Untuk dataset kedua bernama \textit{Outdoor crowds in the dark RGBD}, berada di lingkungan \textit{Fudan University} pada malam hari sehingga kondisi pencahayaannya relatif redup dengan jumlah citra total adalah 275 buah untuk masing-masing tipe citra. Berbeda dengan \textit{dataset} pertama, pada \textit{dataset Outdoor} terdapat tiga buah folder bernama 31, 54, dan 56. Jumlah citra pada masing-masing folder bervariasi dan telah dilengkapi dengan berkas \textit{ground truth}. Penjelasan lebih rinci mengenai \textit{dataset} ini dapat dilihat pada tabel \ref{tbl:Outdoor}.
%\begin{table}[H]
%\centering
%\begin{small}
%\captionof{table}{Rincian \textit{Dataset Outdoor} \label{tbl:Outdoor}}
%\begin{tabular}{|p{1cm}|p{3cm}|p{3cm}|}
%	\hline
%	\textbf{Folder} & \textbf{Jumlah Citra} & \textbf{Jumlah Posisi Manusia}\\
%	\hline
%	31 & 57 & 273 \\
%	\hline
%	54 & 95 & 427 \\
%	\hline
%	56 & 123 & 548 \\
%	\hline
%\end{tabular}
%\end{small}
%\end{table}

%\noindent Contoh sepasang citra yang terdapat pada dataset kedua dapat dilihat pada gambar \ref{fig:DatasetOutdoor}, di mana seharusnya terdapat 6 orang yang terdeteksi. \\
%\begin{adjustbox}{width=1\textwidth}
%\noindent\begin{minipage}{\linewidth}
%	\framebox[\textwidth]{\includegraphics[width=12cm]{images/DBOutdoor.jpg}}
%	\captionof{figure}{\textit{Dataset} OUTDOOR \cite{15}}
%	\label{fig:DatasetOutdoor}
%\end{minipage}
%\end{adjustbox}\\

%\noindent Pada gambar \ref{fig:GroundTruthOutdoor} terlihat \textit{ground truth} dari setiap citra \textit{dataset Outdoor}, di mana nilai yang berada dalam tanda kutip dua merupakan lokasi dengan nama citra dan setiap nilai di dalam tanda kurung menyatakan keberadaan posisi manusia pada citra, yang masing-masing terdiri atas $x_{1},y_{1}$ untuk koordinat titik kiri atas dan $x_{2},y_{2}$ untuk koordinat titik kanan bawah. \\
%\begin{adjustbox}{width=1\textwidth}
%\noindent\begin{minipage}{\linewidth}
%	\framebox[\textwidth]{\includegraphics[width=8cm]{images/GroundTruthOutdoor.jpg}}
%	\captionof{figure}{Berkas \textit{ground truth} OUTDOOR \cite{15}}
%	\label{fig:GroundTruthOutdoor}
%\end{minipage}
%\end{adjustbox}\\

%\noindent \textit{Dataset} pertama akan dibagi dengan perbandingan 80:20, 80\% akan digunakan sebagai \textit{data training} dan 20\% digunakan sebagai {data testing}. Untuk \textit{dataset} kedua, folder 56 akan digunakan sebagai \textit{data training} dan folder 31 dan 54 digunakan sebagai {data testing}. \textit{Dataset} negatif yang tidak mengandung manusia akan diambil secara acak berukuran 65 $\times$ 80 dari masing-masing \textit{dataset} dengan ketentuan, tidak berpotongan dengan posisi \textit{ground truth}.

\newpage
	\setcounter{page}{1}
	%-----------------------------------------------------------------------------%
\chapter{ANALISIS DAN PERANCANGAN SISTEM}
%-----------------------------------------------------------------------------%

%
\vspace{4.5pt}

\noindent Bab ini memaparkan analisis masalah yang diatasi berserta pendekatan dan alur kerja dari perangkat lunak yang dikembangkan, mengimplementasikan metode yang digunakan dan hasil yang akan ditampilkan.
\\
\section{Analisis Masalah}
\noindent Pada bab 1 telah dijelaskan bahwa penelitian mengenai sistem pengenalan plat nomor kendaraan merupakan bidang yang masih berkembang dan implementasinya memegang peranan penting dalam bidang transportasi. Pada penelitian ini, metode yang akan digunakan adalah \textit{Hough Transform} untuk mendeteksi lokasi plat kendaraan, kemudian menggunakan metode \textit{Histogram of Oriented Gradient} untuk mengekstraksi fitur dari citra karakter dari plat nomor yang sudah disegmentasi dengan menghitung grafik horizontal pita, kemudian dilakukan klasifikasi dengan menggunakan metode \textit{Support Vector Machine}.
\noindent Masukan untuk sistem deteksi dan pengenalan plat nomor kendaraan ini adalah citra yang ditangkap oleh kamera DSLR Canon EOS 500 D dan Canon EOS 550 D beresolusi 15 dan 18 megapiksel. Citra tangkapan kemudian akan diubah resolusinya menjadi 1024 $\times$ 640 piksel. Setiap citra masukan berisi bagian depan dari kendaraan yang memiliki plat nomor kendaraan.
\noindent Keluaran atau hasil dari sistem akan berupa teks hasil dari pengenalan karakter pada citra plat nomor kendaraan masukan.\\ 

\section{Kerangka Pemikiran}
\noindent Berikut ini adalah kerangka pemikiran dari metode yang diusulkan untuk melakukan deteksi plat nomor kendaraan dan melakukan pengenalan karakter pada citra karakter yang terdapat pada plat nomor.
\\
\begin{adjustbox}{width=1\textwidth}
	\noindent
	\begin{minipage}{\linewidth}
		\framebox[\textwidth]{\includegraphics[width=13cm]{images/KerangkaPemikiran.png}}
		\captionof{figure}{Kerangka Pemikiran}
		\label{fig:KerangkaPemikiran}
	\end{minipage}
\end{adjustbox}

\noindent Seperti pada gambar \ref{fig:KerangkaPemikiran}, terdapat beberapa variabel indikator yang memengaruhi hasil dan perlu dilakukan penyesuaian, seperti ukuran sel pada metode \textit{Histogram of Oriented Gradient}, jumlah \textit{bin} yang menentukan batasan sudut yang digunakan, dan nilai sigma untuk \textit{classifier} \textit{Support Vector Machine}. Penelitian ini memiliki tujuan untuk menerapkan \textit{Histogram of Oriented Gradient} dan \textit{Support Vector Machine} untuk sistem pengenalan karakter pada plat nomor dengan menguji beragam faktor yang diduga akan mempengaruhi hasil akurasi dari penggabungan kedua metode tersebut. Hasil pengenalan karakter akan diukur dengan menggunakan \textit{Confusion Matrix}.\\

\section{Urutan Proses Global}
\noindent Dalam sistem pengenalan plat nomor kendaraan terbagi atas dua proses yaitu proses \textit{training} dan proses \textit{testing}. Proses \textit{training} dilakukan untuk mendapatkan kelas-kelas dari karakter-karakter yang akan dikenali. Proses \textit{testing} dilakukan untuk menghitung hasil yang berupa akurasi dari pengenalan karakter pada plat nomor kendaraan.\\

\begin{adjustbox}{width=1\textwidth}
	\noindent
	\begin{minipage}{\linewidth}
		\framebox[\textwidth]{\includegraphics[width=12cm]{images/FlowchartGlobal.png}}
		\captionof{figure}{\textit{Flowchart Global} Sistem Pengenalan Plat Nomor Kendaraan\\}
		\label{fig:FlowchartGlobal}
	\end{minipage}
\end{adjustbox}

\noindent Seperti pada gambar \ref{fig:FlowchartGlobal}. Bagian kiri merupakan \textit{flowchart} dari proses \textit{training} dan bagian kanan merupakan \textit{flowchart} dari proses \textit{testing}. Berikut ini adalah uraian dari proses-proses yang terjadi ketika tahapan \textit{training}:
\begin{enumerate}
	\item Citra masukan adalah citra plat nomor kendaraan mobil yang di-\textit{crop} secara manual dari citra mobil utuh hal ini untuk memastikan citra yang didapatkan adalah plat yang benar. 
	\item Citra plat kemudian akan melalui tahapan \textit{preprocessing} yang meliputi \textit{grayscaling} untuk menghilangkan informasi warna yang tidak diperlukan, \textit{Gaussian Smoothing} untuk menghilangkan derau pada citra, \textit{Binarization} untuk mengubah citra menjadi citra biner, menghilangkan objek kecil (untuk menghilangkan objek seperti baut pada plat) dengan cara menghilangkan objek yang luasnya kurang dari \textit{threshold} sebesar 100 piksel, dan terakhir \textit{Inverse Binarization} untuk mendapatkan citra karakter berwarna hitam dengan latar belakang berwarna putih.
	\item Citra plat selanjutnya melalui tahapan segmentasi karakter, tahapan segmentasi bertujuan untuk mendapatkan citra-citra karakter dari plat nomor tersebut tahapan segmentasi dilakukan dua tahapan, yaitu segmentasi horizontal dan segmentasi vertikal.
	\item Citra karakter hasil \textit{segmentasi} akan di-\textit{scaling} menjadi berukuran 32 $\times$ 32 piksel. Kumpulan karakter yang digunakan adalah karakter angka dari 0 sampai dengan 9 dan karakter huruf dari A sampai dengan Z, tidak ada karakter huruf kecil dikarenakan plat nomor kendaraan tidak ada yang menggunakan karakter huruf kecil.
	\item \textit{Histogram of Oriented Gradient} berfungsi untuk mendapatkan fitur dari citra hasil segmentasi. Hasil dari ekstraksi fitur dengan menggunakan HOG adalah \textit{HOG descriptor}, yang mendeskripsikan distribusi dari gradien berarah pada suatu area citra.
	\item Untuk ukuran sel dan jumlah \textit{bin} yang digunakan untuk proses ekstraksi fitur dengan menggunakan \textit{HOG}, dikarenakan kedua parameter tersebut merupakan indikator uji seperti ditunjukan pada kerangka pemikiran \ref{fig:KerangkaPemikiran}, maka nilai dari ukuran sel dan jumlah \textit{bin} akan menyesuaikan dengan kondisi yang akan diuji.
	\item Setelah tahapan ekstraksi fitur, fitur-fitur akan disimpan dalam berkas CSV dan proses pelabelan fitur untuk setiap karakter akan dilakukan terhadap berkas CSV tersebut.
	\item \textit{Support Vector Machine} (SVM) digunakan untuk mengklasifikasikan fitur-fitur yang sudah didapatkan ke dalam kelas-kelas dari karakter yang akan dikenali. Metode \textit{SVM} yang digunakan pada penelitian ini berasal dari \textit{library WEKA}.
\end{enumerate}
%\subsection{Proses \textit{Training}}
%\begin{adjustbox}{width=1\textwidth}
%	\noindent
%	\begin{minipage}{\linewidth}
%		\framebox[\textwidth]{\includegraphics[width=12cm]{images/FlowchartTraining.jpg}}
%		\captionof{figure}{\textit{Flowchart Training} Sistem Pengenalan Plat Nomor Kendaraan\\}
%		\label{fig:FlowchartTraining}
%	\end{minipage}
%\end{adjustbox}
%Berikut ini adalah uraian dari \textit{flowchart} pada gambar \ref{fig:FlowchartTraining} yang dilakukan dalam penelitian ini:
%\begin{enumerate}
%	\item Citra yang menjadi masukkan adalah citra karakter hasil segmentasi dari citra plat nomor kendaraan. Citra karakter masukkan berukuran 32 $\times$ 32 piksel. Citra karakter berwarna hitam dengan latar belakang berwarna putih. Kumpulan karakter yang digunakan adalah karakter angka dari 0 sampai dengan 9 dan karakter huruf dari A sampai dengan Z, tidak ada karakter huruf kecil dikarenakan plat nomor kendaraan tidak ada yang menggunakan karakter huruf kecil.
%	\item \textit{Histogram of Oriented Gradient} berfungsi untuk mendapatkan fitur dari dari citra masukan. Hasil dari ekstraksi fitur dengan menggunakan HOG adalah \textit{HOG descriptor}, yang mendeskripsikan distribusi dari gradien berarah pada suatu area citra.
%	\item Ukuran sel dan blok yang digunakan untuk proses ekstraksi fitur dengan menggunakan \textit{HOG} adalah beragam sesuai dengan ukuran-ukuran sel yang akan digunakan untuk proses testing dan jumlah \textit{bin} yang digunakan juga akan beragam sesuai dengan ukuran \textit{bin} yang digunakan untuk proses testing. Ukuran sudut yang akan dipakai adalah dari 0 sampai dengan 180 derajat.
%	\item \textit{Support Vector Machine} (SVM) digunakan untuk mengklasifikasikan fitur-fitur yang sudah didapatkan ke dalam kelas-kelas dari karakter yang akan dikenali. Metode \textit{SVM} yang digunakan pada penelitian ini berasal dari \textit{library WEKA}.\\
%\end{enumerate}
%
%\subsection{Proses \textit{Testing}}
%\begin{adjustbox}{width=1\textwidth}
%	\noindent
%	\begin{minipage}{\linewidth}
%		\framebox[\textwidth]{\includegraphics[width=7cm]{images/FlowchartTesting.png}}
%		\captionof{figure}{\textit{Flowchart Testing} Sistem Deteksi dan Pengenalan Plat Nomor\\}
%		\label{fig:FlowchartTesting}
%	\end{minipage}
%\end{adjustbox}
\noindent Kemudian untuk proses \textit{testing}, seperti terlihat pada gambar \ref{fig:FlowchartGlobal}, terdapat beberapa proses yang sama seperti pada proses \textit{training}. Berikut ini adalah uraian dari \textit{flowchart} bagian kanan pada gambar \ref{fig:FlowchartGlobal} yang dilakukan dalam penelitian ini:
\begin{enumerate}
	\item Citra pengujian yang digunakan didapatkan dari \textit{dataset} plat nomor kendaraan Universitas Telkom yang bernama \textit{Tel-U Vehicle Data-set V1.0}, penggunaan dari dataset ini sesuai dengan perizinan dari institusi yang bersangkutan. 
	\item Citra kendaraan akan melalui tahapan deteksi area plat kendaraan untuk mendapatkan citra plat.
	\item Citra plat yang didapatkan kemudian disegmentasi untuk mendapatkan citra karakter. 
	\item Citra yang akan menjadi input dari \textit{HOG} adalah citra hasil dari segmentasi karakter pada citra plat kendaraan hasil deteksi lokasi plat nomor kendaraan.
	\item Ukuran dari sel dan blok yang digunakan untuk proses ekstraksi fitur dengan menggunakan \textit{HOG} akan beragam sesuai dengan pengujian yang akan dilakukan.
	\item Pada tahap \textit{testing} model SVM yang digunakan berasal dari hasil keluaran model SVM pada tahap \textit{training}.
	\item Hasil keluaran akan berupa sebuah \textit{string} yang menunjukkan kumpulan karakter yang berhasil dikenali oleh sistem.\\
\end{enumerate}

\section{Analisis Manual}
\noindent Pada bagian ini dilakukan analisis tahapan proses dengan melakukan perhitungan manual.\\

\subsection{\textit{Dataset}}
\noindent Citra plat kendaraan yang digunakan sebagai dataset adalah citra plat yang di-\textit{crop} secara manual seperti terlihat pada gambar \ref{fig:ContohCitraPlat}.

\begin{adjustbox}{width=1\textwidth}
	\noindent
	\begin{minipage}{\linewidth}
		\framebox[\textwidth]{\includegraphics[width=8cm]{images/ContohPlat.PNG}}
		\captionof{figure}{Contoh citra plat yang digunakan\\}
		\label{fig:ContohCitraPlat}
	\end{minipage}
\end{adjustbox}

\noindent Dari citra plat akan dilakukan tahapan \textit{preprocessing}. Hasil akhir dari tahapan-tahapan tersebut dapat dilihat pada gambar \ref{fig:ContohCitraPlatHasilPreprocessing}.

\begin{adjustbox}{width=1\textwidth}
	\noindent\begin{minipage}{\linewidth}
		\framebox[\textwidth]{\includegraphics[width=8cm]{images/ContohHasilPreprocessing.PNG}}
		\captionof{figure}{Contoh citra plat setelah proses \textit{preprocessing}\\}
		\label{fig:ContohCitraPlatHasilPreprocessing}
	\end{minipage}
\end{adjustbox}

\noindent Dari citra plat hasil \textit{preprocessing}, berikutnya dilakukan proses segmentasi terhadap citra plat. Segmentasi horizontal bertujuan untuk memisahkan area karakter plat nomor dengan karakter tanggal masa berlaku plat nomor. Sedangkan segmentasi vertikal bertujuan untuk mendapatkan karakter karakter. Hasilnya dapat dilihat pada gambar \ref{fig:CitraPlatHasilSegmentasiHorizontal} dan gambar \ref{fig:CitraPlatHasilSegmentasiVertikal}.

\begin{adjustbox}{width=1\textwidth}
	\noindent
	\begin{minipage}{\linewidth}
		\framebox[\textwidth]{\includegraphics[width=8cm]{images/HasilSegmentasiHorizontal.PNG}}
		\captionof{figure}{Citra plat setelah proses segmentasi horizontal\\}
		\label{fig:CitraPlatHasilSegmentasiHorizontal}
	\end{minipage}
\end{adjustbox}

\begin{adjustbox}{width=1\textwidth}
	\noindent\begin{minipage}{\linewidth}
		\framebox[\textwidth]{\includegraphics[width=14cm]{images/HasilSegmentasiVertikal.PNG}}
		\captionof{figure}{Citra plat setelah proses segmentasi vertikal\\}
		\label{fig:CitraPlatHasilSegmentasiVertikal}
	\end{minipage}
\end{adjustbox}

\noindent Setelah didapatkan karakter-karakter dari plat nomor tersebut, berikutnya dilakukan proses \textit{scaling} dari ukuran seperti pada gambar \ref{fig:CitraPlatHasilSegmentasiVertikal} menjadi ukuran 32 $\times$ 32 piksel. Alasan digunakannya ukuran 32 $\times$ 32 piksel adalah agar ukuran fitur dari \textit{HOG Descriptor} yang dihasilkan tidak terlalu besar. Hasil dari proses \textit{scaling} citra karakter dapat dilihat pada gambar \ref{fig:ContohCitraDatasetPakai}. Citra karakter hasil proses \textit{scaling} itulah yang akan digunakan sebagai data latih.

\noindent Karakter terdiri dari angka 0 sampai dengan 9 dan karakter huruf kapital dari A sampai dengan Z. Untuk setiap karakter akan digunakan citra latih sebanyak tiga sampai dengan enam citra. Terlihat contoh citra karakter yang ditunjukan pada gambar \ref{fig:ContohCitraDatasetPakai} merupakan contoh karakter angka dan huruf kapital yang akan dipakai.

\begin{adjustbox}{width=1\textwidth}
	\noindent\begin{minipage}{\linewidth}
		\framebox[\textwidth]{\includegraphics[width=8cm]{images/CitraDatasetPakai.PNG}}
		\captionof{figure}{Contoh citra karakter yang digunakan untuk tahap \textit{training}\\}
		\label{fig:ContohCitraDatasetPakai}
	\end{minipage}
\end{adjustbox}
\\

\subsection{Tahap Pendeteksian Lokasi Plat Nomor}
\noindent Skema alur dari tahap pendeteksian lokasi plat nomor adalah:

\begin{adjustbox}{width=1\textwidth}
	\noindent
	\begin{minipage}{\linewidth}
		\framebox[\textwidth]{\includegraphics[width=13cm]{images/FlowchartDeteksiPlat.png}}
		\captionof{figure}{Skema Alur Pendeteksian Plat\\}
		\label{fig:SkemaAlurPendeteksianPlat}
	\end{minipage}
\end{adjustbox}\\

\subsubsection{\textit{Grayscale}}
\noindent Proses pertama adalah mengubah citra masukan dari citra RGB menjadi citra \textit{grayscale}, tujuan dari \textit{grayscaling} citra adalah untuk menghilangkan informasi warna dari setiap piksel citra. Untuk menghitung nilai derajat keabuan setiap piksel, diperoleh dengan menggunakan persamaan \ref{eq:grayscale}.
\noindent Di bawah merupakan contoh matriks citra asli dengan 3 \textit{channel} warna yaitu \textit{Red}, \textit{Green}, dan \textit{Blue} berukuran 5 $\times$ 5 piksel. 

\begin{adjustbox}{width=1\textwidth}
	\noindent\begin{minipage}{\linewidth}
		\framebox[\textwidth]{\includegraphics[width=8cm]{images/CitraMatriksAsal.PNG}}
		\captionof{figure}{Matriks Citra Asal berukuran 5 $\times$ 5\\}
		\label{fig:MatriksCitraAsal}
	\end{minipage}
\end{adjustbox} 

\noindent Dengan menggunakan persamaan \ref{eq:grayscale}, maka nilai matriks citra \textit{grayscale} pada titik (4,4) akan menjadi sebagai berikut:
\begin{table}[H]
	\begin{adjustbox}{width=1\textwidth}
		\begin{tabular}{|p{13.55cm}|}
			\hline
			\begin{equation}\nonumber
			\begin{aligned}
			Matriks[4,4] &= (0.299 * 133) + (0.587 * 138) + (0.114 * 144) \\
						 &= 137.189 \approx 137 
			\end{aligned}
			\end{equation}\\
			\hline
		\end{tabular}
	\end{adjustbox}
\end{table}
\noindent Perhitungan di atas dilakukan terhadap seluruh nilai matriks citra asal dan hasilnya adalah matriks citra berukuran 5 $\times$ 5 dengan satu nilai derajat keabuan.

\begin{adjustbox}{width=1\textwidth}
	\noindent\begin{minipage}{\linewidth}
		\framebox[\textwidth]{\includegraphics[width=6cm]{images/CitraMatriksGrayscale.PNG}}
		\captionof{figure}{Matriks Citra Hasil \textit{Grayscale}\\}
		\label{fig:MatriksCitraGrayscale}
	\end{minipage}
\end{adjustbox} \\

\subsubsection{Deteksi Tepi Canny}
\noindent Proses deteksi tepi dilakukan terhadap citra hasil \textit{grayscaling}. Pada penelitian ini, metode \textit{Canny Edge Detection} digunakan untuk mendapatkan tepian. Berikut adalah algoritme dari metode \textit{Canny Edge Detection} untuk mendapatkan tepian.
\begin{enumerate}[leftmargin=16pt]
	\item Citra masukan adalah citra dari hasil \textit{grayscaling} pada tahapan sebelumnya.
	\item Citra masukan diperhalus dengan menggunakan \textit{Gaussian Filter} untuk membuang derau.
	\item Lakukan operasi perhitungan gradien menggunakan operator Sobel untuk mendapatkan tepian yang tebal.
	\item Untuk menipiskan tepian yang didapat dari operasi sebelumnya maka teknik \textit{Non-Maxima Suppression} dilakukan dengan mencari nilai maksimum pada tepian.
	\item Buat 2 nilai \textit{threshold} yaitu \textit{high threshold} dan \textit{low threshold} untuk menentukan piksel mana yang masuk dalam kategori tepian kuat, tepian lemah, dan bukan tepian. Jika nilai dari piksel tersebut di atas \textit{high threshold}, maka piksel tersebut masuk ke dalam kategori tepian kuat, apabila nilai piksel berada di antara batas \textit{high threshold} dan \textit{low threshold}, maka piksel tersebut masuk ke dalam kategori tepian lemah, selebihnya akan masuk ke dalam kategori bukan tepian.
	\item Tahapan terakhir adalah \textit{Edge Linking} untuk menghubungkan tepian lemah dengan tepian kuat. Apabila piksel tepian lemah memiliki tetangga piksel (terhubung), maka piksel tersebut akan menjadi tepian.
	\item Citra keluaran adalah citra biner yang merupakan hasil pendeteksian tepi.
	
	\begin{adjustbox}{width=1\textwidth}
		\noindent
		\begin{minipage}{\linewidth}
			\framebox[\textwidth]{\includegraphics[width=13cm]{images/HasilCanny.png}}
			\captionof{figure}{Contoh Citra hasil deteksi tepi \textit{Canny}\\}
			\label{fig:HasilDeteksiTepi}
		\end{minipage}
	\end{adjustbox}\\
\end{enumerate}

\subsubsection{\textit{Hough Transform}}
\noindent Metode \textit{Hough Transform} yang digunakan adalah untuk identifikasi garis lurus. Dalam ekstraksi fitur \textit{Hough Transform} perlu menspesifikasikan \textit{accumulator space} untuk menyimpan nilai \textit{voting}. Untuk tahapan \textit{Hough Transform} ini akan menggunakan \textit{library} dari \textit{OpenCV} yaitu dengan menggunakan \textit{function} \textit{Imgproc.HoughLines()}. \textit{Function} \textit{Imgproc.HoughLines()} ini memiliki parameter masukan berupa citra biner hasil dari metode \textit{Canny Edge Detection}, \textit{range} sudut yang akan digunakan sebagai $\theta$ untuk membatasi sudut yang akan dicari, dan  nilai rho yang merupakan panjang garis dalam piksel. Spesifikasi \textit{accumulator space} ditentukan berdasarkan ukuran citra input. Berikut adalah algoritme dari metode \textit{Hough Transform} untuk mendapatkan garis lurus:
\begin{enumerate}
	\item Citra masukan adalah citra biner hasil deteksi tepi pada tahapan sebelumnya.
	\item Matriks \textit{Accumulator Space} didefinisikan sebagai \textit{array} 2 dimensi dengan sumbu horizontal menunjukkan nilai sudut ($\theta$) yang digunakan dan sumbu vertikal adalah nilai-nilai dari $\rho$.
	
	\begin{adjustbox}{width=1\textwidth}
		\noindent
		\begin{minipage}{\linewidth}
			\centering\framebox[8cm]{\includegraphics[width=8cm]{images/accumulatorspace.jpg}}
			\captionof{figure}{Ilustrasi matriks \textit{Accumulator Space}\\}
			\label{fig:MatriksAccumulatorSpace}
		\end{minipage}
	\end{adjustbox}
	\item Untuk setiap nilai piksel dari citra biner hasil deteksi tepian, apabila nilai piksel tersebut adalah 0, piksel tersebut diabaikan. Jika nilai piksel tersebut tidak 0, maka lakukan perhitungan nilai $\rho$ untuk piksel tersebut dengan menggunakan persamaan \ref{eq:PersamaanRho} dan lakukan \textit{voting} terhadap setiap nilai $\theta$ yang digunakan dengan menambahkan nilai pada matriks akumulator dengan koordinat ($\theta$, $\rho$) sebesar satu.
	\item Hasil dari perhitungan \textit{voting} akan dicari hasil-hasil \textit{voting} tertinggi untuk dijadikan kandidat garis melalui tahapan pencarian \textit{Hough Peaks}. Tahapan ini dilakukan dengan menentukan nilai \textit{threshold}, \textit{neighbourhood}, dan jumlah \textit{peaks} yang akan diambil.
	\item Hasil pencarian \textit{Hough Peaks} akan menghasilkan kumpulan nilai $\rho$ dan $\theta$, nilai ini kemudian diubah menjadi koordinat titik.
	\item Keluaran dari tahapan ini adalah \textit{array} yang berisi pasangan koordinat titik dari kandidat-kandidat garis yang didapat.\\
\end{enumerate}

\subsubsection{Tahap Validasi Plat Kendaraan}
\noindent Tahap selanjutnya setelah mendapatkan kandidat-kandidat garis adalah menyeleksi area plat nomor. Hal ini dilakukan melalui serangkaian tahapan sebagai berikut:
\begin{enumerate}
	\item Dari keseluruhan kandidat garis yang didapat, pisahkan kandidat garis vertikal dengan kandidat garis horizontal.
	\item Dari kandidat-kandidat garis vertikal akan ada yang menjadi batas kiri dan batas kanan dari area plat kendaraan. Setiap kandidat garis vertikal akan dipasangkan dengan garis vertikal lain dengan cara membandingkan mana garis yang lebih kanan. Hasilnya akan disimpan dalam \textit{array} yang berisi koordinat x dari masing-masing pasangan garis.
	\item Hitung lebar citra yang dibatasi dengan pasangan garis vertikal yang didapatkan, apabila lebar citra sesuai batasan ukuran yang ditentukan, maka pasangan garis tersebut akan menjadi kandidat dari batas kiri dan batas kanan dari plat nomor.
	\item Untuk setiap kandidat batas kiri dan batas kanan plat nomor, pasangkan dengan kandidat garis horizontal dan hitung tinggi citra yang dibatasi dengan batas horizontal, apabila tinggi citra sesuai dengan batasan ukuran yang ditentukan, maka pasangan garis tersebut akan menjadi batas atas dan batas bawah dari citra plat kendaraan.
	\item Jika masih terdapat lebih dari satu kandidat, maka pilih kandidat dengan rasio panjang : lebar yang paling mendekati rasio plat nomor kendaraan Indonesia, yaitu 1 : 3.
	\item Hasil dari tahapan ini adalah 4 titik koordinat yang merupakan koordinat citra plat. Citra plat akan diambil dari citra asal dengan menggunakan titik-titik koordinat tersebut.
\end{enumerate}

\begin{adjustbox}{width=1\textwidth}
	\noindent\begin{minipage}{\linewidth}
		\centering\includegraphics[width=8cm]{images/HasilPlat.png}
		\captionof{figure}{Contoh hasil citra plat\\}
		\label{fig:OutputPlat}
	\end{minipage}
\end{adjustbox}\\

\subsection{Tahapan Segmentasi Karakter}
\noindent Setelah mendapatkan kandidat plat, maka berikutnya dilakukan segmentasi karakter untuk mendapatkan citra-citra karakter yang terdapat pada plat nomor kendaraan. Pada tahapan segmentasi karakter, akan dilakukan dua tahapan, yaitu segmentasi vertikal untuk mendapatkan batas atas dan batas bawah daerah karakter, dan segmentasi horizontal untuk mendapatkan batas kiri dan batas kanan untuk setiap karakter. Untuk tahapan segmentasi karakter ini akan menggunakan \textit{method} dari \textit{library JavaOCR} seperti yang disebutkan pada \ref{tbl:FunctionJavaOCR}, yaitu \textit{LineExtractor.slice()} dan \textit{CharacterExtractor.slice()}. \textit{LineExtractor.slice()} digunakan untuk melakukan segmentasi horizontal, parameter dari \textit{method} tersebut adalah berkas citra plat yang sudah melalui tahapan \textit{preprocessing} dan berkas citra yang akan menampung hasil segmentasi horizontal. Sedangkan \textit{CharacterExtractor.slice()} digunakan untuk melakukan segmentasi vertikal dengan parameter citra hasil segmentasi horizontal, berkas citra yang akan menampung hasil segmentasi vertikal, dan dua parameter berikutnya adalah ukuran lebar dan tinggi citra untuk hasil proses \textit{scaling}. Berikut adalah langkah-langkah dari proses segmentasi karakter:
\begin{enumerate}
\item Citra masukan adalah citra plat hasil tahapan deteksi plat kendaraan yang sudah dilakukan \textit{preprocessing} menjadi citra biner.
\item Lakukan segmentasi vertikal untuk mendapatkan batas atas dan batas bawah dari area kandidat karakter.
\item Lakukan segmentasi horizontal untuk mendapatkan batas kiri dan batas kanan dari setiap citra karakter.
\item Setiap citra karakter yang didapatkan akan di-\textit{scaling} menjadi ukuran 32 $\times$ 32 piksel. Hal ini bertujuan untuk menjaga konsistensi ukuran citra karakter yang digunakan untuk proses \textit{training} dan proses \textit{testing}. Hasil dari segmentasi seperti yang ditunjukkan pada gambar \ref{fig:OutputSegmentasi} ditambahkan satu piksel lebih untuk setiap batas kiri, kanan, atas, dan bawah dari citra, tujuannya adalah agar bentuk karakter yang didapatkan ketika proses ekstraksi fitur dengan menggunakan metode \textit{Histogram of Oriented Gradient} menjadi lebih baik.
\item Keluaran dari tahapan ini adalah citra-citra karakter yang terdapat pada plat nomor.
\end{enumerate}

\noindent Pada penelitian ini, tahapan segmentasi karakter akan menggunakan \textit{library} dari Java OCR dan \textit{method} atau \textit{function} yang digunakan dapat dilihat pada tabel \ref{tbl:FunctionJavaOCR}.\\
\\
\begin{adjustbox}{width=1\textwidth}
	\noindent\begin{minipage}{\linewidth}
		\centering\includegraphics[width=14cm]{images/OutputSegmentasi.png}
		\captionof{figure}{Contoh hasil keluaran dari tahapan segmentasi\\}
		\label{fig:OutputSegmentasi}
	\end{minipage}
\end{adjustbox}\\

\subsection{\textit{Histogram of Oriented Gradient}}
\noindent Pada proses \textit{Histogram of Oriented Gradients}, masukan untuk proses ini berupa citra yang berasal dari hasil segmentasi. Perhitungan fitur dari metode \textit{Histogram of Oriented Gradient} ini dilakukan per citra karakter hasil segmentasi. Keluaran dari proses ini adalah matriks fitur vektor dari hasil perhitungan \textit{Histogram of Oriented Gradients}. Berikut merupakan langkah-langkah untuk menghitung matriks fitur vektor. Pada gambar dapat dilihat hasil dari proses \textit{resize} dan \textit{crop} citra \textit{grayscale} berukuran 8 $\times$ 4 piksel.

\begin{table}[H]
	\centering
	\begin{small}
		\begin{tabular}{|p{2cm}|p{2cm}|p{2cm}|p{2cm}|}
			\hline
			89 & 92 & 88 & 92 \\
			\hline
			90 & 88 & 90 & 86 \\
			\hline
			91 & 90 & 90 & 94 \\
			\hline
			91 & 122 & 91 & 122 \\
			\hline
			89 & 90 & 89 & 91 \\
			\hline
			90 & 85 & 90 & 86 \\
			\hline
			91 & 90 & 92 & 93 \\
			\hline
			91 & 122 & 91 & 120 \\
			\hline
		\end{tabular}
	\end{small}
	\captionof{figure}{Matriks citra hasil \textit{preprocessing}\\}
	\label{fig:MatriksCitraHasilPreprocessing}
\end{table}

\begin{enumerate}
\item Proses pertama adalah untuk menghitung nilai gradien dari posisi vertikal dan horizontal untuk setiap piksel menggunakan persamaan \ref{eq:PersamaanGradienX} dan \ref{eq:PersamaanGradienY}. Contoh perhitungannya untuk piksel koordinat (2,5) dan hasil dari tahap ini dapat dilihat pada gambar \ref{fig:MatriksCitraHasilGradienX} dan \ref{fig:MatriksCitraHasilGradienY} di bawah:
\begin{equation*}
	G_{x}(2,5) = 89 - 89 = 0
\end{equation*}
\begin{equation*}
	G_{y}(2,5) = 85 - 122 = -37
\end{equation*}
\begin{table}[H]
	\centering
	\begin{small}
		\begin{tabular}{|p{2cm}|p{2cm}|p{2cm}|p{2cm}|}
			\hline
			92 & -1 & 0 & -88 \\
			\hline
			88 & 0 & -2 & -90 \\
			\hline
			90 & -1 & 4 & -90 \\
			\hline
			122 & 0 & 0 & -91 \\
			\hline
			90 & 0 & 1 & -89 \\
			\hline
			85 & 0 & 1 & -90 \\
			\hline
			90 & 1 & 3 & -92 \\
			\hline
			122 & 0 & -2 & -91 \\
			\hline
		\end{tabular}
	\end{small}
	\captionof{figure}{Matriks hasil Perhitungan Gradien sumbu X\\}
	\label{fig:MatriksCitraHasilGradienX}
\end{table}
\begin{table}[H]
	\centering
	\begin{small}
		\begin{tabular}{|p{2cm}|p{2cm}|p{2cm}|p{2cm}|}
			\hline
			90 & 88 & 90 & 86 \\
			\hline
			2 & -2 & 2 & 2 \\
			\hline
			1 & 34 & 1 & 36 \\
			\hline
			-2 & 0 & -1 & -3 \\
			\hline
			-1 & -37 & -1 & -36 \\
			\hline
			2 & 0 & 3 & 2 \\
			\hline
			1 & 37 & 1 & 34 \\
			\hline
			-91 & -90 & -92 & -93 \\
			\hline
		\end{tabular}
	\end{small}
	\captionof{figure}{Matriks hasil Perhitungan Gradien sumbu Y\\}
	\label{fig:MatriksCitraHasilGradienY}
\end{table}
\item Untuk setiap piksel, hitung \textit{magnitude} gradien dan arah gradien menggunakan persamaaan \ref{eq:PersamaanMagnitude} dan \ref{eq:PersamaanArah}. Contoh perhitungannya untuk piksel koordinat (2,5) dan hasil dari tahap ini dapat dilihat pada gambar \ref{fig:MatriksHasilMagnitude}:
\begin{equation*}
M(2,5) = \sqrt{0^2 + (-37)^2} = 37
\end{equation*}
\begin{equation*}
\theta(2,5) = arctan\frac{-37}{0} \approx 90
\end{equation*}
\begin{table}[H]
	\centering
	\begin{small}
		\begin{tabular}{|p{2cm}|p{2cm}|p{2cm}|p{2cm}|}
			\hline
			128.70 & 88.01 & 90 & 123.05 \\
			\hline
			88.03 & 2 & 2.83 & 90.02 \\
			\hline
			90.01 & 34.02 & 4.12 & 96.93 \\
			\hline
			122.02 & 0 & 1 & 91.05 \\
			\hline
			90.01 & 37 & 1.41 & 96.01 \\
			\hline
			85.02 & 0 & 3.16 & 90.02 \\
			\hline
			90.01 & 37.01 & 3.16 & 98.08 \\
			\hline
			152.2 & 90 & 92.02 & 130.12 \\
			\hline
		\end{tabular}
	\end{small}
	\captionof{figure}{Matriks hasil Perhitungan \textit{Magnitude}\\}
	\label{fig:MatriksHasilMagnitude}
\end{table}
\begin{table}[H]
	\centering
	\begin{small}
		\begin{tabular}{|p{2cm}|p{2cm}|p{2cm}|p{2cm}|}
			\hline
			44.37 & 90.65 & 89.99 & 135.66 \\
			\hline
			1.30 & 90.03 & 135 & 178.73 \\
			\hline
			0.64 & 91.69 & 14.04 & 158.19 \\
			\hline
			179.06 & 0 & 90.06 & 1.89 \\
			\hline
			179.36 & 90 & 135 & 22.02 \\
			\hline
			1.35 & 0 & 71.57 & 178.73 \\
			\hline
			0.64 & 88.45 & 18.44 & 159.72 \\
			\hline
			143.28 & 90 & 88.76 & 45.62 \\
			\hline
		\end{tabular}
	\end{small}
	\captionof{figure}{Matriks hasil Perhitungan Arah\\}
	\label{fig:MatriksHasilPerhitunganArah}
\end{table}
\item Kemudian, tentukan ukuran sel, ukuran blok dan jumlah \textit{oriented histogram bins}. Pada penelitian Dalas dan Triggs untuk mendeteksi pejalan kaki sebelumnya, didapat bahwa ukuran sel sebesar 8 $\times$ 8 piksel, ukuran blok sebesar 2 $\times$ 2 ukuran sel dan jumlah bin sebanyak 9 sudah dapat menghasilkan akurasi yang dapat mendeteksi pejalan kaki dengan cukup baik dibandingkan dengan ukuran-ukuran lainnya. Untuk contoh perhitungan analisis kali ini jumlah \textit{oriented histogram bins} yang dipakai sebanyak 4 buah, sehingga didapat nilai sudut setiap \textit{histogram bin} yaitu 180 / 4 = 45. Untuk ukuran sel dipilih sebesar 2 $\times$ 2 piksel dan ukuran blok sebesar 2 $\times$ 2 sel. Untuk setiap blok, terdapat \textit{overlapping} sebesar 50\% dari ukuran blok. Dengan demikian akan didapatkan perhitungan sebagai berikut:
\begin{itemize}
\item Jumlah sel adalah 8, terdiri dari 4 sel vertikal dan 2 sel horizontal.
\item Jumlah blok adalah 3, terdiri dari 3 blok vertikal dan 1 blok horizontal.
\end{itemize}
\item Kemudian untuk setiap sel, tentukan perhitungan \textit{Histogram of Oriented Gradient} dengan melakukan \textit{voting} dari arah gradien dan \textit{magnitude} gradien, dimana arah gradien akan menjadi sudut \textit{bin}, dan \textit{magnitude} gradien akan menjadi bobot nilai. Berikut merupakan contoh proses \textit{voting} untuk piksel dengan koordinat (2,5).
\begin{equation*}
M(2,5) = 37
\end{equation*}
\begin{equation*}
\theta(2,5) = 90
\end{equation*}
Sehingga untuk \textit{bin} dengan sudut 90 akan mendapat nilai bobot sebesar 37 yang didapatkan dari nilai gradien \textit{magnitude}-nya.
Lakukan proses tersebut untuk setiap sel sehingga masing-masing sel akan mempunyai \textit{Histogram of Oriented Gradient}. Berikut contoh hasil perhitungan metode \textit{Histogram of Oriented Gradient} pada sel yang terdapat koordinat piksel (2,5) dapat dilihat pada gambar \ref{fig:HasilHOG}.\\
\begin{adjustbox}{width=1\textwidth}
	\noindent\begin{minipage}{\linewidth}
		\framebox[\textwidth]{\includegraphics[width=12cm]{images/HistogramOfOrientedGradient.PNG}}
		\captionof{figure}{Contoh hasil \textit{Histogram of Oriented Gradient} untuk sel yang memiliki piksel dengan koordinat (2,5)}
		\label{fig:HasilHOG}
	\end{minipage}
\end{adjustbox}
\item Kemudian untuk setiap blok, akan dilakukan normalisasi dengan menggabungkan hasil histogram dari setiap sel dalam bloknya. Adapun proses normalisasi dapat menggunakan 4 algoritme yaitu, \textit{L1-Norm}, \textit{L1-Sqrt}, \textit{L2-Norm}, dan \textit{L2-Hys}. Pada penelitian ini, penulis menggunakan algoritme normalisasi \textit{L2-Norm} karena berdasarkan penelitian sebelumnya, hasil yang didapat lebih baik dari algoritme lainnya. Persamaan algoritme untuk proses normalisasi menggunakan \textit{L2-Norm} didapat dengan menggunakan persamaan \ref{eq:L2-Norm}. Di bawah adalah contoh perhitungan normalisasi untuk blok pertama:
\begin{table}[H]
	\centering
	\begin{small}
		\begin{tabular}{|p{1cm}|p{1cm}|p{1cm}|p{1cm}|p{1cm}|p{1cm}|p{1cm}|p{1cm}|}
			\hline
			87.28 & 129.45 & 88.73 & 1.28 & 89.28 & 0 & 89.99 & 126.62 \\
			\hline
			208.2 & 1.27 & 32.74 & 3.82 & 140.04 & 5.11 & 0.99 & 46.96 \\
			\hline
			171.21 & 2.55 & 36.99 & 1.28 & 136.49 & 48.28 & 1.87 & 3.96 \\
			\hline
			116.74 & 2.55 & 125.74 & 124.19 & 55.74 & 132.16 & 91.28 & 44.21 \\
			\hline
		\end{tabular}
	\end{small}
	\captionof{figure}{Matriks hasil Perhitungan Histogram untuk seluruh sel\\}
	\label{fig:MatriksHasilPerhitunganHistogram}
\end{table}
Berdasarkan matriks pada gambar \ref{fig:MatriksHasilPerhitunganHistogram}. Elemen matriks yang akan kita gunakan dalam perhitungan normalisasi ini adalah seluruh elemen baris pertama dan baris kedua.
\begin{equation*}
L2_{Norm} = \sqrt{87.28^2 + 129.45^2 + \ldots + 0.99^2 + 46.96^2} = 361.428
\end{equation*}
Kemudian untuk setiap nilai dari histogram dari sel dalam blok tersebut akan dibagi dengan nilai hasil normalisasinya. Di bawah adalah contoh hasil normalisasi histogram dari sel pertama (matriks hasil perhitungan histogram baris pertama kolom 1-4):
\begin{gather*}
\begin{bmatrix}
0.24148 & 0.35815 & 0.2455 & 0.00353 \\
\end{bmatrix}
\end{gather*}
Lakukan proses normalisasi untuk setiap blok dengan menggeser secara horizontal sejauh 1 kali ukuran sel dan secara vertikal sejauh 1 kali ukuran sel sampai blok tersebut sudah berada di bawah kanan dari citra. Kemudian hasil dari proses normalisasi akan disusun menjadi matriks besar dengan jumlah kolom sebesar  \textit{jumlah bin} $\times$ \textit{lebar blok dalam satuan sel} $\times$ \textit{jumlah pergeseran horizontal} dan jumlah baris sebesar \textit{jumlah pergeseran vertikal} $\times$ \textit{tinggi blok dalam satuan sel} , dengan perhitungan tersebut, dalam analisa saat ini didapatkan ukuran matriks sebesar 6 $\times$ 8. Dalam analisa ini, hasil keluaran dari metode \textit{Histogram of Oriented Gradient} ada sebanyak 48 fitur. Di bawah adalah hasil fitur vektor untuk metode \textit{Histogram of Oriented Gradient} setelah melewati proses normalisasi.\\
\begin{table}[H]
	\centering
	\begin{small}
		\begin{tabular}{|p{1cm}|p{1cm}|p{1cm}|p{1cm}|p{1cm}|p{1cm}|p{1cm}|p{1cm}|}
			\hline
			0.24 & 0.36 & 0.25 & 0.00 & 0.25 & 0.00 & 0.25 & 0.35 \\ \hline
			0.58 & 0.00 & 0.09 & 0.01 & 0.39 & 0.01 & 0.00 & 0.13 \\ \hline
			0.61 & 0.00 & 0.10 & 0.01 & 0.41 & 0.01 & 0.00 & 0.14 \\ \hline
			0.50 & 0.01 & 0.11 & 0.00 & 0.40 & 0.14 & 0.01 & 0.01 \\ \hline
			0.48 & 0.01 & 0.10 & 0.00 & 0.38 & 0.14 & 0.01 & 0.01 \\ \hline
			0.33 & 0.01 & 0.35 & 0.35 & 0.16 & 0.37 & 0.26 & 0.12 \\ \hline
		\end{tabular}
	\end{small}
	\captionof{figure}{Matriks hasil Normalisasi\\}
	\label{fig:MatriksHasilNormalisasi}
\end{table}
Setelah mendapatkan matriks \textit{HOG descriptor} di atas. Langkah berikutnya adalah menjadikan matriks tersebut sebagai vektor. Hal ini dilakukan dengan mengambil setiap baris dari matriks dan memasukkannya ke dalam matriks vektor berukuran 1 $\times$ jumlah fitur. Vektor inilah yang akan dijadikan sebagai masukan bagi metode \textit{Machine Learning} yang akan digunakan dalam penelitian ini.\\
\end{enumerate}

\subsection{\textit{Support Vector Machine}}
\noindent Tahapan terakhir dari sistem deteksi dan pengenalan plat nomor kendaraan adalah klasifikasi karakter. Masukan untuk proses ini berupa fitur \textit{HOG Descriptor} untuk setiap citra karakter plat nomor. Tahapan ini bertujuan untuk mengklasifikasikan fitur-fitur dari \textit{HOG descriptor} yang dihasilkan dari perhitungan metode \textit{Histogram of Oriented Gradient} agar dapat dikenali sebagai karakter. \textit{Support Vector Machine} yang akan digunakan dalam penelitian menggunakan \textit{library} dari Weka SVM. Untuk \textit{function} atau \textit{method} yang digunakan pada \textit{library} tersebut dapat dilihat pada tabel \ref{tbl:FunctionWeka}. \textit{Support Vector Machine} termasuk dalam algoritme \textit{supervised learning}. Konsep dasar dari metode ini adalah untuk menemukan sebuah \textit{separating hyperplane} (bidang) yang dapat memisahkan dua kelas sebagai keputusan klasifikasi. Dalam penelitian ini karakter yang akan dikenali adalah huruf A sampai dengan Z dan angka dari 0 sampai dengan 9 sehingga akan terdapat 36 kelas untuk proses klasifikasi.
%\noindent Tabel \ref{tbl:filteredDataset} merupakan rincian jumlah citra pada masing-masing \textit{dataset} setelah dilakukan pemilihan.
%\begin{table}[H]
%	\centering
%	\begin{small}
%		\captionof{table}{Rincian \textit{Dataset} yang telah dipilih\label{tbl:filteredDataset}}
%		\begin{tabular}{|p{4cm}|p{1cm}|p{3cm}|p{3cm}|}
%			\hline
%			\textbf{Nama \textit{Dataset}} &\textbf{Folder} & \textbf{Jumlah Citra} & \textbf{Jumlah Posisi Manusia}\\
%			\hline
%			\textit{Clothing Store} 			& - & 770 & 1631 \\
%			\hline
%			\multirow{3}{*}\textit{Outdoor}	& 31 & 46 & 218 \\\cline{2-4}
%			& 54 & 84 & 371 \\\cline{2-4}
%			& 56 & 113 & 496 \\\hline
%		\end{tabular}
%	\end{small}
%\end{table}



\newpage
	%\setcounter{page}{1}
	%\setlength\LTleft{0pt}            % default: \fill
	%\setlength\LTright{0pt}           % default: \fill	
	%%-----------------------------------------------------------------------------%
\chapter{IMPLEMENTASI DAN PENGUJIAN}
%-----------------------------------------------------------------------------%

%
\vspace{4.5pt}
\noindent Pada bab ini akan menjelaskan mengenai proses implementasi dan pengujian terhadap sistem yang telah dibangun berdasarkan penjelasan pada bab sebelumnya.\\

\section{Lingkungan Implementasi}
\noindent Pada lingkungan implementasi, akan dijelaskan mengenai perangkat yang digunakan dalam proses pembangunan sistem baik dari perangkat keras maupun perangkat lunak yang digunakan.\\

\subsection{Spesifikasi Perangkat Keras}
\noindent Spesifikasi dari perangkat keras yang digunakan dalam pembangunan aplikasi adalah sebagai berikut:
\begin{enumerate}[noitemsep]
\item \textit{Laptop} ASUS A442UQ
\item \textit{Processor} Intel Core i7-7500U CPU @ 2.7GHz
\item \textit{Hard Disk} kapasitas 1TB
\item RAM 16GB\\
\end{enumerate}

\subsection{Lingkungan Perangkat Lunak}
\noindent Spesifikasi dari perangkat lunak yang digunakan dalam pembangunan aplikasi adalah sebagai berikut:
\begin{enumerate}[noitemsep]
\item Sistem Operasi Windows 10 Home 64 bit.
\item Netbeans IDE 8.2
\item Java Development Kit (JDK) 1.8.0{\_}161
\item \textit{Library} OpenCV 3.4.6\\
\end{enumerate}

\section{Implementasi Perangkat Lunak}
\noindent Pada bab ini akan dijelaskan mengenai implementasi aplikasi untuk pengenalan karakter pada citra plat kendaraan. Di bawah ini merupakan daftar \textit{class} dan \textit{method} beserta penjelasan mengenai cara kerja program.\\
\subsection{Daftar \textit{Class} dan \textit{Method} Gradient}
\noindent Berikut adalah tabel berisi \textit{method} pada \textit{class} Gradient. \textit{Class} Gradient digunakan untuk menyimpan nilai \textit{orientation} dan nilai \textit{magnitude} dari suatu piksel citra.
\begin{small}
	\begin{longtable}{| p {0.5cm} | p {3cm} | p {4cm} | p {1.5cm} | p {3cm} |}
		\caption{Daftar \textit{Method Class Gradient} } \\
		\hline
		\textbf{No}  & \textbf{Nama \textit{method}}  & \textbf{Masukan}  & \textbf{Keluaran} & \textbf{Keterangan} \\ \hline
		\endhead	
		1	& Gradient & double orientation, double magnitude & void & Metode \textit{constructor} yang digunakan untuk inisialisasi objek dari kelas Gradient dengan nilai \textit{orientation} dan \textit{magnitude} yang didapatkan dari perhitungan. \\
		\hline
		2	& getOrientation() & & double & Metode untuk mengembalikan nilai \textit{orientation} dari suatu piksel.\\
		\hline
		3	& getMagnitude() & & double & Metode yang digunakan untuk mengembalikan nilai \textit{magnitude} dari suatu piksel.\\
		\hline
		4	& setOrientation() & double orientation & void & Metode untuk mengatur nilai \textit{orientation} dari objek Gradient berdasarkan nilai \textit{orientation} yang dijadikan masukkan.\\
		\hline
		5	& setMagnitude() & double magnitude & void & Metode yang digunakan untuk mengatur nilai \textit{magnitude} dari objek Gradient berdasarkan nilai \textit{magnitude}  yang dijadikan masukkan.\\
		\hline
	\end{longtable}
\end{small}

\subsection{Daftar \textit{Class} dan \textit{Method} GradientCell}
\noindent Berikut adalah tabel berisi \textit{method} pada \textit{class} GradientCell. \textit{Class} GradientCell digunakan untuk menyimpan nilai gradien dari setiap sel.
\begin{small}
	\begin{longtable}{| p {0.5cm} | p {3cm} | p {3cm} | p {2.5cm} | p {3cm} |}
		\caption{Daftar \textit{Method Class GradientCell} } \\
		\hline
		\textbf{No}  & \textbf{Nama \textit{method}}  & \textbf{Masukan}  & \textbf{Keluaran} & \textbf{Keterangan} \\ \hline
		\endhead	
		1	& GradientCell() & int length & void & Metode \textit{constructor} yang digunakan untuk inisialisasi objek dari kelas GradientCell. \\
		\hline
		2	& getGradients() & & List \textless Gradient \textgreater & Metode untuk mengembalikan \textit{List} dari gradien-gradien yang terdapat .\\
		\hline
	\end{longtable}
\end{small}

\subsection{Daftar \textit{Class} dan \textit{Method} HOG}
\noindent Berikut adalah tabel berisi \textit{method} pada \textit{class} HOG. \textit{Class} HOG digunakan untuk proses ekstraksi fitur dari citra karakter.
\begin{small}
	\begin{longtable}{| p {0.5cm} | p {4.5cm} | p {2.5cm} | p {1.5cm} | p {3cm} |}
		\caption{Daftar \textit{Method Class HOG} } \\
		\hline
		\textbf{No}  & \textbf{Nama \textit{method}}  & \textbf{Masukan}  & \textbf{Keluaran} & \textbf{Keterangan} \\ \hline
		\endhead	
		1	& HOG() & Integer[][] image, int cellHeight, int cellWidth, int blockSize, int numBins & void & Metode \textit{constructor} yang digunakan untuk inisialisasi objek dari kelas HOG. \\
		\hline
		2	& extractHOGFeatures() & & double[] & Metode untuk mengekstraksi fitur \textit{HOG descriptor} dari citra.\\
		\hline
		3	& calculateGradientAndCells() & & void & Metode yang digunakan untuk menghitung nilai gradien, \textit{magnitude}, dan orientasi untuk setiap sel.\\
		\hline
		4	& createHistograms() & 	& void & Metode untuk membentuk histogram untuk mencatat persebaran arah dari setiap sel.\\
		\hline
		5	& histogramNormalization() & & void & Melakukan normalisasi \textit{L2-Norm} untuk setiap elemen pada histogram.\\
		\hline
		6	& createDescriptor() & & void & Membentuk \textit{HOG descriptor} dari hasil normalisasi histogram.\\
		\hline
	\end{longtable}
\end{small}

\subsection{Daftar \textit{Class} dan \textit{Method} SVM}
\noindent Berikut adalah tabel berisi \textit{method} pada \textit{class} SVM. \textit{Class} SVM digunakan untuk perhitungan klasifikasi.
\begin{small}
	\begin{longtable}{| p {0.5cm} | p {3.5cm} | p {3cm} | p {2cm} | p {3cm} |}
		\caption{Daftar \textit{Method Class SVM} } \\
		\hline
		\textbf{No}  & \textbf{Nama \textit{method}}  & \textbf{Masukan}  & \textbf{Keluaran} & \textbf{Keterangan} \\
		\hline
		\endfirsthead
		\endhead	
		1	& calculateRBFKernel() & double[][] data, double sigma,
		int classSource, int classTarget	& double &	Menghitung nilai RBF Kernel.\\
		\hline
		2	& createRBFMatrix() & double[][] data, double[] sigma & double[][] & Membentuk matriks RBF dari data fitur.\\
		\hline
		3	& createLinearEquation() & double[][] rbfMatrix, double[] classList	& double[][]	& Membuat persamaan linear dari matriks RBF.\\
		\hline
		4	& getSolutions() & double[][] linearEquationMatix, double[] classList	& Matrix & Mendapatkan solusi dari persamaan linear yaitu nilai alpha dan bias.\\
		\hline
		5	& createRBFTestMatrix() & double[][] data, double sigma, double[] classList	& double & Membentuk matriks RBF untuk data pengujian.\\
		\hline
		6	& classify() & double[][] solutions, double[] rbfTest, double[] classList	& double & Mendapatkan nilai hasil klasifikasi berdasarkan data uji dan nilai alpha dan bias.\\
		\hline
		7	& getDataFromText() & String path	& double[][] & Membaca matriks fitur dari berkas teks.\\
		\hline
		
	\end{longtable}
\end{small}

\subsection{Daftar \textit{Class} dan \textit{Method} ConfusionMatrix}
\noindent Berikut adalah tabel berisi \textit{method} pada \textit{class} ConfusionMatrix. \textit{Class} ConfusionMatrix digunakan untuk perhitungan akurasi dari hasil klasifikasi karakter yang dilakukan dengan metode SVM. \textit{Confusion Matrix} juga biasanya digunakan sebagai alat ukur untuk menghitung kinerja dari algoritme klasifikasi yang digunakan.

\begin{small}
	\begin{longtable}{| p {0.5cm} | p {3cm} | p {3cm} | p {1.5cm} | p {4cm} |}
		\caption{Daftar \textit{Method Class ConfusionMatrix} } \\
		\hline
		\textbf{No}  & \textbf{Nama \textit{method}}  & \textbf{Masukan}  & \textbf{Keluaran} & \textbf{Keterangan} \\ \hline
		\endhead	
		1	& getClassIndex() & String label & int & Metode yang digunakan untuk mengembalikan nilai indeks \textit{array} dari label yang dimasukkan. \\
		\hline
		2	& getConfusionMatrix() & int[][] result & void & Metode untuk melakukan perhitungan \textit{Confusion Matrix} dan juga menghitung akurasi dari hasil klasifikasi, metode ini juga akan menampilkan hasil \textit{Confusion Matrix} sebagai keluaran pada antarmuka aplikasi.\\
		\hline
	\end{longtable}
\end{small}

\subsection{Tampilan Antarmuka Antar Aplikasi}
\noindent Subbab ini akan menjelaskan tampilan antarmuka dari aplikasi pengenalan plat nomor kendaraan. Tampilan awal dari aplikasi ketika dibuka adalah seperti pada Gambar \ref{fig:TampilanAntarmuka}. \\
\\
\begin{adjustbox}{width=1\textwidth}
	\noindent\begin{minipage}{\linewidth}
		\centering\includegraphics[width=14cm]{images/TampilanAntarmuka.png}
		\captionof{figure}{Tampilan antarmuka aplikasi pengenalan plat nomor kendaraan\\}
		\label{fig:TampilanAntarmuka}
	\end{minipage}
\end{adjustbox}\\
\\
\noindent Pada Gambar \ref{fig:TampilanAntarmuka}, terdapat beberapa tombol diantaranya adalah tombol \textit{Load Train Data Path}, \textit{Load Test Data Path}, \textit{Extract HOG Descriptor}, \textit{Train Classifier}, dan \textit{Show Testing Result}. Terdapat juga beberapa tombol \textit{dropdown} yang berfungsi untuk memilih nilai dari parameter metode HOG dan SVM. Untuk metode HOG tombol \textit{dropdown} yang tersedia adalah tombol \textit{dropdown} untuk memilih ukuran sel (\textit{Cell Size}) dan jumlah \textit{bins} yang akan digunakan (\textit{Num of Bins}). Sedangkan untuk metode SVM tombol \textit{dropdown} yang tersedia adalah tombol \textit{dropdown} untuk memilih nilai sigma yang akan digunakan. Tahapan proses pengenalan plat nomor kendaraan di aplikasi terbagi menjadi dua, yaitu proses \textit{training} dan proses \textit{testing}. Tahapan proses \textit{training} pada aplikasi adalah sebagai berikut:
\begin{enumerate}
\item Memasukkan \textit{path} dari kumpulan citra yang akan digunakan sebagai data latih untuk proses pengenalan karakter.
\item Memasukkan ukuran sel dan jumlah \textit{bin} yang digunakan untuk metode HOG dengan memilih menggunakan tombol \textit{dropdown Cell Size} dan tombol \textit{dropdown Num of Bins}.
\item Klik tombol \textit{Extract HOG Descriptor} untuk menjalankan proses ekstraksi fitur.
\item Klik tombol \textit{Train Classifier} untuk menjalankan proses \textit{training}.
\end{enumerate}
\begin{adjustbox}{width=1\textwidth}
	\noindent\begin{minipage}{\linewidth}
		\centering\framebox{\includegraphics[width=14cm]{images/TampilanAntarmukaTraining.png}}
		\captionof{figure}{Tampilan antarmuka hasil \textit{training}\\}
		\label{fig:TampilanAntarmukaTraining}
	\end{minipage}
\end{adjustbox}\\
\\
\noindent Pada Gambar \ref{fig:TampilanAntarmukaTraining} aplikasi akan menampilkan \textit{path} dari folder citra yang akan digunakan sebagai data latih, ukuran sel dan jumlah \textit{bins} yang digunakan untuk metode HOG, pesan bahwa proses ekstraksi fitur sudah berjalan dan lokasi penyimpanan fitur, dan pesan bahwa proses \textit{training} sudah selesai.
\noindent Sedangkan untuk tahapan proses \textit{testing} pada aplikasi adalah sebagai berikut:
\begin{enumerate}
\item Lakukan proses \textit{training} terlebih dahulu.
\item Setelah proses \textit{training} selesai dilakukan, kemudian masukkan \textit{path} dari folder citra yang akan digunakan sebagai data uji.
\item Klik tombol \textit{Show Testing Result} untuk menjalankan proses testing otomatis.
\end{enumerate}
\begin{adjustbox}{width=1\textwidth}
	\noindent\begin{minipage}{\linewidth}
		\centering\framebox{\includegraphics[width=14cm]{images/TampilanAntarmukaTesting1.png}}
		\captionof{figure}{Keluaran \textit{path} data untuk \textit{testing} dan hasil pengenalan plat}
		\label{fig:TampilanAntarmukaTesting1}
	\end{minipage}
\end{adjustbox}

\begin{adjustbox}{width=1\textwidth}
	\noindent\begin{minipage}{\linewidth}
		\centering\framebox{\includegraphics[width=14cm]{images/TampilanAntarmukaTesting2.png}}
		\captionof{figure}{Keluaran tingkat akurasi pengenalan plat dan \textit{Confusion Matrix}}
		\label{fig:TampilanAntarmukaTesting2}
	\end{minipage}
\end{adjustbox}

\begin{adjustbox}{width=1\textwidth}
	\noindent\begin{minipage}{\linewidth}
		\centering\framebox{\includegraphics[width=14cm]{images/TampilanAntarmukaTesting3.png}}
		\captionof{figure}{Keluaran hasil tingkat akurasi pengenalan karakter\\}
		\label{fig:TampilanAntarmukaTesting3}
	\end{minipage}
\end{adjustbox}\\
\\
\noindent Pada Gambar \ref{fig:TampilanAntarmukaTesting1} aplikasi akan menampilkan \textit{path} dari folder citra yang akan digunakan sebagai data uji, teks asli dari plat nomor pada citra beserta hasil prediksi dari SVM, jumlah plat yang teridentifikasi dengan benar, jumlah total plat yang diproses, akurasi plat yang terdeteksi dan teridentifikasi dengan benar, hasil dari \textit{confusion matrix} untuk klasifikasi karakter, jumlah karakter yang terklasifikasi dengan benar, jumlah keseluruhan karakter, dan akurasi dari klasifikasi karakter.\\

\subsection{Implementasi Validasi Plat Kendaraan}
\noindent Bagian ini merupakan penjelasan implementasi tahap validasi plat kendaraan seperti yang disebutkan pada subbab 3.4.2.4, citra masukan berupa citra RGB mobil akan diproses untuk mendapatkan citra plat nomor dari citra RGB mobil. Berikut ini merupakan langkah pemrosesan citra dimulai dari masukan citra RGB mobil hingga menjadi citra plat.
\begin{enumerate}
	\item Baca citra RGB mobil dengan menggunakan \textit{Imgcodecs.imread()}. Parameter \textit{method} tersebut adalah lokasi citra yang akan diproses.
	\item Ubah citra RGB menjadi citra \textit{grayscale} dengan menggunakan \textit{Imgproc.cvtColor(mat1, mat2, Imgproc.COLORRGB2GRAY)}. Dengan parameter \textit{mat1} merupakan citra RGB dan \textit{mat2} adalah penampung citra hasil \textit{grayscale} dan \textit{COLORRBG2GRAY} adalah kode konversi \textit{color space} yang digunakan, kode yang digunakan adalah untuk mengkonversi citra RGB yang merupakan citra dengan 3 \textit{channel} warna menjadi citra dengan 1 \textit{channel} warna.
	\item Lakukan proses deteksi tepi \textit{Canny} dengan menggunakan \textit{Imgproc.Canny(gray, edge)}. Dengan parameter \textit{gray} adalah citra hasil \textit{grayscale} dan parameter \textit{edge} adalah penampung citra hasil deteksi tepi \textit{Canny}.
	\item Lakukan proses \textit{Hough Transform} sebanyak dua kali, yang pertama untuk mencari kandidat garis vertikal dengan \textit{range} $\theta$ -90 sampai dengan -85 derajat dan yang kedua adalah untuk mencari garis horizontal dengan \textit{range} $\theta$ -10 sampai dengan 10 derajat. Masing-masing dari kandidat garis vertikal dan horizontal disimpan dalam variabel dengan jenis \textit{ArrayList$<$Line$>$}.
	\item Untuk setiap elemen dalam \textit{ArrayList} yang berisi kandidat garis vertikal. Lakukan perbandingan antar elemen dengan membandingkan garis mana yang lebih kiri dan mana yang lebih kanan, kemudian dihitung lebar dari area yang dibatasi oleh kedua garis tersebut, apabila ukurannya berada dalam kisaran 255 - 390 piksel. Maka koordinat x dari titik awal kedua garis tersebut akan disimpan ke dalam matriks 2 dimensi yang berfungsi untuk menampung batas kiri, kanan, atas, dan bawah dari kandidat area plat.
	\item Untuk setiap elemen dalam \textit{ArrayList} yang berisi kandidat garis horizontal. Lakukan perbandingan antar elemen dengan membandingkan garis mana yang lebih atas dan mana yang lebih bawah, kemudian pasangkan dengan garis batas kiri dan kanan pada \textit{array} penampung kandidat batas kiri dan kanan plat nomor, kemudian hitung tinggi dari area yang dibatasi kedua garis horizontal, apabila berada di kisaran 85 - 130 piksel, maka koordinat y dari titik awal kedua garis tersebut akan ditambahkan pada elemen matriks yang berisi kandidat batas kiri dan kanan tadi, sehingga matriks 2 dimensi akan berisi koordinat titik yang merupakan area plat.
	\item Jika kandidat masih lebih dari satu kandidat, maka koordinat yang dipakai adalah koordinat yang rasio tinggi terhadap lebarnya paling mendekati 0.33.
	\item Kemudian lakukan pemotongan citra RGB dengan menggunakan koordinat yang didapat, dan hasil dari proses ini adalah citra plat.\\
\end{enumerate}

\section{Pengujian}
\noindent Pada bagian ini, akan dilakukan berbagai skenario pengujian dengan beragam parameter dari metode HOG. Tujuan dari penelitian ini adalah untuk menerapkan metode HOG pada proses ekstraksi fitur karakter pada sistem pengenalan karakter, oleh karena itu perlu diketahui berapa ukuran sel, ukuran blok, dan jumlah \textit{bin} yang akan menghasilkan fitur yang paling baik untuk akurasi pengenalan karakter. Pengujian ini akan dilakukan dengan data latih sebanyak 117 citra karakter hasil segmentasi dari plat nomor untuk 36 kelas karakter yang terdiri dari 10 kelas angka dan 26 kelas huruf, dimana setiap kelas karakter memiliki jumlah data latih sebanyak 3-5 citra.

\noindent Fitur dari metode \textit{HOG} akan digunakan pada proses klasifikasi karakter menggunakan \textit{library} SVM dari Weka dan akan diukur akurasinya menggunakan \textit{Confusion Matrix}. Adapun nilai dari setiap parameter yang akan digunakan untuk kombinasi, yaitu:
\begin{enumerate}
	\item Ukuran sel yang akan digunakan (dalam satuan piksel) : 2, 4, 8, 16
	\item Jumlah \textit{bin} yang akan digunakan : 4, 6, 9, 18
	\item Nilai \textit{sigma} pada metode \textit{SVM} yang akan digunakan: 0.01, 0.1, 1
	\item Campuran : Kombinasi parameter yang menghasilkan akurasi paling optimal untuk setiap karakter.\\
\end{enumerate}
%\noindent Adapun untuk nilai sigma pada metode \textit{SVM} yang digunakan dalam kombinasi adalah 0.01, 0.1, dan 1.

%\noindent Setelah setiap kombinasi ukuran sel, jumlah \textit{bin}, dan nilai sigma diuji, akan dipilih kombinasi parameter yang menghasilkan akurasi paling optimal untuk setiap karakter. Kemudian beberapa kombinasi parameter tersebut akan digabungkan dan diuji kembali, tujuannya untuk menilai apakah gabungan dari beberapa kombinasi parameter tersebut dapat meningkatkan akurasi.\\

\subsection{Pengujian Kombinasi Parameter}
\noindent Pada skenario pengujian ini, pengujian akan dilakukan dengan menggunakan ukuran sel berukuran 2 $\times$ 2 piksel, 4 $\times$ 4 piksel, 8 $\times$ 8 piksel, dan 16 $\times$ 16 piksel. Setiap ukuran sel akan menggunakan ukuran blok 2 $\times$ 2 sel. Jumlah bin yang akan digunakan adalah 4, 6, 9, dan 18. Sedangkan untuk nilai sigma pada metode \textit{SVM} yang akan digunakan adalah 0.01, 0.1, dan 1. Berikut adalah hasil pengujian untuk setiap kombinasi parameter tersebut:

%\begin{longtable}[c]{|c|c|c|c|c|}
%	\caption{Hasil Pengujian dengan ukuran sel 2 $\times$ 2 piksel}
%	\label{tab:HasilPengujianSel2}\\
%	\hline
%	\multicolumn{3}{|c|}{Parameter} &                                &                                \\ \cline{1-3}
%	CellSize   & NumBins   & Sigma  & \multirow{-2}{*}{CRR}          & \multirow{-2}{*}{OVR}          \\ \hline
%	\endhead
%	%
%	2          & 4         & 0.01   & {\color[HTML]{FE0000} 58.52\%} & {\color[HTML]{FE0000} 10.34\%} \\ \hline
%	2          & 4         & 0.1    & 26.13\%                        & 0\%                            \\ \hline
%	2          & 4         & 1      & 18.18\%                        & 0\%                            \\ \hline
%	2          & 6         & 0.01   & 50\%                           & 6.89\%                         \\ \hline
%	2          & 6         & 0.1    & 25.56\%                        & 0\%                            \\ \hline
%	2          & 6         & 1      & 18.18\%                        & 0\%                            \\ \hline
%	2          & 9         & 0.01   & 48.29\%                        & 3.44\%                         \\ \hline
%	2          & 9         & 0.1    & 25\%                           & 0\%                            \\ \hline
%	2          & 9         & 1      & 18.18\%                        & 0\%                            \\ \hline
%	2          & 18        & 0.01   & 44.88\%                        & 3.44\%                         \\ \hline
%	2          & 18        & 0.1    & 25\%                           & 0\%                            \\ \hline
%	2          & 18        & 1      & 18.18\%                        & 0\%                            \\ \hline
%\end{longtable}
%\begin{longtable}[c]{|r|r|r|r|r|}
%	\caption{Pengujian dengan ukuran sel 2 $\times$ 2 piksel}
%	\label{tab:HasilPengujianSel2}\\
%	\hline
%	\multicolumn{3}{|c|}{Parameter} &                                &                                \\ \cline{1-3}
%	CellSize   & NumBins   & Sigma  & \multicolumn{1}{c|}{\multirow{-2}{*}{CRR}} & \multicolumn{1}{c|}{\multirow{-2}{*}{OVR}} \\ \hline
%	\endhead
%	%
%	2          & 4         & 0.01   & {\color[HTML]{FE0000} 61.53\%} & {\color[HTML]{FE0000} 14.28\%} \\ \hline
%	2          & 4         & 0.1    & 28.67\%                        & 0\%                            \\ \hline
%	2          & 4         & 1      & 20.27\%                        & 0\%                            \\ \hline
%	2          & 6         & 0.01   & 53.84\%                        & 9.52\%                         \\ \hline
%	2          & 6         & 0.1    & 27.97\%                        & 0\%                            \\ \hline
%	2          & 6         & 1      & 20.27\%                        & 0\%                            \\ \hline
%	2          & 9         & 0.01   & 52.44\%                        & 4.76\%                         \\ \hline
%	2          & 9         & 0.1    & 27.27\%                        & 0\%                            \\ \hline
%	2          & 9         & 1      & 20.27\%                        & 0\%                            \\ \hline
%	2          & 18        & 0.01   & 48.25\%                        & 4.76\%                         \\ \hline
%	2          & 18        & 0.1    & 27.97\%                        & 0\%                            \\ \hline
%	2          & 18        & 1      & 20.27\%                        & 0\%                            \\ \hline
%\end{longtable}
\begin{adjustbox}{width=1\textwidth}
	\noindent\begin{minipage}{\linewidth}
		\centering\framebox{\includegraphics[width=14cm]{images/HasilPengujianUkuranSel2.png}}
		\captionof{figure}{Hasil Pengujian Kombinasi Parameter dengan Ukuran Sel 2\\}
		\label{fig:HasilPengujianUkuranSel2}
	\end{minipage}
\end{adjustbox}

\noindent Berdasarkan Gambar \ref{fig:HasilPengujianUkuranSel2}, dapat disimpulkan bahwa tingkat akurasi pengenalan karakter (CRR) maksimal yang didapatkan apabila menggunakan ukuran sel 2 $\times$ 2 piksel adalah 61.53\%. Kombinasi parameter yang digunakan untuk mencapai hasil tersebut adalah ukuran sel 2 $\times$ 2 piksel, jumlah \textit{bin} sebanyak 4 sehingga besar setiap \textit{bin} adalah 45 derajat, kemudian nilai \textit{sigma} yang digunakan untuk metode \textit{SVM} adalah 0.01. Dengan citra karakter masukan berukuran 32 $\times$ 32 piksel. Maka panjang vektor fitur dari \textit{HOG descriptor} yang dihasilkan adalah 3600 fitur.

\noindent Grafik batang berwarna biru menunjukkan tingkat akurasi pengenalan karakter (CRR). CRR sendiri merupakan akronim dari \textit{Character Recognition Rate} yang memiliki rumus jumlah karakter yang dikenali dibagi dengan keseluruhan karakter yang terdeteksi. Berdasarkan hasil CRR tertinggi pada Gambar \ref{fig:HasilPengujianUkuranSel2}, dari 143 karakter yang terdeteksi, sebanyak 88 di antaranya dapat diklasifikasikan dengan baik. Grafik batang berwarna kuning menunjukkan performa keseluruhan aplikasi (OVR). OVR merupakan akronim dari \textit{Overall Performance} yang memiliki rumus jumlah plat nomor yang terdeteksi dan dikenali dengan benar dibagi dengan keseluruhan jumlah plat nomor yang ada. Berdasarkan hasil OVR tertinggi dari Gambar \ref{fig:HasilPengujianUkuranSel2}, dari 21 plat nomor yang terdeteksi, hanya 3 plat nomor yang dapat dikenali dengan baik. 

\noindent Tabel \ref{tab:hasilklasifikasisel2} merupakan tabel yang menunjukkan hasil klasifikasi karakter dengan parameter HOG (\textit{CellSize} dan \textit{NumBins}) masing-masing 2 dan 4, dan nilai \textit{sigma} untuk metode \textit{SVM} 0.01.

\begin{longtable}[c]{|r|r|r|r|r|}
	\caption{Klasifikasi karakter dengan parameter CellSize = 2, NumBins = 4, dan Sigma = 0.01}
	\label{tab:hasilklasifikasisel2}\\
	\hline
	\textbf{No} & \textbf{Karakter} & \textbf{Prediksi Benar} & \textbf{Prediksi Salah} & \textbf{Akurasi} \\ \hline
	\endfirsthead
	\hline
	\textbf{No} & \textbf{Karakter} & \textbf{Prediksi Benar} & \textbf{Prediksi Salah} & \textbf{Akurasi} \\ \hline
	\endhead
	1           & 0                 & 2                       & 0                       &100.00\%            \\ \hline
	2           & 1                 & 15                       & 7                       &68.18\%            \\ \hline
	3           & 2                 & 6                       & 3                       &66.67\%            \\ \hline
	4           & 3                 & 2                       & 6                       &25.00\%            \\ \hline
	5           & 4                 & 3                       & 5                       &37.50\%            \\ \hline
	6           & 5                 & 2                       & 2                       &50.00\%            \\ \hline
	7           & 6                 & 1                       & 3                       &25.00\%            \\ \hline
	8           & 7                 & 6                       & 7                       &46.15\%            \\ \hline
	9           & 8                 & 1                       & 2                       &33.33\%            \\ \hline
	10           & 9                 & 2                       & 4                       &33.33\%            \\ \hline
	11           & A                 & 0                       & 0                       & -            \\ \hline
	12           & B                 & 1                       & 3                       &25.00\%            \\ \hline
	13           & C                 & 1                       & 0                       &100.00\%            \\ \hline
	14           & D                 & 20                       & 0                       &100.00\%            \\ \hline
	15           & E                 & 1                       & 0                       &100.00\%            \\ \hline
	16           & F                 & 0                       & 0                       & -            \\ \hline
	17           & G                 & 1                       & 0                       &100.00\%            \\ \hline
	18           & H                 & 1                       & 0                       &100.00\%            \\ \hline
	19           & I                 & 1                       & 2                       &33.33\%            \\ \hline
	20           & J                 & 2                       & 1                       &66.67\%            \\ \hline
	21           & K                 & 2                       & 0                       &100.00\%            \\ \hline
	22           & L                 & 4                       & 0                       &100.00\%            \\ \hline
	23           & M                 & 1                       & 0                       &100.00\%            \\ \hline
	24           & N                 & 0                       & 0                       & -            \\ \hline
	25           & O                 & 1                       & 0                       &100.00\%            \\ \hline
	26           & P                 & 1                       & 3                       &25.00\%            \\ \hline
	27           & Q                 & 2                       & 0                       &100.00\%            \\ \hline
	28           & R                 & 1                       & 1                       &50.00\%            \\ \hline
	29           & S                 & 1                       & 1                       &50.00\%            \\ \hline
	30           & T                 & 1                       & 0                       &100.00\%            \\ \hline
	31           & U                 & 1                       & 0                       &100.00\%            \\ \hline
	32           & V                 & 1                       & 0                       &100.00\%            \\ \hline
	33           & W                 & 2                       & 5                       &28.57\%            \\ \hline
	34           & X                 & 0                       & 0                       & -            \\ \hline
	35           & Y                 & 1                       & 0                       &100.00\%            \\ \hline
	36           & Z                 & 1                       & 0                       &100.00\%            \\ \hline
\end{longtable}

\noindent Dengan tingkat akurasi pengenalan karakter (\textit{Character Recognition Rate}) sebesar 61.53\%, dapat dilihat pada tabel \ref{tab:hasilklasifikasisel2} bahwa dari 36  karakter yang ada, yang dapat diprediksi dengan benar 100\% adalah sebanyak 16 karakter, terdapat juga beberapa karakter yang memiliki akurasi pengenalan yang rendah, diantaranya adalah angka 3, angka 4, angka 6, angka 7, angka 8, angka 9, huruf B, huruf I, huruf P, dan huruf W. Dari hasil akurasi klasifikasi di atas, dapat disimpulkan bahwa penggunaan ukuran sel 2 $\times$ 2 piksel kurang tepat untuk digunakan dalam proses ekstraksi fitur HOG dalam sistem pengenalan karakter ini.

%\subsection{Pengujian dengan Ukuran Sel 4}
%\noindent Pada bagian ini, pengujian akan dilakukan dengan menggunakan ukuran sel berukuran 4 $\times$ 4 piksel dan ukuran blok 2 $\times$ 2 sel (8 $\times$ 8 piksel). Jumlah bin yang akan digunakan adalah 4, 6, 9, dan 18. Sedangkan untuk nilai sigma pada metode \textit{SVM} yang akan digunakan adalah 0.01, 0.1, dan 1. Berikut adalah hasil pengujian untuk setiap kombinasi parameter tersebut:
%\begin{longtable}[c]{|c|c|c|}
%	\caption{Hasil Pengujian dengan ukuran sel 4 $\times$ 4 piksel}
%	\label{tab:HasilPengujianSel4}\\
%	\hline
%	\begin{tabular}[c]{@{}c@{}}Parameter\\ (CellSize, NumBins, Sigma)\end{tabular} & CRR     & OVR     \\ \hline
%	\endhead
%	%
%	(4, 4, 0.01)                                                                   & 86.36\% & 27.58\% \\ \hline
%	(4, 4, 0.1)                                                                    & 47.72\% & 0\%     \\ \hline
%	(4, 4, 1)                                                                      & 26.13\% & 0\%     \\ \hline
%	(4, 6, 0.01)                                                                   & 85.22\% & 31.03\%  \\ \hline
%	(4, 6, 0.1)                                                                    & 40.90\% & 0\%     \\ \hline
%	(4, 6, 1)                                                                      & 25\%    & 0\%     \\ \hline
%	(4, 9, 0.01)                                                                   & {\color[HTML]{FE0000} 89.77\%} & {\color[HTML]{FE0000} 37.93\%}  \\ \hline
%	(4, 9, 0.1)                                                                    & 40.90\% & 0\%     \\ \hline
%	(4, 9, 1)                                                                      & 23.86\% & 0\%     \\ \hline
%	(4, 18, 0.01)                                                                  & 84.09\% & 34.48\%  \\ \hline
%	(4, 18, 0.1)                                                                   & 40.90\% & 0\%     \\ \hline
%	(4, 18, 1)                                                                     & 24.43\% & 0\%     \\ \hline
%\end{longtable}
%\begin{longtable}[c]{|r|r|r|r|r|}
%	\caption{Pengujian dengan ukuran sel 4 $\times$ 4 piksel}
%	\label{tab:HasilPengujianSel4}\\
%	\hline
%	\multicolumn{3}{|c|}{Parameter} &                                &                                \\ \cline{1-3}
%	CellSize   & NumBins   & Sigma  & \multicolumn{1}{c|}{\multirow{-2}{*}{CRR}} & \multicolumn{1}{c|}{\multirow{-2}{*}{OVR}} \\ \hline
%	\endhead
%	%
%	4          & 4         & 0.01   & 88.11\%                        & 38.09\% \\ \hline
%	4          & 4         & 0.1    & 50.34\%                        & 0\%                            \\ \hline
%	4          & 4         & 1      & 27.97\%                        & 0\%                            \\ \hline
%	4          & 6         & 0.01   & 86.71\%                        & 42.85\%                         \\ \hline
%	4          & 6         & 0.1    & 44.05\%                        & 0\%                            \\ \hline
%	4          & 6         & 1      & 26.57\%                        & 0\%                            \\ \hline
%	4          & 9         & 0.01   & {\color[HTML]{FE0000} 90.20\%} & {\color[HTML]{FE0000} 52.38\%} \\ \hline
%	4          & 9         & 0.1    & 44.05\%                        & 0\%                            \\ \hline
%	4          & 9         & 1      & 25.87\%                        & 0\%                            \\ \hline
%	4          & 18        & 0.01   & 87.41\%                        & 47.61\%                         \\ \hline
%	4          & 18        & 0.1    & 44.05\%                        & 0\%                            \\ \hline
%	4          & 18        & 1      & 25.87\%                        & 0\%                            \\ \hline
%\end{longtable}
\begin{adjustbox}{width=1\textwidth}
	\noindent\begin{minipage}{\linewidth}
		\centering\framebox{\includegraphics[width=14cm]{images/HasilPengujianUkuranSel4.png}}
		\captionof{figure}{Hasil Pengujian Kombinasi Parameter dengan Ukuran Sel 4\\}
		\label{fig:HasilPengujianUkuranSel4}
	\end{minipage}
\end{adjustbox}

\noindent Berdasarkan Gambar \ref{fig:HasilPengujianUkuranSel4}, dapat disimpulkan bahwa tingkat akurasi pengenalan karakter (CRR) maksimal yang didapatkan apabila menggunakan ukuran sel 4 $\times$ 4 piksel adalah 90.20\%. Kombinasi parameter yang digunakan untuk mencapai hasil tersebut adalah ukuran sel 4 $\times$ 4 piksel, jumlah \textit{bin} sebanyak 9 sehingga besar setiap \textit{bin} adalah 20 derajat, kemudian nilai \textit{sigma} yang digunakan untuk metode \textit{SVM} adalah 0.01. Dengan citra karakter masukan berukuran 32 $\times$ 32 piksel. Maka panjang vektor fitur dari \textit{HOG descriptor} yang dihasilkan adalah 1764 fitur. Jika dibandingkan dengan pengujian sebelumnya, jumlah fitur yang lebih sedikit justru mampu mendapatkan akurasi pengenalan karakter yang lebih baik.

\noindent Berdasarkan hasil CRR tertinggi pada Gambar \ref{fig:HasilPengujianUkuranSel4}, dari 143 karakter yang terdeteksi, sebanyak 129 di antaranya dapat diklasifikasikan dengan baik. Sedangkan dari hasil OVR tertinggi pada Gambar \ref{fig:HasilPengujianUkuranSel4}, dari 21 plat nomor yang terdeteksi, 11 plat nomor dapat dikenali dengan baik, hal ini merupakan peningkatan apabila dibandingkan dengan hasil pengujian sebelumnya, namun masih cukup banyak plat yang tidak dapat dikenali dengan baik. 

\noindent Tabel \ref{tab:hasilklasifikasisel4} merupakan tabel yang menunjukkan hasil klasifikasi karakter dengan parameter HOG (\textit{CellSize} dan \textit{NumBins}) masing-masing 4 dan 9, dan nilai \textit{sigma} untuk metode \textit{SVM} 0.01.

\begin{longtable}[c]{|r|r|r|r|r|}
	\caption{Klasifikasi karakter dengan parameter CellSize = 4, NumBins = 9, dan Sigma = 0.01}
	\label{tab:hasilklasifikasisel4}\\
	\hline
	\textbf{No} & \textbf{Karakter} & \textbf{Prediksi Benar} & \textbf{Prediksi Salah} & \textbf{Akurasi} \\ \hline
	\endhead
	1           & 0                 & 0                       & 2                       &0.00\%            \\ \hline
	2           & 1                 & 18                       & 4                       &81.82\%            \\ \hline
	3           & 2                 & 7                       & 2                       &77.78\%            \\ \hline
	4           & 3                 & 8                       & 0                       &100.00\%            \\ \hline
	5           & 4                 & 8                       & 0                       &100.00\%            \\ \hline
	6           & 5                 & 4                       & 0                       &100.00\%            \\ \hline
	7           & 6                 & 4                       & 0                       &100.00\%            \\ \hline
	8           & 7                 & 13                       & 0                       &100.00\%            \\ \hline
	9           & 8                 & 3                       & 0                       &100.00\%            \\ \hline
	10           & 9                 & 6                       & 0                       &100.00\%            \\ \hline
	11           & A                 & 0                       & 0                       & -            \\ \hline
	12           & B                 & 4                       & 0                       &100.00\%            \\ \hline
	13           & C                 & 1                       & 0                       &100.00\%            \\ \hline
	14           & D                 & 19                       & 1                       &95.00\%            \\ \hline
	15           & E                 & 1                       & 0                       &100.00\%            \\ \hline
	16           & F                 & 0                       & 0                       & -            \\ \hline
	17           & G                 & 1                       & 0                       &100.00\%            \\ \hline
	18           & H                 & 1                       & 0                       &100.00\%            \\ \hline
	19           & I                 & 3                       & 0                       &100.00\%            \\ \hline
	20           & J                 & 2                       & 1                       &66.67\%            \\ \hline
	21           & K                 & 2                       & 0                       &100.00\%            \\ \hline
	22           & L                 & 4                       & 0                       &100.00\%            \\ \hline
	23           & M                 & 1                       & 0                       &100.00\%            \\ \hline
	24           & N                 & 0                       & 0                       & -            \\ \hline
	25           & O                 & 1                       & 0                       &100.00\%            \\ \hline
	26           & P                 & 4                       & 0                       &100.00\%            \\ \hline
	27           & Q                 & 0                       & 2                       &0.00\%            \\ \hline
	28           & R                 & 2                       & 0                       &100.00\%            \\ \hline
	29           & S                 & 2                       & 0                       &100.00\%            \\ \hline
	30           & T                 & 1                       & 0                       &100.00\%            \\ \hline
	31           & U                 & 0                       & 1                       &0.00\%            \\ \hline
	32           & V                 & 1                       & 0                       &100.00\%            \\ \hline
	33           & W                 & 6                       & 1                       &85.71\%            \\ \hline
	34           & X                 & 0                       & 0                       & -            \\ \hline
	35           & Y                 & 1                       & 0                       &100.00\%            \\ \hline
	36           & Z                 & 1                       & 0                       &100.00\%            \\ \hline
\end{longtable}

\noindent Dengan tingkat akurasi pengenalan karakter (\textit{Character Recognition Rate}) sebesar 90.20\%, dapat dilihat pada tabel \ref{tab:hasilklasifikasisel4} bahwa dari 36  karakter yang ada, yang dapat diprediksi dengan benar 100\% adalah sebanyak 24 karakter, terdapat juga beberapa karakter yang memiliki akurasi pengenalan yang rendah, diantaranya adalah angka 0, huruf Q, dan huruf U. Dari hasil akurasi klasifikasi di atas, dapat  disimpulkan bahwa penggunaan ukuran sel 4 $\times$ 4 piksel sudah dapat meningkatkan akurasi pengenalan karakter namun masih belum optimal.

%\subsection{Pengujian dengan Ukuran Sel 8}
%\noindent Pada bagian ini, pengujian akan dilakukan dengan menggunakan ukuran sel berukuran 8 $\times$ 8 piksel dan ukuran blok 2 $\times$ 2 sel (16 $\times$ 16 piksel). Jumlah bin yang akan digunakan adalah 4, 6, 9, dan 18. Sedangkan untuk nilai sigma pada metode \textit{SVM} yang akan digunakan adalah 0.01, 0.1, dan 1. Berikut adalah hasil pengujian untuk setiap kombinasi parameter tersebut:
%\begin{longtable}[c]{|c|c|c|}
%	\caption{Hasil Pengujian dengan ukuran sel 8 $\times$ 8 piksel}
%	\label{tab:HasilPengujianSel8}\\
%	\hline
%	\begin{tabular}[c]{@{}c@{}}Parameter\\ (CellSize, NumBins, Sigma)\end{tabular} & CRR     & OVR     \\ \hline
%	\endhead
%	%
%	(8, 4, 0.01)                                                                   & 15.90\% & 0\% \\ \hline
%	(8, 4, 0.1)                                                                    & 93.18\% & 51.72\%     \\ \hline
%	(8, 4, 1)                                                                      & 46.02\% & 0\%     \\ \hline
%	(8, 6, 0.01)                                                                   & 21.02\% & 0\%  \\ \hline
%	(8, 6, 0.1)                                                                    & 93.18\% & 51.72\%     \\ \hline
%	(8, 6, 1)                                                                      & 42.61\% & 0\%     \\ \hline
%	(8, 9, 0.01)                                                                   & 28.40\% & 0\%  \\ \hline
%	(8, 9, 0.1)                                                                    & 93.75\% & 51.72\%     \\ \hline
%	(8, 9, 1)                                                                      & 39.20\% & 0\%     \\ \hline
%	(8, 18, 0.01)                                                                  & 21.59\% & 0\%  \\ \hline
%	(8, 18, 0.1)                                                                   & {\color[HTML]{FE0000} 94.88\%} & {\color[HTML]{FE0000} 58.62\%}     \\ \hline
%	(8, 18, 1)                                                                     & 40.34\% & 0\%     \\ \hline
%\end{longtable}
%\begin{longtable}[c]{|r|r|r|r|r|}
%	\caption{Pengujian dengan ukuran sel 8 $\times$ 8 piksel}
%	\label{tab:HasilPengujianSel8}\\
%	\hline
%	\multicolumn{3}{|c|}{Parameter} &                                &                                \\ \cline{1-3}
%	CellSize   & NumBins   & Sigma  & \multicolumn{1}{c|}{\multirow{-2}{*}{CRR}} & \multicolumn{1}{c|}{\multirow{-2}{*}{OVR}} \\ \hline
%	\endhead
%	%
%	8          & 4         & 0.01   & 17.48\%                        & 0\% \\ \hline
%	8          & 4         & 0.1    & 94.40\%                        & 71.42\%                            \\ \hline
%	8          & 4         & 1      & 48.95\%                        & 0\%                            \\ \hline
%	8          & 6         & 0.01   & 23.07\%                        & 0\%                         \\ \hline
%	8          & 6         & 0.1    & 94.40\%                        & 71.42\%                            \\ \hline
%	8          & 6         & 1      & 46.15\%                        & 0\%                            \\ \hline
%	8          & 9         & 0.01   & 29.37\%                        & 0\% \\ \hline
%	8          & 9         & 0.1    & 94.40\%                        & 71.42\%                            \\ \hline
%	8          & 9         & 1      & 41.95\%                        & 0\%                            \\ \hline
%	8          & 18        & 0.01   & 23.77\%                        & 0\%                         \\ \hline
%	8          & 18        & 0.1    & {\color[HTML]{FE0000} 95.80\%} & {\color[HTML]{FE0000} 80.95\%}                            \\ \hline
%	8          & 18        & 1      & 43.35\%                        & 0\%                            \\ \hline
%\end{longtable}
\begin{adjustbox}{width=1\textwidth}
	\noindent\begin{minipage}{\linewidth}
		\centering\framebox{\includegraphics[width=14cm]{images/HasilPengujianUkuranSel8.png}}
		\captionof{figure}{Hasil Pengujian Kombinasi Parameter dengan Ukuran Sel 8\\}
		\label{fig:HasilPengujianUkuranSel8}
	\end{minipage}
\end{adjustbox}

\noindent Berdasarkan Gambar \ref{fig:HasilPengujianUkuranSel8}, dapat disimpulkan bahwa tingkat akurasi pengenalan karakter maksimal yang didapatkan apabila menggunakan ukuran sel 8 $\times$ 8 piksel adalah 95.80\%. Kombinasi parameter yang digunakan untuk mencapai hasil tersebut adalah ukuran sel 8 $\times$ 8 piksel, jumlah \textit{bin} sebanyak 18 sehingga besar setiap \textit{bin} adalah 10 derajat, kemudian nilai \textit{sigma} yang digunakan untuk metode \textit{SVM} adalah 0.1. Dengan citra karakter masukan berukuran 32 $\times$ 32 piksel. Maka panjang vektor fitur dari \textit{HOG descriptor} yang dihasilkan adalah 648 fitur. Sama seperti pengujian sebelumnya, jika dibandingkan dengan pengujian sebelumnya, jumlah fitur yang lebih sedikit justru mampu mendapatkan akurasi pengenalan karakter yang lebih baik.

\noindent Dari hasil CRR tertinggi pada Gambar \ref{fig:HasilPengujianUkuranSel8}, dari 143 karakter yang terdeteksi, sebanyak 137 di antaranya dapat diklasifikasikan dengan baik. Sedangkan dari hasil OVR tertinggi pada Gambar \ref{fig:HasilPengujianUkuranSel8} menunjukkan, dari 21 plat nomor yang terdeteksi, 17 plat nomor dapat dikenali dengan baik, hal ini merupakan peningkatan apabila dibandingkan dengan hasil pengujian sebelumnya.

\noindent Tabel \ref{tab:hasilklasifikasisel8} merupakan tabel yang menunjukkan hasil klasifikasi karakter dengan parameter HOG (\textit{CellSize} dan \textit{NumBins}) masing-masing 8 dan 18, dan nilai \textit{sigma} untuk metode \textit{SVM} 0.1.

\begin{longtable}[c]{|r|r|r|r|r|}
	\caption{Klasifikasi karakter dengan parameter CellSize = 8, NumBins = 18, dan Sigma = 0.1}
	\label{tab:hasilklasifikasisel8}\\
	\hline
	\textbf{No} & \textbf{Karakter} & \textbf{Prediksi Benar} & \textbf{Prediksi Salah} & \textbf{Akurasi} \\ \hline
	\endhead
	1           & 0                 & 2                       & 0                       &100.00\%            \\ \hline
	2           & 1                 & 21                       & 1                       &95.45\%            \\ \hline
	3           & 2                 & 7                       & 2                       &77.78\%            \\ \hline
	4           & 3                 & 8                       & 0                       &100.00\%            \\ \hline
	5           & 4                 & 8                       & 0                       &100.00\%            \\ \hline
	6           & 5                 & 4                       & 0                       &100.00\%            \\ \hline
	7           & 6                 & 4                       & 0                       &100.00\%            \\ \hline
	8           & 7                 & 13                       & 0                       &100.00\%            \\ \hline
	9           & 8                 & 3                       & 0                       &100.00\%            \\ \hline
	10           & 9                 & 6                       & 0                       &100.00\%            \\ \hline
	11           & A                 & 0                       & 0                       & -            \\ \hline
	12           & B                 & 4                       & 0                       &100.00\%            \\ \hline
	13           & C                 & 1                       & 0                       &100.00\%            \\ \hline
	14           & D                 & 19                       & 1                       &95.00\%            \\ \hline
	15           & E                 & 1                       & 0                       &100.00\%            \\ \hline
	16           & F                 & 0                       & 0                       & -            \\ \hline
	17           & G                 & 1                       & 0                       &100.00\%            \\ \hline
	18           & H                 & 1                       & 0                       &100.00\%            \\ \hline
	19           & I                 & 3                       & 0                       &100.00\%            \\ \hline
	20           & J                 & 3                       & 0                       &100.00\%            \\ \hline
	21           & K                 & 2                       & 0                       &100.00\%            \\ \hline
	22           & L                 & 4                       & 0                       &100.00\%            \\ \hline
	23           & M                 & 1                       & 0                       &100.00\%            \\ \hline
	24           & N                 & 0                       & 0                       & -            \\ \hline
	25           & O                 & 1                       & 0                       &100.00\%            \\ \hline
	26           & P                 & 4                       & 0                       &100.00\%            \\ \hline
	27           & Q                 & 0                       & 2                       &0.00\%            \\ \hline
	28           & R                 & 2                       & 0                       &100.00\%            \\ \hline
	29           & S                 & 2                       & 0                       &100.00\%            \\ \hline
	30           & T                 & 1                       & 0                       &100.00\%            \\ \hline
	31           & U                 & 1                       & 0                       &100.00\%            \\ \hline
	32           & V                 & 1                       & 0                       &100.00\%            \\ \hline
	33           & W                 & 7                       & 0                       &100.00\%            \\ \hline
	34           & X                 & 0                       & 0                       & -            \\ \hline
	35           & Y                 & 1                       & 0                       &100.00\%            \\ \hline
	36           & Z                 & 1                       & 0                       &100.00\%            \\ \hline
\end{longtable}

\noindent Dengan tingkat akurasi pengenalan karakter (\textit{Character Recognition Rate}) sebesar 95.80\%, dapat dilihat pada tabel \ref{tab:hasilklasifikasisel8} bahwa dari 36  karakter yang ada, yang dapat diprediksi dengan benar 100\% adalah sebanyak 29 karakter, dari karakter yang  tersisa, hanya satu karakter yang memiliki akurasi rendah, yaitu huruf Q yang mana dari 2 karakter yang terdapat di data \textit{testing}, tidak ada yang dapat diklasifikasikan dengan benar. Dari hasil pengujian ini dapat  disimpulkan bahwa penggunaan ukuran sel 8 $\times$ 8 piksel sudah dapat menghasilkan akurasi pengenalan karakter yang baik.\\

%\subsection{Pengujian dengan Ukuran Sel 16}
%\noindent Pada bagian ini, pengujian akan dilakukan dengan menggunakan ukuran sel berukuran 16 $\times$ 16 piksel dan ukuran blok 2 $\times$ 2 sel (32 $\times$ 32 piksel). Jumlah bin yang akan digunakan adalah 4, 6, 9, dan 18. Sedangkan untuk nilai sigma pada metode \textit{SVM} yang akan digunakan adalah 0.01, 0.1, dan 1. Berikut adalah hasil pengujian untuk setiap kombinasi parameter tersebut:
%\begin{longtable}[c]{|c|c|c|}
%	\caption{Hasil Pengujian dengan ukuran sel 16 $\times$ 16 piksel}
%	\label{tab:HasilPengujianSel16}\\
%	\hline
%	\begin{tabular}[c]{@{}c@{}}Parameter\\ (CellSize, NumBins, Sigma)\end{tabular} & CRR     & OVR     \\ \hline
%	\endhead
%	%
%	(16, 4, 0.01)                                                                   & 13.06\% & 0\% \\ \hline
%	(16, 4, 0.1)                                                                    & 18.75\% & 0\%     \\ \hline
%	(16, 4, 1)                                                                      & 84.09\% & 17.24\%     \\ \hline
%	(16, 6, 0.01)                                                                   & 13.06\% & 0\%  \\ \hline
%	(16, 6, 0.1)                                                                    & 22.72\% & 0\%     \\ \hline
%	(16, 6, 1)                                                                      & 84.09\% & 13.79\%     \\ \hline
%	(16, 9, 0.01)                                                                   & 13.06\% & 0\%  \\ \hline
%	(16, 9, 0.1)                                                                    & 28.97\% & 0\%     \\ \hline
%	(16, 9, 1)                                                                      & {\color[HTML]{FE0000} 86.36\%} & {\color[HTML]{FE0000} 27.58\%}     \\ \hline
%	(16, 18, 0.01)                                                                  & 13.06\% & 0\%  \\ \hline
%	(16, 18, 0.1)                                                                   & 23.29\% & 0\%     \\ \hline
%	(16, 18, 1)                                                                     & 85.79\% & 24.13\%     \\ \hline
%\end{longtable}
%\begin{longtable}[c]{|r|r|r|r|r|}
%	\caption{Pengujian dengan ukuran sel 16 $\times$ 16 piksel}
%	\label{tab:HasilPengujianSel16}\\
%	\hline
%	\multicolumn{3}{|c|}{Parameter} &                                &                                \\ \cline{1-3}
%	CellSize   & NumBins   & Sigma  & \multicolumn{1}{c|}{\multirow{-2}{*}{CRR}} & \multicolumn{1}{c|}{\multirow{-2}{*}{OVR}} \\ \hline
%	\endhead
%	%
%	16          & 4         & 0.01   & 13.98\%                        & 0\%                            \\ \hline
%	16          & 4         & 0.1    & 20.27\%                        & 0\%                            \\ \hline
%	16          & 4         & 1      & 83.91\%                        & 23.80\%                        \\ \hline
%	16          & 6         & 0.01   & 13.98\%                        & 0\%                            \\ \hline
%	16          & 6         & 0.1    & 25.17\%                        & 0\%                            \\ \hline
%	16          & 6         & 1      & 83.21\%                        & 19.04\%                        \\ \hline
%	16          & 9         & 0.01   & 13.98\%                        & 0\%                            \\ \hline
%	16          & 9         & 0.1    & 30.76\%                        & 0\%                            \\ \hline
%	16          & 9         & 1      & {\color[HTML]{FE0000} 86.71\%} & {\color[HTML]{FE0000} 38.09\%} \\ \hline
%	16          & 18        & 0.01   & 13.98\%                        & 0\%                            \\ \hline
%	16          & 18        & 0.1    & 25.87\%                        & 0\%                            \\ \hline
%	16          & 18        & 1      & 86.01\%                        & 33.33\%                        \\ \hline
%\end{longtable}
\begin{adjustbox}{width=1\textwidth}
	\noindent\begin{minipage}{\linewidth}
		\centering\framebox{\includegraphics[width=14cm]{images/HasilPengujianUkuranSel16.png}}
		\captionof{figure}{Hasil Pengujian Kombinasi Parameter dengan Ukuran Sel 16\\}
		\label{fig:HasilPengujianUkuranSel16}
	\end{minipage}
\end{adjustbox}

\noindent Berdasarkan Gambar \ref{fig:HasilPengujianUkuranSel16}, dapat disimpulkan bahwa akurasi pengenalan karakter maksimal yang didapatkan apabila menggunakan ukuran sel 16 $\times$ 16 piksel adalah 86.71\%. Kombinasi parameter yang digunakan untuk mencapai hasil tersebut adalah ukuran sel 16 $\times$ 16 piksel, jumlah \textit{bin} sebanyak 9 sehingga besar setiap \textit{bin} adalah 20 derajat, kemudian nilai \textit{sigma} yang digunakan untuk metode \textit{SVM} adalah 1. Dengan citra karakter masukan berukuran 32 $\times$ 32 piksel. Maka panjang vektor fitur dari \textit{HOG descriptor} yang dihasilkan adalah 9 fitur. Berbeda dengan pengujian sebelumnya, kali ini jumlah fitur yang terlalu sedikit justru akan mengurangi akurasi dari proses pengenalan karakter yang sebelumnya sudah mencapai 95.80\%. 
%dari keseluruhan pengujian yang sudah dilakukan terhadap jumlah sel, dapat disimpulkan bahwa ukuran sel yang terlampau besar ataupun terlampau kecil pada penggunaan metode \textit{HOG} dapat mengurangi kualitas fitur yang dihasilkan sehingga akan berefek terhadap hasil klasifikasi.

\noindent Berdasarkan hasil CRR tertinggi pada Gambar \ref{fig:HasilPengujianUkuranSel16}, dari 143 karakter yang terdeteksi, sebanyak 123 di antaranya dapat diklasifikasikan dengan baik. Sedangkan berdasarkan hasil OVR tertinggi pada Gambar \ref{fig:HasilPengujianUkuranSel16} dari 21 plat nomor yang terdeteksi, 7 plat nomor dapat dikenali dengan baik, hal ini merupakan dampak dari penurunan akurasi pengenalan karakter.

\noindent Tabel \ref{tab:hasilklasifikasisel16} merupakan tabel yang menunjukkan hasil klasifikasi karakter dengan parameter HOG (\textit{CellSize} dan \textit{NumBins}) masing-masing 16 dan 9, dan nilai \textit{sigma} untuk metode \textit{SVM} 1.

\begin{longtable}[c]{|r|r|r|r|r|}
	\caption{Klasifikasi karakter dengan parameter CellSize = 16, NumBins = 9, dan Sigma = 1.0}
	\label{tab:hasilklasifikasisel16}\\
	\hline
	\textbf{No} & \textbf{Karakter} & \textbf{Prediksi Benar} & \textbf{Prediksi Salah} & \textbf{Akurasi} \\ \hline
	\endhead
	1           & 0                 & 2                       & 0                       &100.00\%            \\ \hline
	2           & 1                 & 20                       & 2                       &90.91\%            \\ \hline
	3           & 2                 & 6                       & 3                       &66.67\%            \\ \hline
	4           & 3                 & 8                       & 0                       &100.00\%            \\ \hline
	5           & 4                 & 8                       & 0                       &100.00\%            \\ \hline
	6           & 5                 & 4                       & 0                       &100.00\%            \\ \hline
	7           & 6                 & 4                       & 0                       &100.00\%            \\ \hline
	8           & 7                 & 13                       & 0                       &100.00\%            \\ \hline
	9           & 8                 & 3                       & 0                       &100.00\%            \\ \hline
	10           & 9                 & 6                       & 0                       &100.00\%            \\ \hline
	11           & A                 & 0                       & 0                       & -            \\ \hline
	12           & B                 & 3                       & 1                       &75.00\%            \\ \hline
	13           & C                 & 1                       & 0                       &100.00\%            \\ \hline
	14           & D                 & 19                       & 1                       &95.00\%            \\ \hline
	15           & E                 & 1                       & 0                       &100.00\%            \\ \hline
	16           & F                 & 0                       & 0                       & -            \\ \hline
	17           & G                 & 1                       & 0                       &100.00\%            \\ \hline
	18           & H                 & 1                       & 0                       &100.00\%            \\ \hline
	19           & I                 & 0                       & 3                       &0.00\%            \\ \hline
	20           & J                 & 3                       & 0                       &100.00\%            \\ \hline
	21           & K                 & 2                       & 0                       &100.00\%            \\ \hline
	22           & L                 & 4                       & 0                       &100.00\%            \\ \hline
	23           & M                 & 1                       & 0                       &100.00\%            \\ \hline
	24           & N                 & 0                       & 0                       & -            \\ \hline
	25           & O                 & 1                       & 0                       &100.00\%            \\ \hline
	26           & P                 & 4                       & 0                       &100.00\%            \\ \hline
	27           & Q                 & 0                       & 2                       &0.00\%            \\ \hline
	28           & R                 & 2                       & 0                       &100.00\%            \\ \hline
	29           & S                 & 2                       & 0                       &100.00\%            \\ \hline
	30           & T                 & 1                       & 0                       &100.00\%            \\ \hline
	31           & U                 & 1                       & 0                       &100.00\%            \\ \hline
	32           & V                 & 1                       & 0                       &100.00\%            \\ \hline
	33           & W                 & 0                       & 7                       &0.00\%            \\ \hline
	34           & X                 & 0                       & 0                       & -            \\ \hline
	35           & Y                 & 1                       & 0                       &100.00\%            \\ \hline
	36           & Z                 & 1                       & 0                       &100.00\%            \\ \hline
\end{longtable}

\noindent Dengan tingkat akurasi pengenalan karakter (\textit{Character Recognition Rate}) sebesar 86.71\%, dapat dilihat pada tabel \ref{tab:hasilklasifikasisel16} bahwa dari 36  karakter yang ada, yang dapat diprediksi dengan benar 100\% adalah sebanyak 25 karakter, terdapat juga karakter yang memiliki akurasi rendah, yaitu huruf I, huruf Q, dan huruf W. Apabila dibandingkan dengan hasil pengujian yang sebelumnya, dengan menggunakan ukuran sel 16 piksel justru malah mengurangi akurasi dari pengenalan karakter dan yang tadinya karakter tersebut sudah dapat dikenali dengan baik (huruf I dan huruf W) malah menjadi tidak bisa diklasifikasikan dengan benar sama sekali (akurasi 0\% untuk kedua karakter tersebut). Dari pengujian ini dapat  disimpulkan bahwa penggunaan ukuran sel 16 $\times$ 16 piksel dapat menghasilkan akurasi pengenalan karakter yang baik namun belum optimal.\\

\subsection{Pengujian Kombinasi Parameter Campuran}
\noindent Dari hasil pengujian kombinasi parameter pada subbab sebelumnya, akan didapatkan kombinasi parameter yang paling optimal untuk setiap karakter. Kombinasi parameter inilah yang akan digunakan dalam skenario pengujian, tujuan dari skenario ini adalah untuk mengetahui apakah penggabungan kombinasi parameter dapat meningkatkan tingkat akurasi pengenalan karakter. Kombinasi parameter yang paling optimal untuk karakter-karakter selain huruf D dan huruf Q adalah ukuran sel 8 $\times$ 8 piksel, jumlah \textit{bin} sebanyak 18, dan nilai sigma sebesar 0.1. Khusus untuk huruf D dan huruf Q, kombinasi parameter yang digunakan adalah ukuran sel 2 $\times$ 2 piksel, jumlah \textit{bin} sebanyak 4, dan nilai sigma sebesar 0.01. Untuk ukuran blok yang digunakan adalah sebesar 2 $\times$ 2 sel. Berikut adalah hasil klasifikasi karakter untuk penggunaan kombinasi parameter campuran:

\begin{longtable}[c]{|r|r|r|r|r|}
	\caption{Klasifikasi karakter dengan kombinasi parameter campuran}
	\label{tab:hasilklasifikasiselCampuran}\\
	\hline
	\textbf{No} & \textbf{Karakter} & \textbf{Prediksi Benar} & \textbf{Prediksi Salah} & \textbf{Akurasi} \\ \hline
	\endhead
	1           & 0                 & 2                       & 0                       &100.00\%            \\ \hline
	2           & 1                 & 21                       & 1                       &95.45\%            \\ \hline
	3           & 2                 & 7                       & 2                       &77.78\%            \\ \hline
	4           & 3                 & 8                       & 0                       &100.00\%            \\ \hline
	5           & 4                 & 8                       & 0                       &100.00\%            \\ \hline
	6           & 5                 & 4                       & 0                       &100.00\%            \\ \hline
	7           & 6                 & 4                       & 0                       &100.00\%            \\ \hline
	8           & 7                 & 13                       & 0                       &100.00\%            \\ \hline
	9           & 8                 & 3                       & 0                       &100.00\%            \\ \hline
	10           & 9                 & 6                       & 0                       &100.00\%            \\ \hline
	11           & A                 & 0                       & 0                       & -            \\ \hline
	12           & B                 & 4                       & 0                       &100.00\%            \\ \hline
	13           & C                 & 1                       & 0                       &100.00\%            \\ \hline
	14           & D                 & 20                       & 0                       &100.00\%            \\ \hline
	15           & E                 & 1                       & 0                       &100.00\%            \\ \hline
	16           & F                 & 0                       & 0                       & -            \\ \hline
	17           & G                 & 1                       & 0                       &100.00\%            \\ \hline
	18           & H                 & 1                       & 0                       &100.00\%            \\ \hline
	19           & I                 & 3                       & 0                       &100.00\%            \\ \hline
	20           & J                 & 3                       & 0                       &100.00\%            \\ \hline
	21           & K                 & 2                       & 0                       &100.00\%            \\ \hline
	22           & L                 & 4                       & 0                       &100.00\%            \\ \hline
	23           & M                 & 1                       & 0                       &100.00\%            \\ \hline
	24           & N                 & 0                       & 0                       & -            \\ \hline
	25           & O                 & 1                       & 0                       &100.00\%            \\ \hline
	26           & P                 & 4                       & 0                       &100.00\%            \\ \hline
	27           & Q                 & 2                       & 0                       &100.00\%            \\ \hline
	28           & R                 & 2                       & 0                       &100.00\%            \\ \hline
	29           & S                 & 2                       & 0                       &100.00\%            \\ \hline
	30           & T                 & 1                       & 0                       &100.00\%            \\ \hline
	31           & U                 & 1                       & 0                       &100.00\%            \\ \hline
	32           & V                 & 1                       & 0                       &100.00\%            \\ \hline
	33           & W                 & 7                       & 0                       &100.00\%            \\ \hline
	34           & X                 & 0                       & 0                       & -            \\ \hline
	35           & Y                 & 1                       & 0                       &100.00\%            \\ \hline
	36           & Z                 & 1                       & 0                       &100.00\%            \\ \hline
\end{longtable}

\begin{adjustbox}{width=1\textwidth}
	\noindent\begin{minipage}{\linewidth}
		\centering\framebox{\includegraphics[width=6cm]{images/OVRKombinasiParameter.png}}
		\captionof{figure}{Hasil OVR Pengujian Kombinasi Parameter Gabungan\\}
		\label{fig:HasilOVRKombinasiParameterGabungan}
	\end{minipage}
\end{adjustbox}

\begin{adjustbox}{width=1\textwidth}
	\noindent\begin{minipage}{\linewidth}
		\centering\framebox{\includegraphics[width=6cm]{images/CRRKombinasiParameter.png}}
		\captionof{figure}{Hasil CRR Pengujian Kombinasi Parameter Gabungan\\}
		\label{fig:HasilCRRKombinasiParameterGabungan}
	\end{minipage}
\end{adjustbox}

\noindent Dari Gambar \ref{fig:HasilCRRKombinasiParameterGabungan} \textit{Character Recognition Rate} yang didapatkan adalah sebesar 97.90\%. Tingkat akurasi \textit{Character Recognition Rate} (CRR) meningkat 2\% terhadap hasil CRR kombinasi parameter dengan ukuran sel 8 $\times$ 8 piksel, jumlah \textit{bin} 18, dan nilai sigma 0.1. Dari 143 karakter yang terdeteksi, sebanyak 140 di antaranya dapat diklasifikasikan dengan baik, hal ini juga meningkatkan jumlah plat nomor yang berhasil dikenali. Sedangkan berdasarkan hasil OVR pada Gambar \ref{fig:HasilOVRKombinasiParameterGabungan} dari 21 plat nomor yang terdeteksi 19 plat nomor dapat dikenali dengan baik.\\

\section{Analisis Pengujian}
\noindent Setiap skenario pengujian yang dilakukan pada subbab 4.3.1 dan 4.3.2 diukur dengan melihat tingkat akurasi ada \textit{Character Recognition Rate} dan \textit{Overall Performance} yang mengukur performa keseluruhan aplikasi. Berdasarkan hasil pengujian pada Gambar \ref{fig:HasilPengujianUkuranSel2} sampai Gambar \ref{fig:HasilPengujianUkuranSel16}, tingginya tingkat akurasi OVR sangat bergantung kepada tingginya tingkat akurasi CRR, semakin tinggi tingkat akurasi dari CRR maka tingkat akurasi dari OVR juga akan semakin meningkat. Pada beberapa hasil pengujian terlihat juga OVR menunjukkan angka 0\%, hal ini disertai dengan rendahnya tingkat akurasi dari CRR pada hasil pengujian tersebut (biasanya terjadi ketika angka CRR berada di bawah 51\%). Hal ini disebabkan karena OVR menghitung jumlah plat yang dikenali dengan benar terhadap jumlah keseluruhan plat yang terdeteksi oleh sistem. Kondisi plat disebut dikenali dengan benar adalah ketika keseluruhan hasil prediksi karakter untuk plat tersebut sama persis dengan karakter asli pada citra plat, satu saja hasil prediksi karakter yang tidak sama akan membuat plat tersebut tidak dapat dimasukkan ke dalam kategori plat yang dapat dikenali dengan benar. Oleh karena itu, jika aplikasi tidak dapat mengenali setiap karakter dengan baik (angka akurasi CRR rendah) maka nilai OVR yang didapat pun akan rendah dan bahkan 0.

\noindent Tingkat akurasi CRR yang didapatkan bergantung terhadap komposisi parameter yang digunakan. Berdasarkan hasil CRR hasil pengujian pada Gambar \ref{fig:HasilPengujianUkuranSel2} sampai Gambar \ref{fig:HasilPengujianUkuranSel16} tingkat akurasi CRR optimal yang didapatkan cenderung meningkat ketika menggunakan kombinasi parameter dengan ukuran sel 2 $\times$ 2 piksel sampai dengan 8 $\times$ 8 piksel. Namun ketika menggunakan kombinasi parameter dengan ukuran sel 16 $\times$ 16 piksel, tingkat akurasi CRR optimal yang didapatkan menurun. Dari pola tersebut dapat disimpulkan untuk menghasilkan tingkat CRR yang optimal diperlukan komposisi parameter yang tepat.

\noindent Untuk nilai sigma pada metode \textit{SVM} yang digunakan agar tingkat akurasi CRR yang dihasilkan optimal, kecenderungan yang didapatkan dari hasil pengujian pada Gambar \ref{fig:HasilPengujianUkuranSel2} sampai Gambar \ref{fig:HasilPengujianUkuranSel16} adalah cenderung mengikuti besarnya ukuran sel, semakin ukuran selnya besar, maka semakin tinggi nilai sigma yang dibutuhkan.

\noindent Kombinasi parameter terbaik untuk setiap karakter dapat ditentukan dengan melihat hasil klasifikasi dari setiap hasil pengujian kombinasi parameter, yaitu pada tabel \ref{tab:hasilklasifikasisel2} sampai dengan tabel \ref{tab:hasilklasifikasisel16}. Dari tabel-tabel tersebut didapati bahwa kombinasi parameter dengan ukuran sel 8 $\times$ 8 piksel dapat mengklasifikasikan mayoritas karakter dengan baik terkecuali huruf Q dan huruf D. Kedua huruf tersebut memiliki akurasi klasifikasi yang paling baik ketika diuji menggunakan kombinasi parameter dengan ukuran 2 $\times$ 2 piksel, jumlah \textit{bin} 4, dan nilai sigma sebesar 0.01. Untuk huruf Q, ketika ia diuji menggunakan kombinasi parameter selain dengan ukuran sel 2 $\times$ 2 piksel, maka huruf tersebut akan selalu tertukar dengan angka 0.

\begin{adjustbox}{width=1\textwidth}
	\noindent\begin{minipage}{\linewidth}
		\centering\framebox{\includegraphics[width=2cm]{images/HurufQ.png}}
		\begin{center}
			(a)
		\end{center}
	\end{minipage}
\end{adjustbox}

\begin{adjustbox}{width=1\textwidth}
	\noindent\begin{minipage}{\linewidth}
		\centering\framebox{\includegraphics[width=2cm]{images/Angka0.png}}
		\begin{center}
			(b)
		\end{center}
		\captionof{figure}{Contoh citra (a) Huruf Q (b) Angka 0\\}
		\label{fig:hurufQangka0}
	\end{minipage}
\end{adjustbox}

\noindent Dari Gambar \ref{fig:hurufQangka0} dapat terlihat bahwa karakteristik objek yang dimiliki huruf Q dengan angka 0 memang mirip sehingga ketika dilakukan ekstraksi fitur terhadap kedua karakter tersebut tidak menutup kemungkinan fitur yang dihasilkan mirip. Oleh karena itu penggunaan ukuran sel 2 $\times$ 2 piksel memang tepat jika digunakan untuk mengklasifikasikan huruf Q dikarenakan ukuran sel 2 $\times$ 2 piksel dapat menangkap detail dari objek huruf Q dengan lebih baik dibandingkan ukuran sel lainnya.

\noindent Untuk huruf D, dari keseluruhan citra huruf D yang terdeteksi sistem, terdapat 1 citra yang awalnya dapat diklasifikasi dengan benar ketika menggunakan ukuran sel 2 $\times$ piksel, namun menjadi misklasifikasi ketika menggunakan ukuran sel lain. Setelah dilakukan analisis lebih lanjut, satu citra yang misklasifikasi tersebut ternyata memiliki masalah pada hasil segmentasi karakternya. Hal itu juga disebabkan karena citra karakter berasalh dari citra plat yang bermasalah (citra plat miring dan tidak tersegmentasi horizontal dengan baik). Hal ini tentunya akan berdampak sangat besar terhadap hasil segmentasi vertikalnya. Citra plat hasil \textit{preprocessing} dan segmentasi dan citra karakter D yang misklasifikasi dapat dilihat pada Gambar \ref{fig:CitraPlatMisklasifikasiD} dan Gambar \ref{fig:CitraHasilSegmentasiD}.

\begin{adjustbox}{width=1\textwidth}
	\noindent\begin{minipage}{\linewidth}
		\centering\framebox{\includegraphics[width=8cm]{images/D1192PS.png}}
		\captionof{figure}{Citra Plat asal karakter D yang misklasifikasi\\}
		\label{fig:CitraPlatMisklasifikasiD}
	\end{minipage}
\end{adjustbox}

\begin{adjustbox}{width=1\textwidth}
	\noindent\begin{minipage}{\linewidth}
		\centering\framebox{\includegraphics[width=1cm]{images/DMisklasifikasi.png}}
		\captionof{figure}{Citra hasil segmentasi karakter huruf D\\}
		\label{fig:CitraHasilSegmentasiD}
	\end{minipage}
\end{adjustbox}\\

\noindent Untuk meningkatkan akurasi pengenalan huruf Q dan huruf D, digunakanlah dua kombinasi parameter seperti yang sudah disinggung pada subbab 4.3.2. Dan dari hasil pengujian yang didapatkan terbukti dapat meningkatkan akurasi pengenalan terhadap huruf Q dan huruf D. Dari hasil pengujian tersebut dapat disimpulkan bahwa kombinasi parameter yang berbeda dapat diterapkan pada sistem untuk mencapai hasil yang lebih optimal.\\

\section{Analisis Kesalahan}
\noindent Berdasarkan keseluruhan hasil skenario pengujian, didapatkan bahwa akurasi pengenalan karakter paling tinggi dihasilkan dari penggabungan kombinasi parameter yang paling optimal untuk setiap karakter. Namun berdasarkan tabel \ref{tab:hasilklasifikasiselCampuran}, terdapat tiga citra karakter yang mengalami misklasifikasi. Karakter-karakter tersebut adalah angka 1 sebanyak satu citra dan angka 2 sebanyak dua citra. Hasil misklasifikasi karakter-karakter tersebut dapat dilihat pada tabel \ref{tab:analisiskesalahan}

\begin{longtable}[c]{|r|r|r|}
	\caption{Hasil klasifikasi karakter yang misklasifikasi}
	\label{tab:analisiskesalahan}\\
	\hline
	No & Karakter & Hasil Klasifikasi \\ \hline
	\endfirsthead
	%
	\multicolumn{3}{c}%
	{{\bfseries Table \thetable\ continued from previous page}} \\
	\hline
	No & Karakter & Hasil Klasifikasi \\ \hline
	\endhead
	%
	1  & 1        & D                 \\ \hline
	2  & 2        & D                 \\ \hline
\end{longtable}
 
%\noindent Berdasarkan hasil pengujian, didapatkan bahwa komposisi parameter yang menghasilkan akurasi paling tinggi dihasilkan oleh komposisi parameter dengan ukuran sel 8 $\times$ 8 piksel dan akurasi paling rendah dihasilkan oleh komposisi parameter dengan ukuran sel 2 $\times$ 2 piksel. Namun meskipun menghasilkan akurasi yang tertinggi dan banyak karakter yang jika dibandingkan dengan hasil pengujian menggunakan ukuran sel 2 $\times$ 2 piksel akurasinya meningkat, masih terdapat juga karakter yang kenaikan akurasinya tidak terlalu signifikan, ada karakter yang akurasinya berkurang, dan bahkan ada karakter yang tadinya dapat diklasifikasi dengan baik dan malah menjadi tidak dapat diklasifikasi dengan benar sama sekali. Karakter-karakter tersebut adalah angka 2, huruf D, dan huruf Q. Penjelasan dari kejadian ini diilustrasikan pada tabel \ref{tab:analisiskesalahan}.
%\begin{longtable}[c]{|c|c|c|c|c|c|c|c|}
%	\caption{Perbandingan hasil akurasi karakter pada ukuran sel 2 dengan ukuran sel 8}
%	\label{tab:analisiskesalahan}\\
%	\hline
%	\multirow{2}{*}{No} & \multirow{2}{*}{Karakter} & \multicolumn{3}{c|}{Ukuran Sel 2} & \multicolumn{3}{c|}{Ukuran Sel 8} \\ \cline{3-8} 
%	&  & Benar & Salah & Akurasi & Benar & Salah & Akurasi \\ \hline
%	\endfirsthead
%	%
%	\multicolumn{8}{c}%
%	{{\bfseries Table \thetable\ continued from previous page}} \\
%	\hline
%	\multirow{2}{*}{No} & \multirow{2}{*}{Karakter} & \multicolumn{3}{c|}{Ukuran Sel 2} & \multicolumn{3}{c|}{Ukuran Sel 8} \\ \cline{3-8} 
%	&  & Benar & Salah & Akurasi & Benar & Salah & Akurasi \\ \hline
%	\endhead
%	%
%	1 & 2 & 6 & 3 & 66.67\% & 7 & 2 & 77.78\% \\ \hline
%	2 & D & 20 & 0 & 100\% & 19 & 1 & 95\% \\ \hline
%	3 & Q & 2 & 0 & 100\% & 0 & 2 & 0\% \\ \hline
%\end{longtable}
\noindent Seperti yang ditunjukkan pada tabel \ref{tab:analisiskesalahan}, keseluruhan angka 1 dan 2 yang misklasifikasi dianggap sebagai karakter huruf D. Kombinasi parameter yang digunakan untuk mengklasifikasikan karakter angka 1 dan angka 2 adalah kombinasi parameter yang paling optimal untuk kedua karakter tersebut berdasarkan hasil pengujian pada subbab 4.3.1, yaitu ukuran sel 8 $\times$ 8 piksel, jumlah \textit{bin} 18, dan nilai sigma 0.1. Setelah ditelusuri, ternyata karakter angka 1 dan satu dari dua angka 2 yang misklasifikasi berasal dari citra plat yang sama, yaitu citra plat pada Gambar \ref{fig:CitraPlatMisklasifikasiD}, sedangkan citra angka 2 yang satunya berasal dari citra plat seperti yang ditunjukkan pada Gambar \ref{fig:CitraPlatMisklasifikasi2}.

\begin{adjustbox}{width=1\textwidth}
	\noindent\begin{minipage}{\linewidth}
		\centering\framebox{\includegraphics[width=8cm]{images/D1278LI.png}}
		\captionof{figure}{Citra Plat asal karakter angka 2 yang misklasifikasi\\}
		\label{fig:CitraPlatMisklasifikasi2}
	\end{minipage}
\end{adjustbox}

\noindent Pada gambar \ref{fig:CitraPlatMisklasifikasiD} dan Gambar \ref{fig:CitraPlatMisklasifikasi2}, dapat dilihat bahwa citra plat yang didapatkan tidak lurus dan tidak tersegmentasi horizontal dengan baik. Hal ini akan mempengaruhi hasil segmentasi vertikal ketika mencari area karakter dari plat tersebut. Hasil segmentasi untuk setiap karakter dapat dilihat pada Gambar \ref{fig:CitraHasilSegmentasiKarakterMisklasifikasi}.

\begin{adjustbox}{width=1\textwidth}
	\noindent\begin{minipage}{\linewidth}
		\centering\framebox{\includegraphics[width=1cm]{images/1Misklasifikasi.png}}
		\begin{center}
			(a)
		\end{center}
	\end{minipage}
\end{adjustbox}

\begin{adjustbox}{width=1\textwidth}
	\noindent\begin{minipage}{\linewidth}
		\centering\framebox{\includegraphics[width=1cm]{images/2Misklasifikasi1.png}}
		\begin{center}
			(b)
		\end{center}
	\end{minipage}
\end{adjustbox}

\begin{adjustbox}{width=1\textwidth}
	\noindent\begin{minipage}{\linewidth}
		\centering\framebox{\includegraphics[width=1cm]{images/2Misklasifikasi2.png}}
		\begin{center}
			(c)
		\end{center}
		\captionof{figure}{Citra hasil segmentasi karakter misklasifikasi (a) Citra angka 1 dengan satu objek lain (b) Citra angka 2 dengan dua objek lain (c) Citra angka 2 dengan satu objek lain \\}
		\label{fig:CitraHasilSegmentasiKarakterMisklasifikasi}
	\end{minipage}
\end{adjustbox}

%\noindent Seperti yang ditunjukkan pada tabel \ref{tab:analisiskesalahan} di atas, angka 2 kenaikan akurasinya tidak begitu signifikan dikarenakan hanya berbeda 1 hasil klasifikasi yang benar, huruf D mengalami penurunan akurasi dikarenakan terdapat karakter yang tadinya dapat diklasifikasi dengan benar dan ketika ukuran sel diganti menjadi 8 $\times$ 8 piksel, karakter tersebut malah menjadi tidak terklasifikasi dengan benar. Sedangkan untuk huruf Q, awalnya karakter tersebut dapat dikenali dengan baik namun setelah menggunakan ukuran sel 8 $\times$ 8 piksel, karakter tersebut menjadi tidak dapat terklasifikasi dengan baik.
%\noindent Pada karakter angka 2, ketika dilihat pada hasil dari \textit{Confusion Matrix} ternyata ketika menggunakan ukuran sel 2 $\times$ 2 piksel, terdapat 3 karakter yang teridentifikasi sebagai karakter huruf D, dan ketika menggunakan ukuran sel 8 $\times$ 8 piksel berubah menjadi teridentifikasi sebagai karakter huruf O. Namun ada satu citra karakter angka 2 yang tadinya tidak dapat diklasifikasikan dengan benar ketika menggunakan ukuran sel 2 $\times$ 2 piksel dan menjadi dapat diklasifikan dengan benar ketika menggunakan ukuran sel 8 $\times$ 8 piksel. Gambar \ref{fig:CitraPlat} merupakan citra plat hasil \textit{preprocessing} dari karakter angka 2 tersebut.\\

%\begin{adjustbox}{width=1\textwidth}
%	\noindent\begin{minipage}{\linewidth}
%		\centering\framebox{\includegraphics[width=8cm]{images/D1623RB.png}}
%		\captionof{figure}{Citra Plat\\}
%		\label{fig:CitraPlat}
%	\end{minipage}
%\end{adjustbox}\\
%\\
%\noindent Dari citra plat dapat dilihat bahwa tidak ada masalah dengan bentuk dari plat itu sendiri (tidak miring dan berhasil tersegmentasi secara horizontal dengan baik). Jadi apabila terdapat kesalahan klasifikasi hal tersebut dikarenakan dari fitur yang dihasilkan dari \textit{HOG} dengan ukuran sel 2 $\times$ 2 piksel tidak cukup baik dibandingkan dengan \textit{HOG} ukuran sel 8 $\times$ 8 piksel. Sedangkan untuk 2 citra karakter angka 2 yang tetap tidak dapat diklasifikasikan dengan baik, berikut adalah citra plat tempat citra karakter berasal.\\
%
%\noindent Dari citra dapat dilihat bahwa kedua citra karakter angka 2 yang misklasifikasi tersebut berasal dari citra plat yang bermasalah (citra plat miring dan tidak tersegmentasi horizontal dengan baik). Hal ini tentunya akan berdampak sangat besar kepada hasil segmentasi vertikalnya. Berikut adalah hasil segmentasi vertikal angka 2 dari kedua plat tersebut.\\\\
%\begin{adjustbox}{width=1\textwidth}
%	\noindent\begin{minipage}{\linewidth}
%		\centering\framebox{\includegraphics[width=1cm]{images/2Misklasifikasi1.png}}
%		\begin{center}
%			(a)
%		\end{center}
%	\end{minipage}
%\end{adjustbox}\\\\\\
%\begin{adjustbox}{width=1\textwidth}
%	\noindent\begin{minipage}{\linewidth}
%		\centering\framebox{\includegraphics[width=1cm]{images/2Misklasifikasi2.png}}
%		\begin{center}
%			(b)
%		\end{center}
%		\captionof{figure}{Citra hasil segmentasi karakter angka 2, (a) citra dengan tiga objek di dalamnya (b) citra dengan dua objek di dalamnya\\}
%		\label{fig:CitraHasilSegmentasi2}
%	\end{minipage}
%\end{adjustbox}\\
%\\
\noindent Dari citra hasil segmentasi pada Gambar \ref{fig:CitraHasilSegmentasiKarakterMisklasifikasi} dapat dilihat bahwa terdapat objek lain selain objek utama pada hasil segmentasi vertikal ketiga citra karakter angka 1 dan citra karakter angka 2 yang misklasifikasi. Hal ini akan berdampak pada hasil fitur yang akan dihasilkan dari metode \textit{HOG} nantinya. Oleh karena itu, bisa disimpulkan kondisi citra plat yang kurang baik yang menyebabkan kedua karakter ini tidak dapat diidentifikasi.

%\noindent Pada tabel \ref{tab:analisiskesalahan} juga dapat dilihat terdapat satu citra karakter D yang awalnya dapat diklasifikasi dengan baik ketika menggunakan ukuran sel 2 $\times$ 2 piksel namun menjadi misklasifikasi ketika menggunakan ukuran sel 8 $\times$ 8 piksel. Setelah melihat hasil klasifikasi pada tabel \textit{Confusion Matrix} ternyata karakter D tersebut teridentifikasi sebagai karakter huruf O. Dan setelah ditelusuri, ternyata citra karakter huruf D tersebut berasal dari citra plat yang sama dengan citra plat yang menyebabkan 2 citra karakter angka 2 misklasfikasi, yaitu citra plat D1192PS seperti yang pernah ditampilkan sebelumnya (Gambar \ref{fig:CitraPlatMisklasifikasi} atas). Adapun hasil segmentasi vertikal dari karakter huruf D tersebut dapat dilihat pada Gambar \ref{fig:CitraHasilSegmentasiD}.
%
%\noindent Dari citra dapat dilihat bahwa terdapat objek garis yang merupakan garis tepi dari plat nomor. Hal inilah yang akan berdampak pada hasil fitur yang akan dihasilkan dari metode \textit{HOG} nantinya. Untuk kasus ini, citra karakter ini bisa diklasifikasikan sebagai karakter huruf D ketika menggunakan ukuran sel 2 $\times$ 2 piksel dan tidak dapat diklasifikan dengan benar ketika menggunakan ukuran sel 8 $\times$ 8 piksel sehingga bisa diasumsikan bahwa peranan ukuran sel berpengaruh dalam hasil klasifikasi, secara teori ukuran sel yang lebih kecil memang memiliki kelebihan untuk menangkap detil yang lebih baik jika dibandingkan dengan ukuran sel yang besar, walaupun dengan demikian ukuran fitur yang dihasilkan oleh ukuran sel yang kecil akan jauh lebih besar dibandingkan fitur dari ukuran sel yang lebih besar.
%
%\noindent Asumsi di atas didukung dengan melihat data berikutnya pada tabel \ref{tab:analisiskesalahan}. Terdapat karakter huruf Q yang ketika menggunakan ukuran sel 2 $\times$ 2 piksel dapat diklasifikasikan dengan baik tetapi ketika menggunakan ukuran sel 8 $\times$ 8 piksel malah terjadi misklasifikasi. Jika dilihat pada tabel hasil klasifikasi dengan ukuran sel di atas 2, memang karakter huruf Q tersebut sudah tidak dapat diklasifikasi dengan benar lagi dan selalu tertukar dengan karakter angka 0.
%
%\noindent Dari hasil analisis kesalahan ini dapat disimpulkan beberapa faktor yang berpengaruh terhadap hasil fitur dari \textit{HOG} yang nantinya juga akan berpengaruh terhadap akurasi dari klasifikasi karakter menggunakan SVM, yaitu faktor eksternal seperti masalah pada citra yang kurang baik (citra plat miring) dan faktor internal seperti ukuran sel dan jika diamati dari hasil pengujian jumlah bin cenderung mengikuti ukuran sel, semakin besar ukuran sel maka untuk menghasilkan akurasi yang maksimal pada ukuran sel tersebut diperlukan jumlah bin yang lebih besar juga, begitu pun dengan nilai sigma yang digunakan untuk metode \textit{Support Vector Machine} sebagai metode klasifikasinya.
\newpage
	%\setcounter{page}{1}
	%%-----------------------------------------------------------------------------%
\chapter{PENUTUP}
%-----------------------------------------------------------------------------%

%
\vspace{4.5pt}
\noindent Bab ini berisi kesimpulan yang dilandasi oleh penelitian dan pengujian yang telah dilakukan, serta dilengkapi dengan saran yang dapat untuk perkembangan ke depan.\\

\section{Kesimpulan}
\noindent Pengujian dari penerapan metode \textit{Histogram of Oriented Gradient} dan \textit{Support Vector Machine} menghasilkan beragam hasil. Hasil pengujian-pengujian tersebut menghasilkan kesimpulan sebagai berikut:
\begin{enumerate}
\item Hasil akurasi terbaik pengenalan plat nomor kendaraan dengan menggunakan metode \textit{Histogram of Oriented Gradient} dan \textit{Support Vector Machine} adalah 97.90\% yang dicapai ketika menggunakan ukuran sel 8 $\times$ 8 piksel yang digabungkan dengan ukuran sel 2 $\times$ 2 piksel untuk karakter-karakter yang kurang dapat dikenali dengan baik ketika menggunakan ukuran sel 8 $\times$ 8 piksel, ukuran blok 2 $\times$ 2 sel (16 $\times$ 16 piksel untuk ukuran sel 8 $\times$ 8 piksel dan 4 $\times$ 4 piksel untuk ukuran sel 2 $\times$ 2 piksel), jumlah \textit{bin} sebanyak 18 untuk ukuran sel 8 $\times$ 8 piksel dan 4 untuk ukuran sel 2 $\times$ 2 piksel, dan nilai sigma untuk metode \textit{Support Vector Machine} sebesar 0.1 untuk ukuran sel 8 $\times$ 8 piksel dan 0.01 untuk ukuran sel 2 $\times$ 2 piksel.

\item Karakteristik objek berpengaruh terhadap ukuran sel HOG yang digunakan, untuk karakter yang memiliki karakteristik yang umum tidak akan mengalami masalah ketika menggunakan ukuran sel yang besar, sedangkan untuk karakter yang memiliki karakteristik khusus memerlukan ukuran sel yang lebih kecil. Hal ini dapat terlihat dari akurasi karakter huruf C, E, G, H, K, L, M, O, T, V, Y, dan huruf Z yang memiliki rata-rata akurasi mencapai 100\% untuk setiap ukuran sel. Sedangkan huruf Q merupakan huruf khusus yang baru dapat dikenali dengan baik ketika menggunakan ukuran sel 2 $\times$ 2 piksel, selain ukuran sel tersebut, huruf Q akan selalu tertukar dengan karakter huruf O. 
%\item  Metode \textit{HOG} memerlukan komposisi parameter yang tepat agar dapat menghasilkan fitur yang baik, ukuran sel yang terlalu kecil ataupun terlalu besar dapat mengurangi tingkat akurasi. Metode ini dapat menghasilkan akurasi yang baik walaupun citra latih karakternya berukuran kecil yaitu 32 $\times$ 32 piksel, asalkan diimbangi dengan pemilihan komposisi parameter yang tepat.

\item Nilai sigma pada metode \textit{Support Vector Machine} yang dapat menghasilkan akurasi yang optimal bergantung terhadap ukuran sel yang digunakan. Kecenderungannya adalah semakin besar ukuran sel yang digunakan, maka nilai sigma yang diperlukan semakin besar dan sebaliknya.\\
%\item Metode \textit{Support Vector Machine} yang digunakan sebagai metode untuk klasifikasi karakter ternyata dapat menggunakan fitur dari metode \textit{HOG} dan dapat menghasilkan akurasi yang baik dengan akurasi tertinggi sebesar 94.88\% dengan nilai sigma yang digunakan adalah 0.1. Semakin besar ukuran sel yang digunakan pada metode \textit{HOG}, maka nilai sigma untuk mendapatkan akurasi yang optimal untuk ukuran sel yang digunakan semakin besar.

%\item Terdapat beberapa karakter yang dapat diklasifikasi dengan baik (rata-rata akurasi mencapai 100\%) walaupun komposisi dari parameter \textit{HOG} dan nilai sigma untuk metode \textit{Support Vector Machine} yang digunakan beragam, karakter tersebut adalah huruf C, E, G, H, K, L, M, O, T, V, Y, dan huruf Z.

%\item Terdapat juga karakter dengan akurasi klasifikasi yang rendah (rata-rata akurasi hanya 25\%), huruf tersebut adalah huruf Q, huruf ini hanya dapat diklasifikasi dengan baik ketika menggunakan ukuran sel 2 $\times$ 2 piksel, ukuran blok 2 $\times$ 2 sel (4 $\times$ 4 piksel) dan nilai sigma untuk metode \textit{Support Vector Machine} sebesar 0.01.\\
\end{enumerate}

\section{Saran}
\noindent Saran untuk pengembangan sistem pengenalan plat nomor kendaraan adalah:
\begin{enumerate}
\item Ekstraksi fitur dapat dicoba menggunakan metode-metode seperti \textit{zoning}, metode yang berbasis \textit{moments} seperti \textit{Geometric Moments, Zernike Moment,} dan \textit{Orthogonal Fourier-Mellin Moments}, atau metode seperti \textit{Discrete Wavelet Transform} dan \textit{Restricted Boltzman Machine} yang dapat menangani permasalahan translasi dan rotasi terhadap karakter.

\item Untuk pengembangan lebih lanjut, pengenalan karakter dapat dicoba dengan menggunakan metode \textit{Deep Learning} seperti misalnya metode \textit{Convolutional Neural Network} yang dapat melakukan ekstraksi fitur dan klasifikasi dalam satu metode.

%\item Metode \textit{HOG} masih memerlukan metode tambahan untuk melakukan proses klasifikasi (misal SVM, K-NN, dan lain-lain), untuk metode \textit{Support Vector Machine} sendiri memiliki kelemahan dalam hal sulitnya memilih parameter yang tepat untuk suatu permasalahan, parameter yang bagus untuk suatu kasus belum tentu bagus juga untuk kasus lainnya, untuk pengembangan lebih lanjut, pengenalan karakter dapat dicoba menggunakan metode \textit{Deep Learning} yang dapat melakukan ekstraksi fitur dan klasifikasi dalam satu metode seperti \textit{Convolutional Neural Network}.
%\item Masih sering terjadi misklasifikasi pada karakter yang berbentuk mirip seperti misalnya huruf D dengan angka 0, huruf B dengan angka 8, huruf S dengan angka 5, huruf A dengan angka 4, dan beberapa karakter lainnya. Oleh karena itu diperlukan menambahkan metode klasifikasi atau metode ekstraksi fitur yang dapat mengambil fitur pasangan karakter yang mirip tersebut dengan lebih baik atau menggunakan metode \textit{deep learning} seperti \textit{Convolutional Neural Network} yang disebut memiliki kemampuan yang baik untuk melakukan ekstraksi fitur																																											.
\end{enumerate}

\newpage
	
	
	
	\pagenumbering{roman}
	\setcounter{page}{\thesavepage}
	\fancypagestyle{plain}{%
		\renewcommand{\headrulewidth}{0pt}%
		\fancyhf{}%
		\fancyfoot[c]{\thepage}%
	}
	\cfoot{\thepage}
	\rfoot{}
	
	% Merubah Nama Bibliografi ke Daftar Pustaka
	\renewcommand{\bibname}{DAFTAR REFERENSI}
	\phantomsection
	\addcontentsline{toc}{chapter}{DAFTAR REFERENSI}
	% Daftar Pustaka
	\begin{thebibliography}{7}

\bibitem{tabrizi}
{Tabrizi, S. S., Cavus, N. (2016). A hybrid KNN-SVM model for Iranian license plate recognition. \emph{Procedia Computer Science, 102}, pp. 588-594.}

\bibitem{gou2014}
{Gou, C., Wang, K., Yu, Z., Xie, H. (2014, October). License plate recognition using MSER and HOG based on ELM. In \emph{Proceedings of 2014 IEEE International Conference on Service Operations and Logistics, and Informatics} (pp. 217-221). IEEE.}

\bibitem{gou2016}
{Gou, C., Wang, K., Yao, Y., Li, Z. (2016). Vehicle license plate recognition based on extremal regions and restricted Boltzmann machines. \emph{IEEE Transactions on Intelligent Transportation Systems, 17}(4), 1096-1107.}

\bibitem{rasheed}
{Rasheed, S., Naeem, A., Ishaq, O. (2012, October). Automated number plate recognition using hough lines and template matching. In \emph{Proceedings of the World Congress on Engineering and Computer Science} (Vol. 1, pp. 24-26).}

\bibitem{gonzalez}
{R. C. Gonzalez and R. E. Woods, \emph{Digital Image Processing}, $2^{nd}$ ed. Prentice Hall, 1992.}

\bibitem{shih}
{Shih, F.Y. (2010). Image Processing And Pattern Recognition Fundamentals and Techniques, Hoboken:John Wiley \& Sons, Inc.}

\bibitem{comvis}
{Computer Vision CITS4240: Lab 6, [Online]. Available: http://teaching.csse.uwa.edu.au/units/CITS4240/Labs/Lab6/lab6.html [Accessed: 16-Apr-2019].}

\bibitem{rectangle}
{Diagonals of a Rectangle, [Online]. Available: https://www.mathopenref.com/rectanglediagonals.html [Accessed: 13-May-2019].}

\bibitem{oechsle}
{Oechsle, Olly (2012). Finding Straight Lines with the Hough Transform, [Online]. Available: http://vase.essex.ac.uk/software/HoughTransform/ [Accessed: 16-Apr-2019].}

\bibitem{svm}
{Ma, Y., \& Guo, G. (Eds.). (2014). \textit{Support Vector Machines Applications} (pp. 23-26). New York: Springer.}

\bibitem{nugroho}
{Nugroho, A., Wardhani, K.R.R. (2011). Aplikasi Sistem Pembaca Plat Nomor Mobil Menggunakan Pengolahan Citra dan Metode Learning Vector Quantization.}

\bibitem{polri}
{Peraturan Kapolri Nomor 5 tahun 2012 tentang Registrasi dan Identifikasi Kendaraan Bermotor, [Online]. Available: http://kepri.polri.go.id/pid/wp-content/uploads/2019/01/PERATURAN-KAPOLRI-NOMOR-5-TAHUN-2012-TENTANG-REGISTRASI-DAN-IDENTIFIKASI-KENDARAAN-BERMOTOR.pdf [Accessed: 22-Apr-2019]}

\bibitem{markham}
{Markham, Kevin (2014). Simple Guide to Confusion Matrix Terminology, [Online]. Available: https://www.dataschool.io/simple-guide-to-confusion-matrix-terminology/ [Accessed: 22-Apr-2019].}

\bibitem{ragb}
{H. K. Ragb and V. K. Asari, \emph{Multi-feature Fusion and PCA Based Approach for Efficient Human Detection}, Applied Imagery Pattern Recognition Workshop (AIRP) IEEE, Washington, DC, USA, 2016, pp. 1-6.}

\bibitem{ashtari}
{Ashtari, A. H., Nordin, M. J., Fathy, M. (2014). An Iranian license plate recognition system based on color features. \emph{IEEE transactions on intelligent transportation systems}, 15(4), 1690-1705.}

\bibitem{gryimage}
{R. Fisher, et all. (2003). Grayscale Images, [Online]. Available: https://homepages.inf.ed.ac.uk/rbf/HIPR2/gryimage.htm [Accessed: 26-Jun-2019]}

%\bibitem{10}
%{R. C. Gonzalez and R. E. Woods, \emph{Digital Image Processing}, $2^{nd}$ ed. Prentice Hall, 1992.}
%\bibitem{12}
%{“CS231n Convolutional Neural Networks for Visual Recognition”, \emph{CS231n}. [Online]. Available: cs231n.github.io/convolutional-networks/. [Accessed: 19-Nov-2017].}
%\bibitem{13}
%{S. Haykin, \emph{Neural Networks and Learning Machines}, $3^{rd}$ ed. Pearson, 2008.}
%\bibitem{14}
%{T. Babb, "How a Kalman filter works, in pictures", \emph{Bzarg}, 2015. [Online]. Available: http://www.bzarg.com/p/how-a-kalman-filter-works-in-pictures/. [Accessed: 11- Dec- 2017].}
%\bibitem{15}
%{G. Zhang, L. Tian, et al., “Robust real-time human perception with depth camera”, \emph{in ECAI 2016: 22nd European Conference on Artificial Intelligence}, 29 August-2 September 2016. IOS Press, 2016, vol. 285, p. 304.}
%\bibitem{16}
%{H. Habibi Aghdam and E. Jahani Heravi, \emph{Guide to Convolutional Neural Networks}, 2017.}
%\bibitem{17}
%{S. Khan, H. Rahmani, S. A. A. Shah, and M. Bennamoun, \emph{A Guide to Convolutional Neural Networks for Computer Vision}, vol. 8, no. 1. 2018.}
%\bibitem{18}
%{Stephanie, “Jaccard Index / Similarity Coefficient'', \emph{Statistics How To}, 2016. [Online]. Available: http://www.statisticshowto.com/jaccard-index/. [Accessed: 20-May-2018].}
%\bibitem{19}
%{J. Jordan, “Setting the learning rate of your neural network”, \emph{Jeremy Jordan}, 2018. [Online]. Available:https://www.jeremyjordan.me/nn-learning-rate/. [Accessed: 22-May-2018].}
%\bibitem{20}
%{A. Sachan, “Zero to Hero: Guide to Object Detection using Deep Learning: Faster R-CNN, YOLO, SSD”, \emph{Learn Machine Learning, AI \& Computer Vision}, 2017. [Online]. Available:http://cv-tricks.com/object-detection/faster-r-cnn-yolo-ssd/. [Accessed: 5-June-2018].}
%\bibitem{21}
%{M. Tyka, “Inceptionism: Going Deeper into Neural Networks”, \emph{Google AI Blog}, 2015. [Online]. Available:https://ai.googleblog.com/2015/06/inceptionism-going-deeper-into-neural.html. [Accessed: 13-June-2018].}

\end{thebibliography}
	
	%
	% Lampiran 
	%
	
	\setcounter{originalpagenumber}{\number\value{page}}%
	\setcounter{page}{0}
	\pagenumbering{arabic}
	
	\onehalfspacing
	\begin{appendix}
		%\include{Lampiran}
	\end{appendix}
	
	\pagenumbering{arabic}% 
	\setcounter{page}{\number\value{originalpagenumber}}
	\clearpage
	\phantomsection \addcontentsline{toc}{chapter}{LAMPIRAN}
	
\end{document}