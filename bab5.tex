%-----------------------------------------------------------------------------%
\chapter{PENUTUP}
%-----------------------------------------------------------------------------%

%
\vspace{4.5pt}
\noindent Bab ini berisi kesimpulan yang dilandasi oleh penelitian dan pengujian yang telah dilakukan, serta dilengkapi dengan saran yang dapat untuk perkembangan ke depan.\\

\section{Kesimpulan}
\noindent Kesimpulan dari sistem pengenalan plat nomor kendaraan adalah:
\begin{enumerate}
\item Hasil akurasi pengenalan karakter yang tertinggi yaitu sebesar 94.88\% dicapai ketika menggunakan ukuran sel 8 $\times$ 8 piksel, ukuran blok 2 $\times$ 2 sel (16 $\times$ 16 piksel), jumlah bin sebanyak 18, dan nilai sigma sebesar 0.1.
\item Metode \textit{HOG} memerlukan komposisi parameter yang tepat agar dapat menghasilkan fitur yang baik, ukuran sel yang terlalu kecil ataupun besar dapat mengurangi tingkat akurasi.
\item Metode \textit{HOG} dapat menghasilkan akurasi yang baik walaupun citra latih karakternya berukuran kecil yaitu 32 $\times$ 32 piksel, asal diimbangi dengan pemilihan komposisi parameter yang tepat.
\item Bagus tidaknya fitur \textit{HOG} yang dihasilkan bergantung daripada ukuran sel dan jumlah bin yang digunakan.\\
\end{enumerate}

\subsection{Saran}
\noindent Saran untuk pengembangan sistem pengenalan plat nomor kendaraan adalah:
\begin{enumerate}
\item Metode \textit{HOG} masih memerlukan metode tambahan untuk melakukan proses klasifikasi (misal SVM, K-NN, dan lain-lain), untuk pengembangan lebih lanjut, pengenalan karakter dapat dicoba menggunakan metode \textit{Deep Learning} yang dapat melakukan ekstraksi fitur dan klasifikasi dalam satu metode seperti \textit{Convolutional Neural Network}.
\item Masih sering terjadi misklasifikasi pada karakter yang berbentuk mirip seperti misalnya huruf D dengan angka 0, huruf B dengan angka 8, huruf S dengan angka 5, huruf A dengan angka 4, dan beberapa karakter lainnya. Oleh karena itu diperlukan menambahkan metode klasifikasi atau metode ekstraksi fitur yang dapat mengambil fitur pasangan karakter yang mirip tersebut dengan lebih baik.
\end{enumerate}

\newpage