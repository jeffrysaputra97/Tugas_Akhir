%-----------------------------------------------------------------------------%
\chapter{PENUTUP}
%-----------------------------------------------------------------------------%

%
\vspace{4.5pt}
\noindent Bab ini berisi kesimpulan yang dilandasi oleh penelitian dan pengujian yang telah dilakukan, serta dilengkapi dengan saran yang dapat untuk perkembangan ke depan.

\section{Kesimpulan}
\noindent Kesimpulan dari sistem pendeteksi dan pelacakan manusia pada video RGB-D adalah:
\begin{enumerate}
\item Informasi pada citra kedalaman memungkinkan untuk mengenali manusia pada kondisi pencahayaan relatif redup.
\item Penggunaan data belajar yang berkualitas dengan sudut pandang yang baik akan meningkatkan akurasi pada tahap mendeteksi manusia.
\item Penerapan metode \textit{median filtering} sebagai tahap \textit{preprocessing} belum tentu meningkatkan akurasi deteksi, tergantung pada \textit{dataset} yang digunakan. Hasil akurasi tertinggi dengan menggunakan \textit{median filtering}, pada \textit{dataset outdoor} memiliki akurasi lebih tinggi sekitar 1\%, sementara pada \textit{dataset clothing store} hasil akurasi lebih rendah sekitar 2\% daripada tanpa menggunakan \textit{median filtering}.
\item Metode \textit{Convolutional Neural Network} dapat mengenali manusia walaupun citra berukuran kecil yaitu 48 $\times$ 64 piksel.
\item Hasil akurasi \textit{Convolutional Neural Network} dipengaruhi oleh penentuan arsitektur metode. Akurasi tertinggi dari hasil pelatihan metode \textit{Convolutional Neural Network} pada \textit{dataset outdoor 56} adalah 99,7\%, akurasi tertinggi pada pengujian \textit{dataset outdoor 31} adalah 90,4\%, dan pada \textit{dataset outdoor 54} adalah 89,38\%. Sementara akurasi tertinggi dari hasil pelatihan pada \textit{dataset clothing store} adalah 89,58\% dan akurasi tertinggi pada pengujian adalah 81,13\%.
\item Akurasi yang dihasilkan prediksi Kalman \textit{filter} dipengaruhi oleh nilai $r_{k}$ dan jumlah iterasi adaptasi yang dilakukan sebelum melakukan prediksi. Hasil akurasi tertinggi dari pengujian adalah 58,17\% dengan melakukan iterasi adaptasi sebanyak 3 kali dan prediksi sebanyak 2 kali.
\item Bila dibandingkan pengujian gabungan \textit{Cascade Classifier} dan \textit{Convolutional Neural Network} dengan pengujian keseluruhan \textit{Cascade Classifier}, \textit{Convolutional Neural Network}, dan Kalman \textit{filter}, pengujian gabungan \textit{Cascade Classifier} dan \textit{Convolutional Neural Network} menghasilkan rata-rata akurasi lebih tinggi 1,33\%, rata-rata \textit{precision} lebih rendah 1,11\%, dan rata-rata \textit{recall} lebih tinggi 4,44\% daripada pengujian keseluruhan dengan menambahkan Kalman \textit{filter}. 
\item Untuk pengujian keseluruhan menggunakan \textit{Cascade Classifier}, \textit{Convolutional Neural Network}, dan Kalman \textit{filter} menghasilkan rata-rata akurasi sebesar 77,7\%, rata-rata \textit{precision} sebesar 79,44\%, dan rata-rata \textit{recall} 91,11\% dengan waktu proses rata-rata per citra adalah 5,388 milidetik.
\item Efisiensi dari Kalman \textit{filter} pada pengujian menggunakan \textit{median filter} mempercepat komputasi 14,41\% sementara pada pengujian tanpa menggunakan \textit{median filter} mempercepat komputasi hingga 41,71\%
\end{enumerate}

\subsection{Saran}
\noindent Saran untuk pengembangan sistem pendeteksi dan pelacakan manusia pada video RGB-D adalah:
\begin{enumerate}
\item Untuk mendeteksi dan melacak manusia menggunakan informasi citra kedalaman sebaiknya memiliki sudut pandang yang frontal dari depan atau dari atas sehingga citra memiliki kualitas yang lebih baik.
\item Daripada menggunakan data belajar negatif dengan posisi acak, sebaiknya menggunakan data belajar negatif yang mencakup keseluruhan daerah pada citra.
\item Metode \textit{Convolutional Neural Network} dapat dikembangkan agar dapat mengenali lokasi manusia secara langsung dan memiliki waktu proses yang lebih cepat seperti menggunakan metode \textit{Spatial Pyramid Pooling Network} (SPP-Net), Fast R-CNN, Faster R-CNN, \textit{You Look Only Once} (YOLO), \textit{Single Shot MultiBox Detector} (SSD), atau \textit{Multi Task Cascaded Neural Network} (MTCNN) \cite{20}.
\end{enumerate}

\newpage